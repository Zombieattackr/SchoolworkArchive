\documentclass{article}

\usepackage{amsmath}
\usepackage{amssymb}
\usepackage{bm}
\usepackage{graphicx}
\usepackage{epstopdf}
\DeclareGraphicsRule{.tif}{png}{.png}{`convert #1 `basename #1 .tif`.png}
\usepackage{color}
\usepackage{pdfsync}
\pagestyle{plain}

\textheight 10.5 true in
\textwidth 8 true in
\hoffset -1.5 true in
\voffset -1.5 true in
\mathsurround=2pt
\parskip=2pt


\begin{document}

Intro to Elctronics S 2024 Crib Sheet Exam 1+2 Hayden Fuller
\begin{large}
\\Basic:
\\Voltage divider: $V_2=\frac{R_2}{R_1+R_2}$ \hfill Current divider: $I_2=I\frac{G_2}{G_1+G_2}$
\\Capacitor: $Z_C=\frac{1}{jwC}$, $I=C\frac{dV}{dt}$, $E=\frac{1}{2}CV^2$ \hfill Inductor: $Z_L=jwC$, $V=C\frac{dI}{dt}$, $E=\frac{1}{2}LI^2$
\\LCR: resonance: $w=\frac{1}{\sqrt{LC}}$
\\Superposition: Short voltage, open current, analyze the thing, add.
\\Voltage to current: $I_{IS}=V_{VS}$, $R_{CS}=\frac{1}{R{VS}}$
\\OP AMP:      Inverting Amp: $A=\frac{V_{out}}{V_{in}}=-\frac{R_F}{R_in}$
\\Summing Inverting Amp: $V_{out}=-R_F(\frac{V_1}{R_1}+\frac{V_2}{R_2}+\frac{V_3}{R_3})$
\\Subtraction: Inverting amp with voltage divider on plus: $V_{out}=V_{plus}\frac{R_G}{R_{in,plus}+R_G}\cdot\frac{R_{in,minus}+R_F}{R_{in,minus}}-V_{minux}\frac{R_F}{R_{in,minus}}$
\\H response: $H(w)=\frac{V_{out}}{V_{in}}$
\\Integrator: $R_F$ is a capacitor, $V_{out}=-\frac{1}{R_{in}C}\int_0^tV_{in}dt$
\\Detailed amplification: op amp
\\diff: $V_{out}=A_{diff}(V_+-V_-)$
\\CM: $V_{out}=A_{CM}(V_++V_-)$
\\total amplification: $V_{out}=A_{diff}(V_+-V_-)+A_{CM}(V_++V_-)$
\\CMRR: common mode rejection ratio: $\frac{A_{diff}}{A_{CM}}$, $20\log(CMRR)dB$
\\Gain Bandwidth Product: Gain$\cdot$Bandwidth, $A\cdot f$
\\Constant gain at low frequency, decreases linearly because slew rate at high frequency
\\knee frequency: $w=\frac{1}{RC}=\frac{1}{T}=\frac{1}{2\pi RC}$, $=>$ $H(w)=\frac{1}{\sqrt{2}}$, $-3dB$
\\above knee frequncy: $H(w)=\frac{1}{wRC}$, $-20dB$ per decade
\\high pass: $H(w)=wRC$, $20dB$ per decade, slope multiplied by N stages.
\\Slew rate: $=dV_{out}/dt$
\\ex: $SR=1V/us$, pulse $0-2V$, $T=2us$,  1V triangle wave
\\good sine wave: $SR>=2\pi f V_0$
\\DIODES:     forward bias, apply +V to holes, push holes(p) and electrons(n) together, get conduction
\\IV: $I=I_0(e^{\frac{V}{V_t}}-1)$
\\Thermal voltage: $V_t=\frac{kT}{e}=26mV$
\\Threashold voltage: $V_{th}~=0.7V$
\\Reverse saturation current: $I_0$ around $10^{-10}$
\\For $V>>V_t$, $I~=I_0e^{\frac{V}{V_t}}$
\\Differential resistance: $r_D=\frac{V_t}{I}$ (I from above equation)
\\Basic DR circuit: $V_B=IR+V_D$, $I=I_D=I_0e^{\frac{V}{V_t}}$
\\Graphically solve: $V_B=I_0e^{\frac{V}{V_t}}R+V_D$, $V_B-V_D=I_0e^{\frac{V}{V_t}}$
\\Approximate analytic solution: $V_D=V_{th}=0.7V$, $I=\frac{V_B-0.7}{R}$
\\Linearization: $r_D=\frac{V_t}{I_D}$, can treat as this resistor for an AC signal, just account for DC offset with superposition.
\\find I with DC (probably $\frac{V-0.7}{R}$), and $r_D=\frac{.026}{I}$, then add AC signal with $r_D$
\\Rectification: $P=V_{th}I$   AM: through diode and R and past C to filter out frequency, left with audio.Also LED and solar cell.
\\ZENER DIODE:  width of depletion region changed with doping
\\Breakdown: at $V=V_{breakdown}=V_Z$
\\IV: turn on at $V_{th}=0.7$, reverse turn on at $-V_Z$, typically used backwards, so flip V, expect $+$   $|<$  $-$
\\$V<V_Z$ $=>$ $r_Z=\frac{dV}{dI}=\infty$ OR $V>=V_Z$, $r_Z=\frac{dV}{dI}=0$
\\$I~=I_0e^{\frac{V-V_Z}{V_t}}$
\\Basic ZDR circuit:
\\Graphical: $V_{Bat}-IR=V_Z(I)$
\\Approx analytical assume $V_Z$=constant, $I=\frac{V_B-V_Z}{R}$
\\Voltage stabalization: $V_{out}=V_Z=$constant, wastes power, must have significant $R_Load$
\\Voltage clipper: two parallel to load, facing away, V clipped at $V_Z+V_{th}$
\\Voltage shifter: one series, cuts $V_Z$ off of input
\newpage\end{large}
Intro to Elctronics S 2024 Crib Sheet Exam 2 Hayden Fuller\begin{large}
\\pnp, E+ arrow in top, holes flow in E, most make it to C, need current E to B, $V_{BE}<0$
\\npn, E+ arrow out bot, holes flow out E, most come from C, need current B to E, $V_{BE}>0$
\\common base:, in in E, out out C, grounded out B
\\common emitter: in out B, out out C, grounded in E
\\common collector: in out B, out in E, grounded out C
\\$I_C=\alpha I_E$
\\$I_C=\beta I_B$
\\$I_E=(\beta+1)I_B$
\\$\beta = \alpha/(1-\alpha)$
\\base $<<$ diffusion length
\\Equivalent:
\\Current from B, $\alpha 'I_C'$ to E, $\alpha I_E'$ to C
\\Diode to B, $I_E '$ from E,  $I_C '$ from C  ($\alpha ' < \alpha$)
\\simplifies to:
\\diode E to B, current out B, current B to C $\alpha I_E=\beta I_B$
\\npn is just flipped, $I_C=\alpha I_E$, diode B to E
\\graph output characteristic $I_C$ vs $V_{CE}$, 
\\steep start goes (nearlly) flat, $I_C=\beta I_B$, flat section goes up with $I_B$
\\linear(small signal): output, current, effect $\propto$ input, voltage, cause: RCL, non: Diode, BJT, FET
\\small signal can count VCC as GND
\\C-B-E, equiv current $\alpha I_E$ - Diode $I_E$, near opp, $r_E+DC$, small signal $r_E$, $r_E=V_t/I_E=26mV/I_E$
\\Common E: E grounded, B input, C output and RC to VCC, 
\\$I_C=-V_{CE}/R_C + V_{CC}/R_C=V_{CC}/R_C-1/R_C$ ($I_C=0$ at $V_CC$)
\\solve $I_C=\beta I_B=-V_{CE}/R_C + V_{CC}/R_C$
\\saturation fully on at top left, $V_{CE}\approx 0.1 - 0.2V$
\\forward activev linear in middle, $I_C=\beta I_B$, $V_{BE}\approx 0.7V$
\\cutoff off at bottom, $V_{BE}<0.7V$, $I_B=I_C=0$, CE impedance =$\infty$
\\Quiescent point around middle of forward active region/dynamic range, 
\\frequently middle of $I_C=V_{CC}/R_C-1/R_C$, determined by DC bias network
\\find B current, find C current, find CE voltage (output voltage)
\\Primative common E
\\Voltage amplification $A_{VOC}=V_{in}/V_{out}=-\beta i_B R_C / i_E r_E = -\beta i_B R_C / (\beta+1)i_B+r_E\approx -R_C/r_E$
\\Current amplification $A_{ISC}=i_{out}/i_{in}$
\\Input impedance $Z_{in}=V_{in}/I_{in}=r_Ei_E/i_B=r_E(\beta+1)i_B=\beta r_E$
\\Output impedance $Z_{out}=V_{out}/I_{out}=R_CI_C/I_C=R_C$
\\Emitter follower, RB to VCC, RE to GND, $V_{out}=V_{in}-V_{BE}$, 
\\$Z_{in}\approx\beta R_E$(high), $Z_{out}\approx R_E$(low), $A_{VOC}\approx 1$(low), $A_{ISC}<=\beta$(high)
\\Non-idealities of BJTs
\\Finite output impedance (current source), $r_{CE}\neq\infty$
\\output characteristic graph isnt flat, but sloped with $r_{CE}$$^{-1}$, 
\\meets at Early Voltage $\approx$-10V to -100V
\\$\beta=\Delta I_C/\Delta I_B$ increases with $V_{CE}$ due to base width modulation
\\$I_C=\alpha I_0 e^{V_{BE}/V_t} (1+(V_{CE}/V_{Early}))$
\\slope=$r_{CE}$$^{-1}=I_C/(V_{Early}+V_{CE})$
\\Breakdown of CB at high VCE, reverse breakdown of neglected BC diode comes into play, out char slopes increase
\\100V for Si, increasing order Ge, Si, GaAs, SiC, GaN, C
\newpage\end{large}
Intro to Elctronics S 2024 Crib Sheet Exam 2 Hayden Fuller\begin{large}
\\DIFFERENTIAL AMPLIFIER, used in op amp and Emitter Coupled Logic
\\C is outputs and $R_C$ to $+V_{CC}$,  B is inputs,  E connects and shares $R_E$ to GND
\\if input 1 is the input, output 1 is inverting
\\single input, single output, oposite sides, $A=R_C/(2 r_E)$, differential output gives $A=R_C/r_E$
\\differential input and output $A=2R_C/r_E$; Q-point $V_{CE}\approx V_{CC}-\beta I_{B1} (R_C+2R_E)$
\\$\Delta I_{C1} = -\Delta I_{C2}$, so total current is constant, $I_{RE}$ is constant, V is constant, Virtual Ground
\\Input Output Gain   $r_E=V_t/I_E=26mV/I_E$
\\single single $R_C/(2r_E)$
\\single differential $R_C/(r_E)$
\\differential differential $2R_C/(r_E)$
\\FLIP FLOP, SRAM(fast, power hungry, small)
\\C connects to $V_{CC}$ through $R_C$ and to other B through $R_B$.  $P\approx V_{CC}$$^2/R_C$
\\$R_C$ limits current flow while on, $R_B$ stops other transistor from getting turned on.
\\ASTABLE FLIP FLOP, Multivibrator $R_B>>R_C$
\\same as above, but replace $R_B$ with capacitor $C$, and have resistor $R_B$ from the other side (B) of $C$ to $V_{CC}$
\\Assume T1 on T2 off, $\tau=RC=R_{B1}C_1$, $T=2\tau=2R_BC$, change f with $R_B$ (could do C)
\\C2 charged to VCC-0.7V, C1 charge slow RB1, T2 VBE hit 0.7 turn on, T2 VC=0, C2 charge brings VB1 negative
\\CURRENT MIRROR
\\T1 and T2 share E GND and B, T1 has CB shorted and $R_C$ above that, load on T2 C 
\\shared $V_{BE}$ so turned on the same, $I_{RC}=(\beta+2)I_B$, $I_{Load}=\beta I_B$, $I_{Load}\approx I_{RC}$
\\CLASS A AMPLIFIERS, basic, quality, high power draw
\\E GND, B in, B $R_B$ to $V_{CC}$, C out, C $R_C$ to $V_{CC}$
\\Q point power $P=I_C * V_{CE}$, constant power draw, good pre amp
\\CLASS B AMPLIFIER, push pull, crossover distortion, low power, 
\\signal split bwtween two, top is NPN (normal), bottom is PNP (inverting), stacked between VCC+ and VCC-
\\less than 0.7V is lost, crossover distortion
\\CLASS AB AMPLIFIER, better push pull, low distortion, med-low power,
\\add caps to isolate each input and resistors + diodes to bias (1.4V between them), both have $V_{BE}=0.7V$
\\CLASS C AMPLIFIER, pulse amplifier
\\CLASS D AMPLIFIER, digital audio quality, low power
\\triangle wave at 100kHz, gets compared to audio. Audio higher, high FET drives high, triangle higher, low fet drives low, gets smothed to 20kHz range by inductor
\\DARLINGTON TRANSITOR
\\C is shared, E connects to next B. $V_{BE}=1.4V$, $\beta=\beta^2$, $V_{CE}=V_{BE2}+V_{CE1}>=0.9V$
\\T2 doesn't saturate, can be good for high speed, used for RF



\end{large}

\end{document}




























