\documentclass{article}

\usepackage{amsmath}
\usepackage{amssymb}
\usepackage{bm}
\usepackage{graphicx}
\usepackage{epstopdf}
\DeclareGraphicsRule{.tif}{png}{.png}{`convert #1 `basename #1 .tif`.png}
\usepackage{color}
\usepackage{pdfsync}
\pagestyle{plain}

\textheight 10.5 true in
\textwidth 8 true in
\hoffset -1.5 true in
\voffset -1.5 true in
\mathsurround=2pt
\parskip=2pt


\begin{document}

Intro to Elctronics S 2024 Crib Sheet Exam 1+2 Hayden Fuller
\begin{large}
\\Basic:
\\Voltage divider: $V_2=\frac{R_2}{R_1+R_2}$ \hfill Current divider: $I_2=I\frac{G_2}{G_1+G_2}$
\\Capacitor: $Z_C=\frac{1}{jwC}$, $I=C\frac{dV}{dt}$, $E=\frac{1}{2}CV^2$ \hfill Inductor: $Z_L=jwC$, $V=C\frac{dI}{dt}$, $E=\frac{1}{2}LI^2$
\\LCR: resonance: $w=\frac{1}{\sqrt{LC}}$
\\Superposition: Short voltage, open current, analyze the thing, add.
\\Voltage to current: $I_{IS}=V_{VS}$, $R_{CS}=\frac{1}{R{VS}}$
\\OP AMP:      Inverting Amp: $A=\frac{V_{out}}{V_{in}}=-\frac{R_F}{R_in}$
\\Summing Inverting Amp: $V_{out}=-R_F(\frac{V_1}{R_1}+\frac{V_2}{R_2}+\frac{V_3}{R_3})$
\\Subtraction: Inverting amp with voltage divider on plus: $V_{out}=V_{plus}\frac{R_G}{R_{in,plus}+R_G}\cdot\frac{R_{in,minus}+R_F}{R_{in,minus}}-V_{minux}\frac{R_F}{R_{in,minus}}$
\\H response: $H(w)=\frac{V_{out}}{V_{in}}$
\\Integrator: $R_F$ is a capacitor, $V_{out}=-\frac{1}{R_{in}C}\int_0^tV_{in}dt$
\\Detailed amplification: op amp
\\diff: $V_{out}=A_{diff}(V_+-V_-)$
\\CM: $V_{out}=A_{CM}(V_++V_-)$
\\total amplification: $V_{out}=A_{diff}(V_+-V_-)+A_{CM}(V_++V_-)$
\\CMRR: common mode rejection ratio: $\frac{A_{diff}}{A_{CM}}$, $20\log(CMRR)dB$
\\Gain Bandwidth Product: Gain$\cdot$Bandwidth, $A\cdot f$
\\Constant gain at low frequency, decreases linearly because slew rate at high frequency
\\knee frequency: $w=\frac{1}{RC}=\frac{1}{T}=\frac{1}{2\pi RC}$, $=>$ $H(w)=\frac{1}{\sqrt{2}}$, $-3dB$
\\above knee frequncy: $H(w)=\frac{1}{wRC}$, $-20dB$ per decade
\\high pass: $H(w)=wRC$, $20dB$ per decade, slope multiplied by N stages.
\\Slew rate: $=dV_{out}/dt$
\\ex: $SR=1V/us$, pulse $0-2V$, $T=2us$,  1V triangle wave
\\good sine wave: $SR>=2\pi f V_0$
\\DIODES:     forward bias, apply +V to holes, push holes(p) and electrons(n) together, get conduction
\\IV: $I=I_0(e^{\frac{V}{V_t}}-1)$
\\Thermal voltage: $V_t=\frac{kT}{e}=26mV$
\\Threashold voltage: $V_{th}~=0.7V$
\\Reverse saturation current: $I_0$ around $10^{-10}$
\\For $V>>V_t$, $I~=I_0e^{\frac{V}{V_t}}$
\\Differential resistance: $r_D=\frac{V_t}{I}$ (I from above equation)
\\Basic DR circuit: $V_B=IR+V_D$, $I=I_D=I_0e^{\frac{V}{V_t}}$
\\Graphically solve: $V_B=I_0e^{\frac{V}{V_t}}R+V_D$, $V_B-V_D=I_0e^{\frac{V}{V_t}}$
\\Approximate analytic solution: $V_D=V_{th}=0.7V$, $I=\frac{V_B-0.7}{R}$
\\Linearization: $r_D=\frac{V_t}{I_D}$, can treat as this resistor for an AC signal, just account for DC offset with superposition.
\\find I with DC (probably $\frac{V-0.7}{R}$), and $r_D=\frac{.026}{I}$, then add AC signal with $r_D$
\\Rectification: $P=V_{th}I$   AM: through diode and R and past C to filter out frequency, left with audio.Also LED and solar cell.
\\ZENER DIODE:  width of depletion region changed with doping
\\Breakdown: at $V=V_{breakdown}=V_Z$
\\IV: turn on at $V_{th}=0.7$, reverse turn on at $-V_Z$, typically used backwards, so flip V, expect $+$   $|<$  $-$
\\$V<V_Z$ $=>$ $r_Z=\frac{dV}{dI}=\infty$ OR $V>=V_Z$, $r_Z=\frac{dV}{dI}=0$
\\$I~=I_0e^{\frac{V-V_Z}{V_t}}$
\\Basic ZDR circuit:
\\Graphical: $V_{Bat}-IR=V_Z(I)$
\\Approx analytical assume $V_Z$=constant, $I=\frac{V_B-V_Z}{R}$
\\Voltage stabalization: $V_{out}=V_Z=$constant, wastes power, must have significant $R_Load$
\\Voltage clipper: two parallel to load, facing away, V clipped at $V_Z+V_{th}$
\\Voltage shifter: one series, cuts $V_Z$ off of input
\newpage\end{large}
Intro to Elctronics S 2024 Crib Sheet Exam 2 Hayden Fuller\begin{large}
\\pnp, E+ arrow in top, holes flow in E, most make it to C, need current E to B, $V_{BE}<0$
\\npn, E+ arrow out bot, holes flow out E, most come from C, need current B to E, $V_{BE}>0$
\\common base:, in in E, out out C, grounded out B
\\common emitter: in out B, out out C, grounded in E
\\common collector: in out B, out in E, grounded out C
\\$I_C=\alpha I_E$
\\$I_C=\beta I_B$
\\$I_E=(\beta+1)I_B$
\\$\beta = \alpha/(1-\alpha)$
\\base $<<$ diffusion length
\\Equivalent:
\\Current from B, $\alpha 'I_C'$ to E, $\alpha I_E'$ to C
\\Diode to B, $I_E '$ from E,  $I_C '$ from C  ($\alpha ' < \alpha$)
\\simplifies to:
\\diode E to B, current out B, current B to C $\alpha I_E=\beta I_B$
\\npn is just flipped, $I_C=\alpha I_E$, diode B to E
\\graph output characteristic $I_C$ vs $V_{CE}$, 
\\steep start goes (nearlly) flat, $I_C=\beta I_B$, flat section goes up with $I_B$
\\linear(small signal): output, current, effect $\propto$ input, voltage, cause: RCL, non: Diode, BJT, FET
\\small signal can count VCC as GND
\\C-B-E, equiv current $\alpha I_E$ - Diode $I_E$, near opp, $r_E+DC$, small signal $r_E$, $r_E=V_t/I_E=26mV/I_E$
\\Common E: E grounded, B input, C output and RC to VCC, 
\\$I_C=-V_{CE}/R_C + V_{CC}/R_C=V_{CC}/R_C-1/R_C$ ($I_C=0$ at $V_CC$)
\\solve $I_C=\beta I_B=-V_{CE}/R_C + V_{CC}/R_C$
\\saturation fully on at top left, $V_{CE}\approx 0.1 - 0.2V$
\\forward activev linear in middle, $I_C=\beta I_B$, $V_{BE}\approx 0.7V$
\\cutoff off at bottom, $V_{BE}<0.7V$, $I_B=I_C=0$, CE impedance =$\infty$
\\Quiescent point around middle of forward active region/dynamic range, 
\\frequently middle of $I_C=V_{CC}/R_C-1/R_C$, determined by DC bias network
\\find B current, find C current, find CE voltage (output voltage)
\\Primative common E
\\Voltage amplification $A_{VOC}=V_{in}/V_{out}=-\beta i_B R_C / i_E r_E = -\beta i_B R_C / (\beta+1)i_B+r_E\approx -R_C/r_E$
\\Current amplification $A_{ISC}=i_{out}/i_{in}$
\\Input impedance $Z_{in}=V_{in}/I_{in}=r_Ei_E/i_B=r_E(\beta+1)i_B=\beta r_E$
\\Output impedance $Z_{out}=V_{out}/I_{out}=R_CI_C/I_C=R_C$
\\Emitter follower, RB to VCC, RE to GND, $V_{out}=V_{in}-V_{BE}$, 
\\$Z_{in}\approx\beta R_E$(high), $Z_{out}\approx R_E$(low), $A_{VOC}\approx 1$(low), $A_{ISC}<=\beta$(high)
\\Non-idealities of BJTs
\\Finite output impedance (current source), $r_{CE}\neq\infty$
\\output characteristic graph isnt flat, but sloped with $r_{CE}$$^{-1}$, 
\\meets at Early Voltage $\approx$-10V to -100V
\\$\beta=\Delta I_C/\Delta I_B$ increases with $V_{CE}$ due to base width modulation
\\$I_C=\alpha I_0 e^{V_{BE}/V_t} (1+(V_{CE}/V_{Early}))$
\\slope=$r_{CE}$$^{-1}=I_C/(V_{Early}+V_{CE})$
\\Breakdown of CB at high VCE, reverse breakdown of neglected BC diode comes into play, out char slopes increase
\\100V for Si, increasing order Ge, Si, GaAs, SiC, GaN, C
\newpage\end{large}
Intro to Elctronics S 2024 Crib Sheet Exam 2 Hayden Fuller\begin{large}
\\DIFFERENTIAL AMPLIFIER, used in op amp and Emitter Coupled Logic
\\C is outputs and $R_C$ to $+V_{CC}$,  B is inputs,  E connects and shares $R_E$ to GND
\\if input 1 is the input, output 1 is inverting
\\single input, single output, oposite sides, $A=R_C/(2 r_E)$, differential output gives $A=R_C/r_E$
\\differential input and output $A=2R_C/r_E$; Q-point $V_{CE}\approx V_{CC}-\beta I_{B1} (R_C+2R_E)$
\\$\Delta I_{C1} = -\Delta I_{C2}$, so total current is constant, $I_{RE}$ is constant, V is constant, Virtual Ground
\\Input Output Gain   $r_E=V_t/I_E=26mV/I_E$
\\single single $R_C/(2r_E)$
\\single differential $R_C/(r_E)$
\\differential differential $2R_C/(r_E)$
\\FLIP FLOP, SRAM(fast, power hungry, small)
\\C connects to $V_{CC}$ through $R_C$ and to other B through $R_B$.  $P\approx V_{CC}$$^2/R_C$
\\$R_C$ limits current flow while on, $R_B$ stops other transistor from getting turned on.
\\ASTABLE FLIP FLOP, Multivibrator $R_B>>R_C$
\\same as above, but replace $R_B$ with capacitor $C$, and have resistor $R_B$ from the other side (B) of $C$ to $V_{CC}$
\\Assume T1 on T2 off, $\tau=RC=R_{B1}C_1$, $T=2\tau=2R_BC$, change f with $R_B$ (could do C)
\\C2 charged to VCC-0.7V, C1 charge slow RB1, T2 VBE hit 0.7 turn on, T2 VC=0, C2 charge brings VB1 negative
\\CURRENT MIRROR
\\T1 and T2 share E GND and B, T1 has CB shorted and $R_C$ above that, load on T2 C 
\\shared $V_{BE}$ so turned on the same, $I_{RC}=(\beta+2)I_B$, $I_{Load}=\beta I_B$, $I_{Load}\approx I_{RC}$
\\CLASS A AMPLIFIERS, basic, quality, high power draw
\\E GND, B in, B $R_B$ to $V_{CC}$, C out, C $R_C$ to $V_{CC}$
\\Q point power $P=I_C * V_{CE}$, constant power draw, good pre amp
\\CLASS B AMPLIFIER, push pull, crossover distortion, low power, 
\\signal split bwtween two, top is NPN (normal), bottom is PNP (inverting), stacked between VCC+ and VCC-
\\less than 0.7V is lost, crossover distortion
\\CLASS AB AMPLIFIER, better push pull, low distortion, med-low power,
\\add caps to isolate each input and resistors + diodes to bias (1.4V between them), both have $V_{BE}=0.7V$
\\CLASS C AMPLIFIER, pulse amplifier
\\CLASS D AMPLIFIER, digital audio quality, low power
\\triangle wave at 100kHz, gets compared to audio. Audio higher, high FET drives high, triangle higher, low fet drives low, gets smothed to 20kHz range by inductor
\\DARLINGTON TRANSITOR
\\C is shared, E connects to next B. $V_{BE}=1.4V$, $\beta=\beta^2$, $V_{CE}=V_{BE2}+V_{CE1}>=0.9V$
\\T2 doesn't saturate, can be good for high speed, used for RF
\newpage
1 Intro to Elctronics S 2024 Crib Sheet Exam Final Hayden Fuller
\\(power supply and LCD excluded)
\\Gate charge induces oposite charge in semiconductor
\\electrons Source to Drain, current Drain to Source
\\$L_G$ is length between S and D, $W_G$ is how "deep" the whole thing is extruded
\\$V_{GS}>0$ negative charge induced in channel, $V_{GS}<0$ positive charge induced in channel, 
\\Threshold Voltage: Voltage needed to induce mobile charges in channel, allow current flow
\\$V_{th}$ can be anything, above or below zero,
\\$V_{GS}>V_{th}$ turns on
\\
\\$I_D$ vs $V_{DS}$:
\\1) ohmic
\\Increased $V_{GS}$ increases slope
\\Slope $=dI_D/dV_{DS}=1/R_{ON/OFF}$
\\$I_D=k (V_{GS}-V_{th})V_{DS}$; $k=k'W_G/L_G$
\\
\\2) Intermediate
\\$I_D=k [(V_{GS}-V_{th})V_{DS}-\frac{1}{2}V_{DS}^2]$
\\linear starts to saturate
\\caused by pinch off on D side of channel when $V_{DS}$ too high
\\$V_{DS}=V_{GS}-V_{th}$ has slope 0
\\
\\3) Saturation
\\$V_{DS}\ge V_{GS}-V_{th}$
\\$V_{DS}$ no longer contributes past $V_{GS}-V_{th}$, so replace in equations
\\$I_D=\frac{1}{2}k(V_{GS}-V_{th})^2$
\\
\\$I_D$ VS $V_{GS}$
\\parabola $I_D=\frac{1}{2}k(V_{GS}-V_{th})^2$, bottom is zero at $V_{th}$
\\
\\Load Line of amp
\\Common S (GND), $R_D$ from D to VCC
\\$I_D=(V_{CC}-V_{DS})/R_D$
\\hits $I_D$ at $V_{CC}/R_D$; hits $V_{DS}$ at VCC
\\Dynamic region and Q point in middle, under top curve of constant $V_{GS}$
\\Which regime to use? SATURATION! 
\\linear is dependent on both $V_{GS}$ and $V_{GS}$, saturation current is only dependent on $V_{GS}$
\\
\\SYMBOLS
\\4 terminal, basic FET symbol with arrow at bulk
\\in from bulk to fet for N-channel
\\out from fet to bulk for P-channel
\\Bulk allows the tuning of $V_{th}$, usually connected to S for 3 terminals
\\
\\3 terminal
\\N-channel: e from S to D, I from D to S
\\Arrow out from G to S
\\Enhancement Mode: normally OFF, $V_{th}>0$, has gap in channel of symbol
\\Depletion Mode: normally ON , $V_{th}<0$, no gap in channel of symbol (default)
\\
\\P-channel: h from S to D, I from S to D
\\Arrow in from S to G
\\Enhancement Mode: normally OFF, $V_{th}<0$, has gap in channel of symbol
\\Depletion Mode: normally ON , $V_{th}>0$, no gap in channel of symbol (default)
\\
\\2 DIGITAL SYMBOLS
\\N-channel enhancement $V_{th}>0$ regular digital
\\P-channel enhancement $V_{th}<0$ inverted digital
\\Complementary MOS!
\\
\\DC-BIASING OF FET
\\FIND Q POINT OF CIRCUIT ($I_D$ and $V_{DS}$)
\\$I_G=0$; $I_D=I_S$
\\EX 1:
\\N-channel common S, RD to VCC, D out
\\R1 R2 divider on G
\\C on input and output
\\$V_{GS}=V_G$
\\$V_{GS}=R2/(R1+R2)$ (voltage divider)
\\$I_D=\frac{1}{2}k(V_{GS}-V_{th})^2$ (saturation)
\\2 equations 2 unknowns, solve
\\verify we're in saturation with $V_{DS}=VCC-R_DI_D\ge V_{GS}-V_{th}$
\\
\\EX 2:
\\N-channel common NONE, RD to VCC, RS to GND, D output
\\R1 R2 divider on G
\\C on input and output
\\$V_{GS}=V_G-V(R_S)$
\\$V_{GS}=R2/(R1+R2)-R_SI_D$ (voltage divider + KVL)
\\$I_D=\frac{1}{2}k(V_{GS}-V_{th})^2$ (saturation)
\\2 equations 2 unknowns, solve, system of equations, quadratic (Right solution, $V_{GS}>V_{th}$)
\\verify we're in saturation with $V_{DS}=VCC-(R_D+R_S)I_D\ge V_{GS}-V_{th}$
\\
\\EX 3:
\\N-channel common S, RD to VCC, D output
\\FEEDBACK: RG from D to G
\\C on input and output
\\Current through RG is zero! no G current, assume no current from input source
\\$V_{GS}=V_G$
\\$V_{GS}=V_D=VCC-I_DR_D$ (no voltage difference)
\\$I_D=\frac{1}{2}k(V_{GS}-V_{th})^2$ (saturation)
\\2 equations 2 unknowns, solve, system of equations, quadratic (Right solution, $V_{GS}>V_{th}$)
\\verify we're in saturation with $V_{DS}=VCC-R_DI_D\ge V_{GS}-V_{th}$
\\
\\EQUIVALENT CIRCUIT OF FET (n-channel)
\\common S, left input $V_{GS}$
\\D to S current source $I_D=\frac{1}{2}k(V_{GS}-V_{th})^2$
\\Small Signal Linearization:
\\D to S current source $i_D=g_mV_{GS}$
\\$g_m=\frac{dI_D}{dV_{GS}}$ (Siemens, Transconductance)
\\$g_m=k(V_{GS}-V_{th})$; ($V_{GS}$ of Q-point)
\\Capacitor like input, no input DC current, no power! Great for low power signals
\\$i_D=g_mv_{GS}$
\\
\\Non Ideal Output Impedance
\\Small signal adds $R_{DS}$ parallel to current source
\\$I_D$ VS $V_{GS}$ adds slope $1/R_{DS}>0$
\\non ideal current source, finite output impedance $R_{DS}$
\\
\\3 FET AMPLIFIERS
\\Common SOURCE
\\RD to VCC, D out
\\Basic bias network voltage divider optional
\\
\\Common DRAIN
\\RS to GND, S out
\\Basic bias network voltage divider optional
\\SOURCE FOLLOWER
\\
\\Common GATE
\\S input (arrow), D output
\\Current Amplification 1.0
\\CURRENT FOLLOWER
\\
\\ANALYSIS
\\R source and R load don't matter
\\C$\rightarrow\infty$ so that $Z_C=1/(j\omega C)=0$
\\realistically, let past all relevant frequencies
\\Small Signal:
\\Anything to VCC goes to GND instead
\\basic common S with basic voltage divider input bias
\\R1 and R2 G to S; current source and RD from D to S
\\$z_{in}=R1||R2=1/(1/R1+1/R2)$
\\$Z_{out}=R_D$ or $R_D||R_{DS}$ for non ideal
\\$A_{ISC}=i_{out}/i_{in}=g_mv_{GS}/(v_{GS}/(R1||R2))=g_m(R1||R2)$; large R1 and R2 means large $A_{ISC}$
\\$A_{VOC}=v_{out}/v_{in}=g_mV_{GS}R_D/V_{GS}=g_mR_D$
\\
\\Add source resistance
\\goes in series with current source
\\$z_{in}=R1||R2$
\\$z_{out}=R_D$
\\$A_{ISC}=g_mV_{GS}/((V_{GS}+R_Sg_mV_{GS})/(R1||R2))=g_m(R1||R2)/(1+g_mR_S)$; $R_S$ reduces amplification
\\$A_{VOC}=g_mV_{GS}R_D/(V_{GS}+R_Sg_mV_{GS})=g_mR_D/(1+g_mR_S)$; $R_S$ reduces amplification
\\
\\Common Drain source follower (with basic voltage divider bias)
\\R1 and R2  from G to GND
\\Drain is grounded, so current source from GND to S (parallel to RS)
\\$z_{in}=R1||R2$
\\$z_{out}=R_S$
\\$A_{ISC}=g_mV_{GS}/(V_{GS}/(R1||R2))=g_m(R1||R2)$; for high R1 and R2, $A_{ISC}>>1$ (large)
\\$A_{VOC}=g_mV_{GS}R_S/(V_{GS}+g_mV_{GS}R_S)=g_mR_S/(1+g_mR_S)$; for large $R_S$, $A_{VOC}\approx 1$
\\S voltage follows G voltage
\\
\\KEEP IN MIND
\\inverting also exists, sometimes negative is left out of $A_{VOC}$, doesn't really matter for audio
\\Smaller is better! $k=k'W_G/L_G$, shorter $L_G$ (length between S and D) is proportional to better amplification
\\
\\MILLER CAPACITASNCE
\\$C_{GS}$, $C_{DG}$, $C_{GS}$
\\$C_{DG}$ is most important
\\Modification: we replace it with $C_M$ G to S and $C_{MO}$ D to S
\\$C_M=(1+A_{VOC})C_{DG}$ for RL$=\infty$, $C_M=(1+A_V)C_{DG}$ otherwise
\\4 $C_{MO}=((1+A_{VOC}^{-1})C_{DG}\approx C_{DG}$
\\$\tau_{in}=R_{in}C_M$; $\tau_{out}=R_LC_{MO}$; (R is counting anything not parallel)
\\$\tau_{out}$ is smaller, can neglect $C_{MO}$
\\$\omega_{knee}=1/(RC_M)$, better response with reduced R, low output impedance pre amp (source follower)
\\
\\FET AS SWITCH
\\dim with PWM, better because no resistive powewr loss
\\perfect switch is $R_{ON}=0\Omega$ and $R_{OFF}=\infty\Omega$
\\OFF is saturation
\\$I_D=\frac{1}{2}k(V_{GS}-V_{th})^2$
\\$V_{GS}<V_{th}\rightarrow I_D=0\rightarrow R_{OFF}=\infty$, perfect off state
\\ON is ohmic
\\$I_D=\frac{1}{2}k(V_{GS}-V_{th})^2$
\\$R_{ON}=1/(k(V_{GS}-V_{th}))>0$, non perfect on state
\\minimize $R_{ON}$: $V_{GS}>>V_{th}$; geometry of fet: wide $W_G$, short $L_G$, increase k, decrease $R_{ON}$
\\transision from mechanical to FETs
\\
\\INTEGRATED CIRCUITS CMOS
\\n-channel: arrow out, non inverting, $V_{th}>0$
\\p-channel: arrow in, inverting, $V_{th}<0$
\\both enhancement
\\inverter: p to VCC, n to GND
\\static power consumption is zero
\\not quite because 1) caps charge and discharge with resistance and 2) $R_{ON}$ isn't quite zero
\\
\\ICONIC CIRCUITS
\\Memory: 
\\horizontal wrod lines enable access, point into above cells
\\vertical bit line(s) read and write data to cells
\\Flip FLop:
\\4 FETS for bistable flip flop; p on top, n on bottom, middle of them is input to other side
\\SRAM:
\\Static Random Access Memory
\\Each (or one) side output has an access transistor (n), who's input comes from the word line, 
\\allowing the bit line to read or write when on
\\6 transistors (4n 2p); fast; VOLATILE
\\DRAM:
\\Dynamic RAM, storage is a capacitor, same word line, bit line, and access n transistor, but now just a grounded cap to store.
\\needs to be refreshed every 64ms or so, still VOLATILE
\\SRAM is better, DRAM is cheaper
\\FLASH:
\\Gate M is completely surrounded by Oxide, stores that charge for 10-20 years, (+ on M for n-channel)
\\need a "flash" of charge through the insulator to change it, high voltage pulse between S and D; NON VOLATILE
\newpage
5 BACK WORK
\\Example 1: work backwards
\\n-channel common NONE, RD and RS, output from D through R3, basic bias voltage divider R1 R2
\\Caps on input, output, and S to GND (RE); RS is shorted at AC
\\$k=20mA/V^2$, $V_{th}=2V$, $I_D=20mA$, $V_{DS}=5V$, $VCC=10V$
\\a) assume $V_D=6V$, determine $R_D$ and $R_S$
\\$V_S=V_D-V_{DS}=6V-5V=1V$
\\$R_S=V(R_S)/I_D=1V/20mA=50\Omega$
\\$R_D=V(R_D)/I_D=4V/20mA=200\Omega$
\\b) assume $R1+R2=100k\Omega$, determine them
\\solve for $V_{GS}=\sqrt{2I_D/k}+V_{th}$
\\$V_{GS}=\sqrt{2*20mA/(20mA/V^2)}+2V=3.41V$
\\Voltage divider: $V_G=VCC*R2/(R1+R2)$; $3.41V+1V=10V*R2/100k\Omega$; $R2=44.1k\Omega$; $R1=55.9k\Omega$
\\c) what is the common name for this? what class amplifier?
\\Common Source, Class A amplifier
\\d) assume $R3=200\Omega$, draw small signal equivalent, give transconductance $g_m$
\\G has R1 R2 to GND, $V_{gs}$ to S; D has current source $g_mV_{GS}$ to S, RD to GND, and R3 to output
\\$g_m=k(V_{GS}-V_{th})=20mA/V^2 (3.41V-2V)=28.2$mSiemens
\\e) $A_{VOC}$, expression and value
\\$v_{in}=V_{GS}$; $v_{out}=g_mV_{GS}R_D$
\\$A_{VOC}=g_mV_{GS}R_D/V_{GS}=g_mR_D$
\\$A_{VOC}=28.2mS*200\Omega=5.64$
\\f) $A_{ISC}$, expression and value
\\$i_{in}=v_{in}/(R1||R2)=V_{GS}/(R1||R2)$
\\$i_{out}=g_mV_{GS} R3^{-1}/(R_D^{-1}+R3^{-1})$; (current divider)
\\$A_{ISC}=g_mV_{GS} R3^{-1}/(R_D^{-1}+R3^{-1})/(V_{GS}/(R1||R2))=g_m R3^{-1}/(R_D^{-1}+R3^{-1}) (R1||R2)$
\\$A_{ISC}=28.2*1/2*24.7=348$
\\g) draw small circuit equivalent, calculate $A_V=V_{Load}/V_{SS}$, expression and value
\\same as before but add source $V_SS$ and input resistance (series with load) $R_{SS}$; and add $R_L$
\\$V_{GS}=V_{SS}(R1||R2)/(R_{SS}+(R1||R2))$
\\$V_{SS}=V_{GS} (R_{SS}+(R1||R2))/(R1||R2)$
\\$V_{Load}=RL*i_D*(R3+RL)^{-1}/(R_D^{-1}+(R3+RL)^{-1})$ (resistance * current divider)
\\$V_{Load}=RL*g_mV_{GS}*(R3+RL)^{-1}/(R_D^{-1}+(R3+RL)^{-1})$
\\$A_V=V_{Load}/V_{SS}=[g_mV_{GS}RL((R3+RL)^{-1})/(R_D^{-1}+(R3+RL)^{-1})]/[V_{GS} (R_{SS}+(R1||R2))/(R1||R2)]$
\\$A_V=g_m (R3+RL)^{-1}/(R_D^{-1}+(R3+RL)^{-1}) * R_L (R1||R2)/(R_{SS}+(R1||R2))$
\\$A_V=1.56$
\\h) calculate $A_I=i_{Load}/i_{SS}$, expression and value
\\$i_{Load}=g_mV_{GS}(R3+RL)^{-1}/(R_D^{-1}+(R3+RL)^{-1})$
\\$i_{SS}=V_{GS}/(R1||R2)$
\\$A_I=[g_mV_{GS}(R3+RL)^{-1}/(R_D^{-1}+(R3+RL)^{-1})]/[V_{GS}/(R1||R2)]=(R1||R2)g_m(R3+RL)^{-1}/(R_D^{-1}+(R3+RL)^{-1})$
\\$A_I=232$
\\Example 2: True/False
\\(a) An astable flip-flop circuit can be made by using BJTs but not by using FETs.
\\False, can be made with BJTs or FETs
\\(b) A CMOS dynamic random access memory (DRAM) 1-bit memory cell has two complementary MOS transistors and one capacitor.
\\False, 1 bit of DRAM requires one transistor and one capacitor
\\(c) The first integrated circuit was made with BJTs.
\\True, though FETs are more popular now, they're newer than BJTs
\\(d) At the present time, integrated circuits made with BJTs are relatively rare.
\\True, BJTs are power hungry, not often a need for them when CMOS technology can do it with less
\\(e) The following equation may be reasonable for some specific circumstances: $g_m = 1⁄3 k$, where gm is the transconductance of an FET, and k is the k-value of the FET.
\\False, $g_m$ is in Siemens, while $k$ is in $mA/V^2$


\end{large}

\end{document}




























