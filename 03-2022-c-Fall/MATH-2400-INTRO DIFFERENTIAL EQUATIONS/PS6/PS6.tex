\documentclass{article}

\usepackage{amsmath}
\usepackage{amssymb}
\usepackage{bm}
\usepackage{graphicx}
\usepackage{epstopdf}
\DeclareGraphicsRule{.tif}{png}{.png}{`convert #1 `basename #1 .tif`.png}
\usepackage{color}
\usepackage{pdfsync}
\pagestyle{plain}

\textheight 9 true in
\textwidth 6.5 true in
\hoffset -.75 true in
\voffset -.75 true in
\mathsurround=2pt
\parskip=2pt

\begin{document}

\begin{center}
\large{ MATH-2400 \hspace{.27in}  INTRODUCTION TO DIFFERENTIAL EQUATIONS \hspace{.27in}FALL 2022\bigskip\\ {\bf Problem Set 6} \smallskip\\ Due: 11pm, Tuesday, October 25, 2022}
\\Hayden Fuller
\end{center}

\bigskip\noindent
\underline{NOTES}
\begin{enumerate}
\item Practice problems listed below and taken from the textbook are for your own practice, and are not to be turned in.
\item There are two parts of the Problem Set, an objective part consisting of multiple choice questions (with no partial credit available) and a subjective part (with partial credit possible).  Please complete all questions.
\item Writing your solutions in {\LaTeX} is preferred but not required.
\item Show all work for problems in the subjective part.  Illegible or undecipherable solutions will not be graded. 
\item Figures, if any, should be neatly drawn by hand, properly labelled and captioned.  
\item Your completed work is to be submitted electronically to LMS  as a \textcolor{red}{single pdf file}. Be sure that the pages are properly oriented and well lighted.  (\textcolor{blue}{Please do not e-mail your work to Muhammad or me.})
\end{enumerate}

\bigskip\noindent
{\bf Practice Problems from the textbook} (Not to be turned in)
\begin{itemize}
\item
Exercises from Chapter 3, pages 72--75: 1(c,d), 2, 4, 7, 8, 12, 13.
\end{itemize}

\bigskip\noindent
{\bf Objective part} (Choose A, B, C or D; no work need be shown, no partial credit available)

\begin{enumerate}
\item (5 points) The displacement $u(t)$ of a mass-spring-damper system is governed by $mu\sp{\prime\prime} + cu\sp\prime + ku = 0$, where $m=2$ and $k=8$.  For what value of the damping coefficient $c$ is the system critically damped? 
\begin{description}
\item[A] $c=4$
\item[B] X$c=8$	X
\item[C] $c=16$
\item[D] None of these choices
\end{description}
$c=\sqrt{4km}=\sqrt{64}=8$

\item (5 points) The displacement $u(t)$ of a forced mass-spring-damper system is governed by the linear DE $mu\sp{\prime\prime} + cu\sp\prime + ku = 5\cos(2t)$.  For what values of the mass $m$, damping coefficient~$c$ and spring constant $k$ is the system in resonance? 
\begin{description}
\item[A] $m=1$, $c=0$, $k=2$
\item[B] $m=2$, $c=1$, $k=4$
\item[C] $m=2$, $c=1$, $k=8$
\item[D] XNone of these choicesX
\end{description}
$5\cos(2t)$; $\omega=2$
\\$\omega_0=\sqrt{\frac{k}{m}}=\sqrt{2}\ne2=\omega$

\newpage
\item (5 points) The displacement $u(t)$ of a forced mass-spring system is governed by $u\sp{\prime\prime} + 2u\sp{\prime} +3u = 4\cos(t)$.  The amplitude $R$ of the forced  response is given by
\begin{description}
\item[A]  $R=1$
\item[B]  X$R=\sqrt2$X
\item[C]  $R=2$  
\item[D] None of these choices
\end{description}
$R=\frac{F_0}{\sqrt{D}}$
\\$F_0=4$
\\$\omega=0$
\\$\omega_0=\sqrt{\frac{k}{m}}=\sqrt{3}$
\\$D=c^2\omega^2+(k-m\omega^2)^2=2^2\omega^2+(3-1\omega^2)^2$
\\$2^21^2+(3-1*1^2)^2=4+4=8$
\\$\frac{4}{\sqrt{8}}=\frac{4\sqrt{8}}{8}=\sqrt{2}$
\end{enumerate}

\bigskip\bigskip\noindent
{\bf Subjective part} (Show work, partial credit available)

\begin{enumerate}

\item (15 points)  A mass weighing $8\;{\rm lb}$ stretches a spring $4\;{\rm in}$.  Assume the mass is pulled downward, stretching the spring a distance of $6\;{\rm in}$, and then set in motion with an upward velocity of $3\;{\rm ft}/{\rm s}$.  There is no damping in the system and the acceleration due to gravity is $g=32\;{\rm ft}/{\rm s}\sp2$.
\begin{enumerate}
\item
Determine an initial-value problem for the downward displacement $u(t)$ in units of ${\rm ft}$.
\\$mg=ku$; $8=k\frac{1}{3}$; $k=24$; $m32=8$; $m=\frac{1}{4}$
\\$u_0=\frac{1}{2}$; $u'_0=-3$
\\$\frac{1}{4}u''+24u=0$

\item
Solve the IVP and express the solution in the polar form $u(t)=R\cos(\omega_0t-\phi)$.
\\$r=\frac{0\pm\sqrt{0-4\frac{1}{4}24}}{2\frac{1}{4}}=\pm2\sqrt{24}i=\pm4\sqrt{6}i$
\\$\omega_0=\sqrt{\frac{k}{m}}=\sqrt{24*4}=4\sqrt{6}$
\\$u(t)=u_0\cos(\omega_0t)+\frac{u'_0}{\omega_0}\sin(\omega_0t)$
\\$u(t)=\frac{1}{2}\cos(4\sqrt{6}t)-\frac{3}{4\sqrt{6}}\sin(4\sqrt{6}t)$
\\$u(t)=\frac{1}{2}\cos(4\sqrt{6}t)-\frac{\sqrt{6}}{8}\sin(4\sqrt{6}t)$
\\$R=\sqrt{\frac{1}{2}^2+\frac{-\sqrt{6}}{8}^2}$
\\$R=\sqrt{\frac{1}{4}+\frac{6}{64}}$
\\$R=\sqrt{\frac{8}{32}+\frac{3}{32}}$
\\$R=\sqrt{\frac{11}{32}}$
\\$\tan(\phi)=\frac{C_2}{C_1}$
\\$\phi=\arctan(\frac{-\frac{\sqrt{6}}{8}}{\frac{1}{2}})$
\\$\phi=\arctan(\frac{-\sqrt{6}}{4})$
\\
\\$u(t)=R\cos(\omega_0t-\phi)$
\\$u(t)=\sqrt{\frac{11}{32}}\cos(4\sqrt{6}t-\arctan(\frac{-\sqrt{6}}{4}))$
\\

\item
Determine the frequency, period and amplitude of the oscillation.  Sketch the solution.
\\frequency $=\frac{2\sqrt{6}}{\pi}$ Hz.
\\period $=\frac{2\pi}{4\sqrt{6}}=\frac{\pi}{2\sqrt{6}}$ seconds.
\\amplitude $=\sqrt{\frac{11}{32}}$ feet.
\\A cosine wave with y intercept at $0.5$ with a slope of $-3$ at that point, an amplitude of $\sqrt{\frac{11}{32}}$, and a period of $\frac{\pi}{2\sqrt{6}}$.
\\https://www.desmos.com/calculator/9bs2160hyx
\end{enumerate}


\bigskip
\item (15 points) A force of $4\;{\rm N}$ stretches a spring $10\;{\rm cm}$.  A mass of $2\;{\rm kg}$ is hung from the spring, and the mass is also attached to a viscous damper that exerts a force of $16\;{\rm N}$ when the velocity of the mass is $2\;{\rm m}/{\rm s}$.  The mass is set into motion from its equilibrium position by an initial downward velocity of $20\;{\rm cm}/{\rm s}$.
\begin{enumerate}
\item
Determine an initial-value problem for the {\bf upward} displacement $u(t)$ in units of meters.
\\$4=0.1k$; $k=40$; $m=2$; $16=2c$; $c=8$
\\$u_0=0$; $u'_0=-0.2$
\\$u(t)=2u''+8u'+40u=0$
\item
Solve the IVP and sketch the solution.
\\$u=e^{rt}$; $2r^2+8r+40=0$; $r=\frac{-8\pm\sqrt{8^2-4*2*40}}{2*2}=\frac{-8\pm\sqrt{64-320}}{4}=\frac{-8\pm\sqrt{-256}}{4}=-2\pm4i$
\\$\lambda=-2$; $\omega=4$
\\$u(t)=e^{\lambda t}(C_1\cos(\omega t)+C_2\sin(\omega t))$
\\$u(t)=e^{-2t}(C_1\cos(4t)+C_2\sin(4t))$
\\$u(0)=0=e^{0}(C_1\cos(0)+C_2\sin(0))$
\\$u(0)=0=C_1$
\\$u(t)=e^{-2t}C_2\sin(4t)$
\\$u'(t)=-2e^{-2t}C_2\sin(4t)+e^{-2t}C_24\cos(4t)$
\\$u'(0)=-0.2=-2e^{0}C_2\sin(0)+e^{0}C_24\cos(0)$
\\$u'(0)=-0.2=4C_2$
\\$C_2=-0.05$; $C_1=0$
\\$u(t)=-e^{-2t}0.05\sin(4t)$
\\A sine wave through the origin with a slope of $-0.2$ at that point and a period of $\frac{\pi}{2}$, with an amplitude in the envelope bounded above by $0.05e^{-2t}$ and below by $-0.05e^{-2t}$.
\\https://www.desmos.com/calculator/y3qnqf77uz
\end{enumerate}


\bigskip
\item (15 points) The displacement $u(t)$ of a forced mass-spring-damper system satisfies the DE
\[
u\sp{\prime\prime}+2u\sp\prime+3u=\cos(\omega t)
\]
\begin{enumerate}
\item The forced response of the system has the form $u_p(t)=A\cos(\omega t)+B\sin(\omega t)$.  Determine formulas for $A$ and $B$.  (Note: your formulas will involve the frequency $\omega$ of the forcing.)
\\$m=1$; $c=2$; $k=3$
\\$\omega_0=\sqrt{\frac{3}{1}}=\sqrt{3}$
\\$u_p(t)=A\cos(\omega t)+B\sin(\omega t)$
\\$u'_p(t)=-A\omega\sin(\omega t)+B\omega\cos(\omega t)$
\\$u''_p(t)=-A\omega^2\cos(\omega t)-B\omega^2\sin(\omega t)$
\\let $s=\sin(\omega t)$ and $c=\cos(\omega t)$
\\$u_p(t)=Ac+Bs$
\\$u'_p(t)=-A\omega s+B\omega c$
\\$u''_p(t)=-A\omega^2c-B\omega^2s$
\\$L[u_p]=(-A\omega^2c-B\omega^2s)+2(-A\omega s+B\omega c)+3(Ac+Bs)=c$
\\$c(-A\omega^2+2B\omega+3A)+s(-B\omega^2-2A\omega+3B)=c$
\\$-A\omega^2+2B\omega+3A=1$
\\$A(-\omega^2+3)+B(2\omega)=1$
\\$A(-\omega^2+3)^2+B(2\omega)(-\omega^2+3)=(-\omega^2+3)$
\\$-B\omega^2-2A\omega+3B=0$
\\$A(-2\omega)+B(-\omega^2+3)=0$
\\$A(-2\omega)^2+B(-\omega^2+3)(-2\omega)=0$
\\$A((-2\omega)^2+(-\omega^2+3)^2)=(-\omega^2+3)$
\\$A=\frac{(-\omega^2+3)}{((-2\omega)^2+(-\omega^2+3)^2)}=\frac{-\omega^2+3}{4\omega^2+(3-\omega^2)^2}$
\\$D=c^2\omega^2+(k-m\omega^2)^2=2^2\omega^2+(3-1\omega^2)^2=4\omega^2+(3-\omega^2)^2$
\\$A=\frac{3-\omega^2}{D}$
\\$A(-2\omega)+B(-\omega^2+3)=0$
\\$\frac{3-\omega^2}{D}(-2\omega)+B(-\omega^2+3)=0$
\\$B(-\omega^2+3)=\frac{3-\omega^2}{D}(2\omega)$
\\$B=\frac{1}{D}(2\omega)$
\\$B=\frac{2\omega}{D}$
\\$u_p(t)=\frac{3-\omega^2}{D}\cos(\omega t)+\frac{2\omega}{D}\sin(\omega t)$
\item The amplitude of the forced response $R$ is given by $R=\sqrt{A\sp2+B\sp2}$, where $A$ and $B$ are given by the formulas from part~(a).  Determine the frequency $\omega$ that maximizes the amplitude of the forced response.  (Hint: consult an in-class example.)
\\$R=\sqrt{A^2+B^2}=\sqrt{(\frac{3-\omega^2}{D})^2+(\frac{2\omega}{D})^2}=\sqrt{\frac{4\omega^2+(3-\omega^2)^2}{D^2}}=\sqrt{\frac{D}{D^2}}=\sqrt{\frac{1}{D}}=D^{-\frac{1}{2}}$
\\$R'=0$; $D'=0$
\\$D=4\omega^2+(3-\omega^2)^2=4\omega^2+9-6\omega^2+\omega^4=\omega^4-2\omega^2+9$
\\$D'=0=4\omega^3-4\omega=4\omega(\omega^2-1)$
\\$\omega=-1$, $0$, or $1$
\\$\omega=1$
\\$\omega_0=\sqrt{3}$
\\$\omega$ is close enough to $\omega_0$, makes sense.
\\$\omega=1$Hz
\end{enumerate}


\end{enumerate}


\end{document}






























