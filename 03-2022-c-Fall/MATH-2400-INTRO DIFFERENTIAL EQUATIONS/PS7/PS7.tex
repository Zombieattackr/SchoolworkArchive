\documentclass{article}

\usepackage{amsmath}
\usepackage{amssymb}
\usepackage{bm}
\usepackage{graphicx}
\usepackage{epstopdf}
\DeclareGraphicsRule{.tif}{png}{.png}{`convert #1 `basename #1 .tif`.png}
\usepackage{color}
\usepackage{pdfsync}
\pagestyle{plain}

\textheight 9 true in
\textwidth 6.5 true in
\hoffset -.75 true in
\voffset -.75 true in
\mathsurround=2pt
\parskip=2pt

\begin{document}

\begin{center}
\large{ MATH-2400 \hspace{.27in}  INTRODUCTION TO DIFFERENTIAL EQUATIONS \hspace{.27in}FALL 2022\bigskip\\ {\bf Problem Set 7} \smallskip\\ Due: 11pm, Tuesday, November 1, 2022}
\\Hayden Fuller
\end{center}

\bigskip\noindent
\underline{NOTES}
\begin{enumerate}
\item Practice problems listed below and taken from the textbook are for your own practice, and are not to be turned in.
\item There are two parts of the Problem Set, an objective part consisting of multiple choice questions (with no partial credit available) and a subjective part (with partial credit possible).  Please complete all questions.
\item Writing your solutions in {\LaTeX} is preferred but not required.
\item Show all work for problems in the subjective part.  Illegible or undecipherable solutions will not be graded. 
\item Figures, if any, should be neatly drawn by hand, properly labelled and captioned.  
\item Your completed work is to be submitted electronically to LMS  as a \textcolor{red}{single pdf file}. Be sure that the pages are properly oriented and well lighted.  (\textcolor{blue}{Please do not e-mail your work to Muhammad or me.})
\end{enumerate}

\bigskip\noindent
{\bf Practice Problems from the textbook} (Not to be turned in)
\begin{itemize}
\item
Exercises from Chapter 6, pages 149--150: 1(c,d,f,i), 2(a,d), 4(a,d).
\item
Exercises from Chapter 6, pages 155--157: 1(b,d), 2(a,d,g,k).
\item
Exercises from Chapter 6, page 163: 1(c,h), 2(b,c,e).
\end{itemize}

\bigskip\noindent
{\bf Objective part} (Choose A, B, C or D; no work need be shown, no partial credit available)

\begin{enumerate}
\item (5 points)  Let $f(t)=e\sp{t}$, $g(t)=e\sp{2t}\sin{3t}$ and $H(t)$ be the Heaviside step function.  Identify the correct statement, or select \lq\lq All of these choices'' if all of the statements are correct.
\begin{description}
\item[A] If $\displaystyle{\hat{F}=\int_0\sp{T} f(t)e\sp{-st}\,dt}$, then $\displaystyle{\lim_{T\rightarrow\infty}\hat{F}}$ exists for $s>1$.
\item[B] If $G(s)=\displaystyle{\int_0\sp\infty g(t)e\sp{-st}\,dt}$, then $G(s)=\displaystyle{{3\over(s-2)\sp2+9}}$ assuming $s>2$.
\item[C] If $h(t)=\cos t+tH(t-1)$, then $h(t)$ is piecewise continuous for $t\in[0,3]$.
\item[D] XAll of these choicesX
\end{description}

\item (5 points) Which of the following is correct, or select \lq\lq None of these choices'' if none are correct.
\begin{description}
\item[A] $\displaystyle{{\cal L}\bigl(\sin\sp2t\bigr)={1\over(s\sp2+1)\sp2}}$
\item[B] ${\cal L}\bigl(u\sp{\prime\prime}(t)\bigr)=s{\cal L}\bigl(u\sp{\prime}(t)\bigr)$ 
\item[C] X$\displaystyle{{\cal L}\bigl(t\sp2e\sp{-t}\bigr)={2\over(s+1)\sp3}}$X
\item[D] None of these choices
\end{description}

\newpage
\item (5 points) Let $f(t) = [1-H(t-\pi)]\sin t + H(t-\pi) \cos t$, where $H(t)$ is the Heaviside step function.  Which of the following is correct, or select \lq\lq None of these choices'' if none are correct.
\begin{description}
\item[A]  $f(0)=f(3\pi)$
\item[B]  X$f(\pi/2)=f(2\pi)$X
\item[C]  $f(2\pi)=f(3\pi)$
\item[D] None of these choices
\end{description}


\end{enumerate}

\bigskip\bigskip\noindent
{\bf Subjective part} (Show work, partial credit available)

\begin{enumerate}

\item (15 points)  Use the properties of the Laplace transform and the table of Laplace transforms discussed in class (or the one given in the text) to find the following:
\begin{enumerate}
\item
Find $F(s)$ and $G(s)$ if
\[
f(t)=2te\sp{-2t}\sin3t,\qquad g(t)=\left\{\begin{array}{cl} 0 & \hbox{if $t<3$}\smallskip\\ e\sp{t}\cos(t-3) & \hbox{if $t\ge3$}\end{array}\right.
\]
\\$f_1=e^{-2t}\sin(3t)$
\\$f=2tf_1$
\\T7,a=-2,b=3,$F_1=\frac{3}{(s+2)^2+9}=\frac{3}{s^2+4s+13}$
\\T9,n=1,$F=2(-1)\frac{(s^2+4s+13)(0)-(3)(2s+4)}{(s^2+4s+13)^2}=-2\frac{-6s+12}{(s^2+4s+13)^2}$
\\$F=-2\frac{-6s+12}{(s^2+4s+13)^2}$
\\
\\$g_1=e^{t+3}\cos(t)=e^3e^t\cos(t)$
\\$g=H(t-3)g_1(t-3)$
\\T12,c=3,$G=G_1e^{-3s}$
\\$g_2=\cos(t)$
\\T4,b=1,$G_2=\frac{s}{s^2+1}$
\\T8,a=1,$G_1=e^3G_2(s-1)=e^3\frac{s-1}{(s-1)^2+1}=e^3\frac{s-1}{s^2-2s+2}$
\\$G=e^3\frac{s-1}{s^2-2s+2}e^{-3s}$
\item
Find $u(t)$ and $v(t)$ if
\[
U(s)={2s+7\over s\sp2+4s+5},\qquad V(s)={e\sp{-s}\over s(s-2)}
\]
\\$U=\frac{2(s+2)+3}{(s+2)^2+1}=3\frac{1}{(s+2)^2+1}+2\frac{s+2}{(s+2)^2+1}$
\\T7,T6,b=1,a=-2,$u=3e^{-2t}\sin(t)+2e^{-2t}\cos(t)$
\\$u=3e^{-2t}\sin(t)+2e^{-2t}\cos(t)$
\\
\\$V=e^{-s}V_1$
\\$V_1=\frac{1}{s(s-2)}$
\\$\frac{1}{s(s-2)}=\frac{A}{s}+\frac{B}{s-2}$
\\$1=A(s-2)+B(s)=(A+B)s-2A$
\\$A+B=0$; $-2A=1$
\\$A=\frac{-1}{2}$; $B=\frac{1}{2}$
\\$V_1=\frac{\frac{-1}{2}}{s}+\frac{\frac{1}{2}}{s-2}$
\\$V_1=\frac{-1}{2}\frac{1}{s}+\frac{1}{2}\frac{1}{s-2}$
\\$V_2=\frac{1}{s}$
\\T1,$v_2=1$
\\$V_3=\frac{1}{s-2}$
\\T3,a=2$v_3=e^{2t}$
\\$v_1=\frac{-1}{2}+\frac{1}{2}e^{2t}$
\\T12,c=1,$v=(\frac{-1}{2}+\frac{1}{2}e^{2(t-1)})H(t-1)$
$v=\frac{1}{2}(-1+e^{2t-2)})H(t-1)$

\end{enumerate}


\bigskip
\item (15 points)  Consider the initial-value problem
\[
y\sp{\prime\prime}+2y\sp\prime=4t,\qquad y(0)=-1, \quad y\sp\prime(0)=3
\]
\begin{enumerate}
\item
Use $y(t)=e\sp{rt}$ to determine the homogeneous solution $y_h(t)$, and use the method of undetermined coefficients to determine the particular solution $y_p(t)$.  Apply the initial conditions to determine the solution $y(t)$ of the IVP.
\\$r^2+2r=0$; $r=0,-2$
\\$y_h=C_1+C_2e^{-2t}$
\\$y_p=A+Bt$; A is homo sol
\\$y_p=At+Bt^2$
\\$y'_p=A+2Bt$
\\$y''_p=2B$
\\$2B+2(A+2Bt)=4t$
\\$2B+2A+4Bt=4t$
\\$2B+2A=0$
\\$4B=4$
\\$B=1$; $A=-1$
\\$y_p=-t+t^2$
\\$y(t)=C_1+C_2e^{-2t}+t^2-t$
\\$y(0)=-1=C_1+C_2e^{0}+0^2-0$
\\$C_1+C_2=-1$
\\$y'(t)=-2C_2e^{-2t}+2t-1$
\\$y'(0)=3=-2C_2e^{0}+0-1$
\\$4=-2C_2$; $C_2=-2$; $C_1=1$
\\$y(t)=1-2e^{-2t}+t^2-t$
\\$y(t)=-2e^{-2t}+t^2-t+1$
\item
Take a Laplace transform of the DE and use the ICs to determine $Y(s)$, the Laplace transform of $y(t)$.  Use the properties of Laplace transforms and the table of Laplace transforms discussed in class (or the one given in the text) to find $y(t)$.  Confirm that the solution found here agrees with the one found in part~(a).
\\${\cal L}y''+2{\cal L}y'=4{\cal L}t$
\\$(s^2Y-sy(0)-y'(0))+2(sY-y(0))=4(\frac{1}{s^2})$
\\$s^2Y+s-3+2sY+2=4\frac{1}{s^2}$
\\$s^2Y+2sY+s-1=4\frac{1}{s^2}$
\\$Y(s^2+2s)=4\frac{1}{s^2}-s+1$
\\$Y=\frac{4\frac{1}{s^2}-s+1}{s^2+2s}$
\\$Y=\frac{-s^3+s^2+4}{s^3(s+2)}=\frac{A_1}{s}+\frac{A_2}{s^2}+\frac{A_3}{s^3}+\frac{A_4}{s+2}$
\\$-s^3+s^2+4=A_1(s+2)s^2+A_2(s+2)s+A_3(s+2)+A_4s^3$
\\$-s^3+s^2+4=A_1(s^3+2s^2)+A_2(s^2+2s)+A_3(s+2)+A_4s^3$
\\$4=2A_3$; $A_3=2$
\\$0=2A_2+A_3$; $0=2A_2+2$; $A_2=-1$
\\$1=2A_1+A_2$; $1=2A_1-1$; $A_1=1$
\\$-1=A_4+A_1$; $-1=A_4+1$; $A_4=-2$
\\$Y=\frac{-s^3+s^2+4}{s^3(s+2)}=\frac{1}{s}+\frac{-1}{s^2}+\frac{2}{s^3}+\frac{-2}{s+2}$
\\T1; T2,n=1; T2,n=2; T3,a=-2; 
\\$y=1-t+t^2-2e^{-2t}$
\\$y(t)=-2e^{-2t}+t^2-t+1$
\end{enumerate}


\bigskip
\item (15 points) The displacement $u(t)$ of a forced mass-spring-damper system is governed by
\[
u\sp{\prime\prime}+2u\sp\prime+10u=f(t),\qquad u(0)=0,\quad u\sp\prime(0)=1
\]
where the external forcing is given by
\[
f(t)=\left\{\begin{array}{cl} 0 & \hbox{if $t<1$}\smallskip\\ 10e\sp{-2(t-1)} & \hbox{if $t\ge1$}\end{array}\right.
\]
\begin{enumerate}
\item Take a Laplace transform of the DE and use the ICs to determine $U(s)$, the Laplace transform of $u(t)$.
\\$f(t)=10e^{-2(t-1)}H(t-1)$
\\T12,c=1,$F(t)=F_1e^{-s}$
\\$f_1=10e^{-2t}$
\\T3,a=-2,$F_1=10\frac{1}{s+2}$
\\$F(t)=\frac{10e^{-s}}{s+2}$
\\${\cal L}u''+2{\cal L}u'+10{\cal L}u={\cal L}f(t)$
\\$(s^2U-su(0)-u'(0))+2(sU-u(0))+10(\frac{1}{s^2})=F$
\\$(s^2U-1)+2(sU)+\frac{10}{s^2}=\frac{10e^{-s}}{s+2}$
\\$s^2U+2sU+\frac{10}{s^2}=\frac{10e^{-s}}{s+2}+1$
\\$s^2U+2sU=\frac{10e^{-s}}{s+2}+1-\frac{10}{s^2}$
\\$U(s^2+2s)=\frac{10e^{-s}}{s+2}+1-\frac{10}{s^2}$
\\$U=\frac{10e^{-s}}{(s+2)(s^2+2s)}+\frac{1}{(s^2+2s)}-\frac{10}{s^2(s^2+2s)}$
\\$U=\frac{10}{s(s+2)^2}e^{-s}-\frac{10}{s^3(s+2)}+\frac{1}{s(s+2)}=U_1-U_2+U_3$
\\$U_1=\frac{10}{s(s+2)^2}e^{-s}$
\\$\frac{10}{s(s+2)^2}=\frac{A_1}{s}+\frac{A_2}{(s+2)}+\frac{A_3}{(s+2)^2}$
\\$\frac{10}{s}=\frac{A_1(s+2)^2}{s}+A_2(s+2)+A_3$
\\$10=A_1(s+2)^2+A_2s(s+2)+A_3s$
\\$10=A_1(s^2+4s+4)+A_2(s^2+2s)+A_3s$
\\$10=4A_1$; $A_1=\frac{5}{2}$
\\$0=A_1+A_2$; $0=\frac{5}{2}+A_2$; $A_2=-\frac{5}{2}$
\\$0=4A_1+2A_2+A_3$; $0=4\frac{5}{2}-2\frac{5}{2}+A_3$; $0=10-5+A_3$; $A_3=-5$
\\$\frac{10}{s(s+2)^2}=\frac{5}{2}\frac{1}{s}-\frac{5}{2}\frac{1}{(s+2)}-5\frac{1}{(s+2)^2}$
\\$U_1=(\frac{5}{2}\frac{1}{s}-\frac{5}{2}\frac{1}{(s+2)}-5\frac{1}{(s+2)^2})e^{-s}$
\\$U_2=\frac{10}{s^3(s+2)}=\frac{A_1}{s}+\frac{A_2}{s^2}+\frac{A_3}{s^3}+\frac{A_4}{s+2}$
\\$\frac{10}{(s+2)}=A_1s^2+A_2s+A_3+\frac{A_4s^3}{s+2}$
\\$10=A_1s^2(s+2)+A_2s(s+2)+A_3(s+2)+A_4s^3$
\\$10=A_1(s^3+2s^2)+A_2(s^2+2s)+A_3(s+2)+A_4s^3$
\\$10=2A_3$; $A_3=5$
\\$0=2A_2+A_3$; $0=2A_2+5$; $A_2=-\frac{5}{2}$
\\$0=2A_1+A_2$; $0=2A_1-\frac{5}{2}$; $A_1=\frac{5}{4}$
\\$0=A_1+A_4$; $0=\frac{5}{4}+A_4$; $A_4=-\frac{5}{4}$
\\$U_2=\frac{5}{4}\frac{1}{s}-\frac{5}{2}\frac{1}{s^2}+5\frac{1}{s^3}-\frac{5}{4}\frac{1}{s+2}$
\\$U_3=\frac{1}{s(s+2)}=\frac{A_1}{s}+\frac{A_2}{s+2}$
\\$\frac{1}{(s+2)}=A_1+\frac{A_2s}{s+2}$
\\$1=A_1(s+2)+A_2s$
\\$1=2A_1$; $A_1=\frac{1}{2}$
\\$0=A_1+A_2$; $0=\frac{1}{2}+A_2$; $A_2=-\frac{1}{2}$
\\$U_3=\frac{1}{2}\frac{1}{s}-\frac{1}{2}\frac{1}{s+2}$
\\
\\$U=((\frac{5}{2}\frac{1}{s}-\frac{5}{2}\frac{1}{(s+2)}-5\frac{1}{(s+2)^2})e^{-s})-(\frac{5}{4}\frac{1}{s}-\frac{5}{2}\frac{1}{s^2}+5\frac{1}{s^3}-\frac{5}{4}\frac{1}{s+2})+(\frac{1}{2}\frac{1}{s}-\frac{1}{2}\frac{1}{s+2})$
\item Use the properties of Laplace transforms and the table of Laplace transforms discussed in class (or the one given in the text) to find $u(t)$.
\\$u_1=(\frac{5}{2}-\frac{5}{2}e^{-2(t-1)}-5e^{-2(t-1)}(t-1))H(t-1)$
\\$u_2=\frac{5}{4}-\frac{5}{2}t+5\frac{t^2}{2}-\frac{5}{4}e^{-2t}$
\\$u_3=\frac{1}{2}-\frac{1}{2}e^{-2t}$
\\$u(t)=u_1-u_2+u_3$
\\$u(t)=(\frac{5}{2}-\frac{5}{2}e^{-2(t-1)}-5e^{-2(t-1)}(t-1))H(t-1)-(\frac{5}{4}-\frac{5}{2}t+5\frac{t^2}{2}-\frac{5}{4}e^{-2t})+(\frac{1}{2}-\frac{1}{2}e^{-2t})$
\\$u(t)=(\frac{5}{2}-\frac{5}{2}e^{-2(t-1)}-5e^{-2(t-1)}(t-1))H(t-1)-\frac{5}{4}+\frac{5}{2}t-5\frac{t^2}{2}+\frac{5}{4}e^{-2t}+\frac{1}{2}-\frac{1}{2}e^{-2t}$
\\$u(t)=(\frac{5}{2}-\frac{5}{2}e^{-2(t-1)}-5e^{-2(t-1)}(t-1))H(t-1)-\frac{3}{4}+\frac{5}{2}t-\frac{5}{2}t^2+\frac{3}{4}e^{-2t}$
\\$u(t)=(\frac{5}{2}-\frac{5}{2}e^{-2(t-1)}-5e^{-2(t-1)}(t-1))H(t-1)+\frac{3}{4}e^{-2t}-\frac{5}{2}t^2+\frac{5}{2}t-\frac{3}{4}$
\end{enumerate}


\end{enumerate}


\end{document}
