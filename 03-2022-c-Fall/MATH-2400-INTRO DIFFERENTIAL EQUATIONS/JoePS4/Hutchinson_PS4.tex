\documentclass{article}

\usepackage{amsmath}
\usepackage{amssymb}
\usepackage{bm}
\usepackage{graphicx}
\usepackage{epstopdf}
\DeclareGraphicsRule{.tif}{png}{.png}{`convert #1 `basename #1 .tif`.png}
\usepackage{color}
\usepackage{pdfsync}
\pagestyle{plain}

\textheight 9 true in
\textwidth 6.5 true in
\hoffset -.75 true in
\voffset -.75 true in
\mathsurround=2pt
\parskip=2pt

\begin{document}

\begin{center}
\large{ MATH-2400 \hspace{.27in}  INTRODUCTION TO DIFFERENTIAL EQUATIONS \hspace{.27in}FALL 2022\bigskip\\ {\bf Problem Set 4} \smallskip\\ Due: 5pm, Friday, September 30, 2022}\\
\textbf{Submitted to LMS By:\\ Joseph Hutchinson\\ 662022852 \\ Section 17 }
\end{center}

\bigskip\noindent
\underline{NOTES}
\begin{enumerate}
\item Practice problems listed below and taken from the textbook are for your own practice, and are not to be turned in.
\item There are two parts of the Problem Set, an objective part consisting of multiple choice questions (with no partial credit available) and a subjective part (with partial credit possible).  Please complete all questions.
\item Writing your solutions in {\LaTeX} is preferred but not required.
\item Show all work for problems in the subjective part.  Illegible or undecipherable solutions will not be graded. 
\item Figures, if any, should be neatly drawn by hand, properly labelled and captioned.  
\item Your completed work is to be submitted electronically to LMS  as a \textcolor{red}{single pdf file}. Be sure that the pages are properly oriented and well lighted.  (\textcolor{blue}{Please do not e-mail your work to Muhammad or me.})
\end{enumerate}

\bigskip\noindent
{\bf Practice Problems from the textbook} (Not to be turned in)
\begin{itemize}
\item
Exercises from Chapter 3, page 50--51: 3(j), 4(h,i,j), 5(a,d,g,f), 6(c).
\item
Exercises from Chapter 3, pages 77--78: 1(a,b,c,d), 2(a,b).
\end{itemize}

\bigskip\noindent
{\bf Objective part} (Choose A, B, C or D; no work need be shown, no partial credit available)

\begin{enumerate}

\item (5 points) Select the linear, {\em homogeneous} DE for which $y(t)=e\sp{-3t}$ is a solution
\begin{description}
\item[A] $y\sp{\prime\prime}+2y\sp\prime=3e\sp{-3t}$
\item[B] $y\sp{\prime\prime}+9y=0$
\item\textcolor{red}{\textbf{[C]} $\bm{ty\sp{\prime\prime}-y\sp\prime-3(1+3t)y=0}$}
\item[D] None of these choices.
\end{description}

\item (5 points) Assume $y(t)$ solves the ODE $y\sp{\prime\prime}+by\sp\prime+cy=0$ and the initial conditions $y(0)=0$, $y\sp\prime(0)=1$.  For what values of $b$ and $c$ does the solution decay to zero as $t\rightarrow\infty$:
\begin{description}
\item\textcolor{red}{\textbf{[A] $\bm{b=4}$ and $\bm{c=4}$}}
\item[B] $b=-2$ and $c=6$
\item[C] Both choices A and B
\item[D] Neither choice A or B
\end{description}

\item (5 points) Select the Cauchy-Euler equation for which $y(x)=x\sp2\cos(\ln x)$, $x>0$, is a solution
\begin{description}
\item[A] $x\sp2y\sp{\prime\prime}-5xy\sp\prime+5y=0$
\item\textcolor{red}{\textbf{[B] $\bm{x\sp2y\sp{\prime\prime}-3xy\sp\prime+5y=0}$}}
\item[C] $x\sp2y\sp{\prime\prime}-3xy\sp\prime+y=0$
\item[D] None of these choices
\end{description}

\end{enumerate}

\newpage\noindent
{\bf Subjective part} (Show work, partial credit available)

\begin{enumerate}

\bigskip
\item (15 points) Consider the linear, homogeneous, second-order ODE
\[
y\sp{\prime\prime}+{3\over2t}\,y\sp\prime-{3\over t\sp2}\,y=0,\qquad t>0
\]
\begin{enumerate}
\item
Verify that $y_1(t)=t\sp{-2}$ is a solution of the ODE, and find a second solution $y_2(t)$ using the method of reduction of order.\\

If $y_1(t)=t\sp{-2}$ is a solution of the ODE, then we should be able to find $y_1\sp{\prime}(t)$ and $y_1\sp{\prime\prime}(t)$. Then, plug these in to the ODE and check if the result$=0$.\\\\
$y_1(t)=t\sp{-2}$\\
$y_1\sp{\prime}(t)=-2t\sp{-3}$\\
$y_1\sp{\prime\prime}(t)=6t\sp{-4}$\\
Plug these into the ODE and simplify:\\
$y\sp{\prime\prime}+{3\over2t}\,y\sp\prime-{3\over t\sp2}\,y=0$\\
$(6t\sp{-4})+{3\over2t}\,(-2t\sp{-3})-{3\over t\sp2}\,(t\sp{-2})=0$\\
$6t\sp{-4}-3t\sp{-4}-3t\sp{-4}=0$\\
$0=0$\\
The terms of the ODE cancel nicely and sum to 0, so $y_1(t)$ is a valid solution!\\\\
Now, find the second solution $y_2(t)$ via reduction of order. It will have the form:\\
$y_2(t)=y_1(t)h(t)$\\
$y_2(t)=t\sp{-2}h$\\
Find the derivatives $y_2\sp\prime(t)$ and $y_2\sp{\prime\prime}(t)$:\\
$y_2\sp\prime(t)=-2t\sp{-3}h + t\sp{-2}h\sp\prime$\\

$y_2\sp{\prime\prime}(t)=(6t\sp{-4}h - 2t\sp{-3}h\sp\prime) + (-2t\sp{-3}h\sp\prime + t\sp{-2}h\sp{\prime\prime})$\\
$y_2\sp{\prime\prime}(t)=6t\sp{-4}h - 4t\sp{-3}h\sp\prime + t\sp{-2}h\sp{\prime\prime}$\\

Plug these into the ODE:\\
$y\sp{\prime\prime}+{3\over2t}\,y\sp\prime-{3\over t\sp2}\,y=0$\\
$(\textcolor{red}{6t\sp{-4}h} - 4t\sp{-3}h\sp\prime + t\sp{-2}h\sp{\prime\prime})+{3\over2t}\,(\textcolor{red}{-2t\sp{-3}h} + t\sp{-2}h\sp\prime)-{3\over t\sp2}\,(\textcolor{red}{t\sp{-2}h})=0$\\
The \textcolor{red}{red terms} above cancel with each other, giving:\\
$ - 4(t\sp{-3}h\sp\prime) + t\sp{-2}h\sp{\prime\prime}+{3\over2}\,( t\sp{-3}h\sp\prime)=0$\\
$ t\sp{-2}h\sp{\prime\prime} - {5\over2}(t\sp{-3}h\sp\prime) =0$\\

Perform a \textbf{change of variable} where $w = h\sp\prime$ and $w\sp\prime = h\sp{\prime\prime}$\\
$ t\sp{-2}w\sp{\prime} - {5\over2}t\sp{-3}w =0$\\
Now we have a first order ODE which can be solved:\\
$ t\sp{-2}w\sp{\prime} = {5\over2}t\sp{-3}w $\\
$ {1\over w}w\sp{\prime} = {5\over2}t\sp{-1} $\\
$ \int{{1\over w}dw} = \int{{5\over2}t\sp{-1}dt} $\\
$ \ln|w| = {5\over2}\ln|t|+C $\\
$ w = Ce\sp{{5\over2}\ln|t|} = Ct\sp{5\over2} $\\
Because of our change of variable, $h(t) = \int{w\,dt}$. Solve for $h(t)$:\\
$h(t) = \int{Ct\sp{5\over2}\,dt}$\\
$h(t) = C{2\over7}t\sp{7\over2}+k$\\
Can choose the constants $C$ and $k$ to be anything we want, so choose $C={7\over2}$ and $k=0$ for convenience:\\
$h(t) = t\sp{7\over2}$\\

Now, the second solution $y_2(t) = y_1(t)h(t) = t\sp{-2}(t\sp{7\over2})$:\\
$\bm{y_2(t) = t\sp{3\over2}}$\\

\item
Compute the Wronskian of $y_1(t)$ and $y_2(t)$ to show that the solutions are independent (and thus form a fundamental set of solutions).\\

$y_1(t) = t\sp{-2}$ and $y_2(t) = t\sp{3\over2}$\\
$y_1\sp\prime(t) = -2t\sp{-3}$ and $y_2\sp\prime(t) = {3\over2}t\sp{1\over2}$\\

$W(t)=\det\begin{bmatrix}y_1(t) & y_2(t) \\ y_1\sp\prime(t) & y_2\sp\prime(t)\end{bmatrix} = \det\begin{bmatrix}t\sp{-2} & t\sp{3\over2} \\ -2t\sp{-3} & {3\over2}t\sp{1\over2}\end{bmatrix}$\\
$W(t)=(t\sp{-2}*{3\over2}t\sp{1\over2}) - (t\sp{3\over2}*-2t\sp{-3})$\\
$W(t)={3\over2}t\sp{-3\over2} + 2t\sp{-3\over2}$\\
\textbf{No value of $\bm{t>0}$ can make the above Wronskian term $\bm{= 0}$, so the solutions are independent (and could be combined to form a general one).}\\

\end{enumerate}

\item (15 points)  Consider the initial-value problem
\[
y\sp{\prime\prime}+4y\sp\prime+13y=0,\qquad y(0)=-1,\quad y\sp\prime(0)=5
\]
\begin{enumerate}
\item Find real-valued solutions $y_1(t)$ and $y_2(t)$ in the general solution $y(t)=C_1y_1(t)+C_2y_2(t)$ of the constant-coefficient ODE, and then apply the initial conditions to determine the constants in the general solution.\\

Because this is a constant-coefficient, linear, homogeneous, 2nd-order ODE, we can perform a substitution assuming an answer comes in the form $y(t)=e\sp{rt}$. Let $y(t)=e\sp{rt}$ in the original ODE, and simplify to find:\\

$r\sp{2}+4r+13=0$\\
Since this is a quadratic polynomial, we can find the root(s) $r$ using the quadratic formula. In this case, $a=1$, $b=4$, and $c=13$.\\
$r=\frac{-b\pm\sqrt{b^2-4ac}}{2a}$\\
$r=\frac{-4\pm\sqrt{16-(4*13)}}{2}$\\
$r=-2\pm\frac{\sqrt{-36}}{2}$\\
$r=-2\pm3i$\\

From this, we know there will be two solutions:\\
$y_1(t) = e\sp{(-2+3i)t}$ and $y_2(t) = e\sp{(-2-3i)t}$\\
To find the real versions of these (without an imaginary component), use Euler's formula. End up with:\\
$y_1(t) = e\sp{\lambda t}\cos(\omega t)$ and $y_2(t) = e\sp{\lambda t}\sin(\omega t)$\\
$y_1(t) = e\sp{-2t}\cos(3t)$ and $y_2(t) = e\sp{-2t}\sin(3t)$\\

For the general solution, perform a linear combination to get:\\
$y(t) = C_{1}e\sp{-2t}\cos(3t) + C_{2}e\sp{-2t}\sin(3t)$\\
Before applying the ICs to determine $C_1$ and $C_2$, determine $y\sp\prime(t)$ using the product rule for each term:\\
$y\sp\prime(t) = C_{1}(-2e\sp{-2t}\cos(3t) - 3e\sp{-2t}\sin(3t))+ C_{2}(-2e\sp{-2t}\sin(3t) + 3e\sp{-2t}\cos(3t))$\\

Plug in $y(0)=-1$ for $y(t)$:\\
$-1 = C_{1}e\sp{0}\cos(0) + C_{2}e\sp{0}\sin(0)$\\
$-1 = C_{1}(1)(1) + C_{2}(1)(0)$\\
$C_{1} = -1$\\

Plug in $y\sp\prime(0)=5$ and $C_{1} = -1$ for $y\sp\prime(t)$ in order to determine $C_2$:\\
$y\sp\prime(t) = -(-2e\sp{0}\cos(0) - 3e\sp{0}\sin(0))+ C_{2}(-2e\sp{0}\sin(0) + 3e\sp{0}\cos(0))$\\
$5 = -(-2 - 0)+ C_{2}(0 + 3)$\\
$5 = 2+ 3C_{2}$\\
$C_2 = 1$\\

\textbf{Given $\bm{C_{1} = -1}$ and $\bm{C_{2} = 1}$, the overall solution $\bm{y(t)}$ is:}\\
$\bm{y(t) = -e\sp{-2t}\cos(3t) + e\sp{-2t}\sin(3t)}$\\


\item Write the solution in part (a) in the \lq\lq polar'' form $y(t)=Re\sp{\lambda t}\cos(\omega t-\phi)$ following an example discussed in class.  Give the constants $R$, $\lambda$, $\omega$ and $\phi$, and use the polar form to sketch the solution.\\

$C_1 = R\cos(\phi)$ and $C_2 = R\sin(\phi)$\\
So $-1 = R\cos(\phi)$ and $1 = R\sin(\phi)$\\
Rearrange and find that: $R ={-1\over \cos(\phi)}$. Plug in to the 2nd equation:\\
$1 = ({-1\over \cos(\phi)})\sin(\phi)$\\
$-1 = {\sin(\phi)\over \cos(\phi)} = \tan(\phi)$\\
$\tan(\phi) = -1$\\
Assume $\bm{\phi = -{\pi \over4}}$ and that only, rather than entertain other possible alternatives.\\
%$\bm{\phi = -{\pi \over4}\pm2n\pi}$ \textbf{OR} $\bm{\phi = {3\pi \over 4}\pm 2n\pi}$\\
Now, solve for $R$:\\
$-1 = R\cos(-{\pi \over 4})$\\
$R = {-1\over \cos(-{\pi \over 4})}$\\
$R = {-1\over \cos(-{\pi \over 4})} = {-1\over {1 \over \sqrt{2}}} = -\sqrt{2}$\\
$R = -\sqrt{2}$\\

From part (a) above, we know that $\bm{\lambda = -2}$ and $\bm{\omega = 3}$\\

\textbf{The polar form of the solution is:}\\
$\bm{y(t)=-\sqrt{2}e\sp{-2t}\cos(3t+{\pi\over4})} $\\

Will touch the envelope bounds when $(3t+{\pi\over4})=0$ or $\pi$, because that is where the $cos()$ term will be maximized/minimized from 1 to -1.\\
\begin{tabular}{ c c } 
 $3t+{\pi\over4}=0$ & $3t+{\pi\over4}=\pi$ \\ 
 $t=-{\pi\over12}$ & $t={\pi\over4}$ \\ 
\end{tabular}

So, the curve will touch either one of the envelope bounds $\sqrt{2}e\sp{-2t}$ or $-\sqrt{2}e\sp{-2t}$ when $t = -{\pi\over12}$ or $t={\pi\over4}$.\\
When $t=0$, the envelope bounds will $= \sqrt{2}$ and $-\sqrt{2}$. The y-intercept of the curve occurs at $y(0) = -1$.\\

At $t = -{\pi\over12}$:\\
$y(-{\pi\over12})=-\sqrt{2}e\sp{2\pi \over 12}\cos(-{3\pi \over 12}+{\pi\over4}) = -\sqrt{2}e\sp{\pi \over 6}\cos(0)$\\
$y(-{\pi\over12})=-\sqrt{2}e\sp{\pi \over 6}(1) = -\sqrt{2}e\sp{\pi \over 6}$\\

At $t = {\pi\over4}$:\\
$y({\pi\over4})=-\sqrt{2}e\sp{-{\pi\over2}}\cos({3\pi\over4}+{\pi\over4}) = -\sqrt{2}e\sp{-{\pi\over2}}\cos(\pi)$\\
$y({\pi\over4})= -\sqrt{2}e\sp{-{\pi\over2}}(-1) = \sqrt{2}e\sp{-{\pi\over2}}$\\\\

Graph would look like:\\
\begin{figure}[h]
%\includegraphics[scale=1.2]{scancropped}
\centering
\end{figure}

\end{enumerate}


\newpage
\item (15 points)
\begin{enumerate}
\item
Find $y(t)$ satisfying the initial-value problem
\[
9y\sp{\prime\prime}+6y\sp\prime+y=0,\qquad y(0)=1,\quad y\sp\prime(0)={2\over3}
\]
Since this is a constant-coefficient, 2nd-order, linear, homogeneous ODE, we can perform a substitution assuming an answer comes in the form $y(t)=e\sp{rt}$. Let $y(t)=e\sp{rt}$ in the original ODE, and simplify to find:\\
$9r\sp{2} + 6r + 1 = 0$\\

Solve this quadratic in order to find its roots, $r_1$ and $r_2$.\\
It appears to be factorable:\\
$(3r + 1)(3r + 1) = 0$\\
$ 3r + 1 = 0$\\
$ r = -\frac{1}{3}$ (double root)\\

For this case of a double root, there will still be two solutions, with the second of a slightly different form:\\
$y_1(t) = e\sp{rt}$ and $y_2(t) = te\sp{rt}$\\
$y_1(t) = e\sp{-\frac{1}{3}t}$ and $y_2(t) = te\sp{-\frac{1}{3}t}$\\

Perform a linear combination to find the general solution $y(t)$:\\
$y(t) = C_{1}e\sp{-\frac{1}{3}t} + C_{2}te\sp{-\frac{1}{3}t}$\\

Before applying the ICs to determine $C_1$ and $C_2$, determine $y\sp\prime(t)$ by taking the first-derivative of each term:\\
$y\sp\prime(t) = -\frac{1}{3}C_{1}e\sp{-\frac{1}{3}t} + C_{2}(e\sp{-\frac{1}{3}t} - \frac{1}{3}te\sp{-\frac{1}{3}t})$\\

Plug in $y(0)=1$ for $y(t)$:\\
$1 = C_{1}e\sp{0} + C_{2}(0)e\sp{0}$\\
$C_{1} = 1$\\

Plug in $y\sp\prime(0)=\frac{2}{3}$ and $C_{1} = 1$ for $y\sp\prime(t)$:\\
$\frac{2}{3} = -\frac{1}{3}(1)e\sp{0} + C_{2}(e\sp{0} - \frac{1}{3}(0)e\sp{0})$\\\\
$\frac{2}{3} = -\frac{1}{3} + C_{2}(1 - 0)$\\\\
$ -\frac{1}{3} + C_{2}= \frac{2}{3} $\\
$ C_{2}= 1 $\\

\textbf{Given $\bm{C_{1} = 1}$ and $\bm{C_{2} = 1}$, the overall solution $\bm{y(t)}$ is:}\\
$y(t) = e\sp{-\frac{1}{3}t} + te\sp{-\frac{1}{3}t}$\\
$\bm{y(t) = (1+t)e\sp{-\frac{1}{3}t}}$\\

\item
Find real-valued functions $u_1(x)$ and $u_2(x)$ in the general solution $u(x)=C_1u_1(x)+C_2u_2(x)$ of the Cauchy-Euler equation
\[
4x\sp2u\sp{\prime\prime}+8xu\sp\prime+u=0,\qquad x>0
\]
Find $C_1$ and $C_2$ in the general solution so that $u(1)=0$ and $u\sp\prime(1)=3$.\\

\newpage
Let $u = x\sp{r}$, the form of a solution, and substitute into the ODE (while also performing simple power rule derivatives for $x\sp{r}$):\\
$ax\sp2(r(r-1)x\sp{r-2})+bx(rx\sp{r-1})+c(x\sp{r})=0$\\
$ax\sp2(r(r-1)x\sp{r-2})+bx(rx\sp{r-1})+c(x\sp{r})=0$\\

In this Cauchy-Euler equation, the constants are as follows:\\
$a=4$, $b=8$, $c=1$\\
The above simplifies to:\\
$x\sp{r}(ar(r-1)+br+c) = 0$\\
The $x\sp{r}$ term cannot be 0, so:\\
$(ar(r-1)+br+c) = 0$\\
Plug in our equation's constants:\\
$(4r(r-1)+8r+1) = 0$\\
$4r\sp{2} +8r - 4r + 1= 0$\\
$4r\sp{2} +4r + 1= 0$\\
$(2r+1)(2r+1)=0$\\
$r = -1/2 = -0.5$, a double root\\

Since the root is real and double, solutions will take the form:\\
$u_1(x) = x\sp{r}$ and $u_2(x) = x\sp{r}\ln(x)$\\
General solution would look like:\\
$u(x) = C_1x\sp{r} + C_2x\sp{r}\ln(x)$\\
$u(x) = C_1x\sp{-0.5} + C_2x\sp{-0.5}\ln(x)$\\

And the first-derivative would be:\\
$u\sp\prime(x) = -0.5C_1x\sp{-1.5} + (-0.5C_2x\sp{-1.5}\ln(x) + {C_2x\sp{-0.5} \over x})$\\
$u\sp\prime(x) = -0.5C_1x\sp{-1.5} - 0.5C_2x\sp{-1.5}\ln(x) + C_2x\sp{-1.5}$\\

Plug in IC $u(1)=0$ to $u(x)$:\\
$0 = C_1(1)\sp{-0.5} + C_2(1)\sp{-0.5}\ln(1)$\\
$0 = C_1 + C_2(0)$\\
\textbf{So} $\bm{C_1 = 0}$\\

Now, plug in IC $u\sp\prime(1)=3$ and $C_1 = 0$ to $u\sp\prime(x)$:\\
$3 = -0.5(0)(1)\sp{-1.5} - 0.5C_2(1)\sp{-1.5}\ln(1) + C_2(1)\sp{-1.5}$\\
$3 = 0 - 0 + C_2$\\
$\bm{C_2 = 3}$\\

\textbf{Now, the overall solution is:}\\
$\bm{u(x) =  3x\sp{-0.5}\ln(x)}$\\

\end{enumerate}


\end{enumerate}


\end{document}

