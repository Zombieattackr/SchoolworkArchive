\documentclass{article}

\usepackage{amsmath}
\usepackage{amssymb}
\usepackage{bm}
\usepackage{graphicx}
\usepackage{epstopdf}
\DeclareGraphicsRule{.tif}{png}{.png}{`convert #1 `basename #1 .tif`.png}
\usepackage{color}
\usepackage{pdfsync}
\pagestyle{plain}

\textheight 9 true in
\textwidth 6.5 true in
\hoffset -.75 true in
\voffset -.75 true in
\mathsurround=2pt
\parskip=2pt

\begin{document}

\begin{center}
\large{ MATH-2400 \hspace{.27in}  INTRODUCTION TO DIFFERENTIAL EQUATIONS \hspace{.27in}FALL 2022\bigskip\\ {\bf Problem Set 3} \smallskip\\ Due: 5pm, Friday, September 23, 2022}\\
\textbf{Submitted to LMS By:\\ Joseph Hutchinson\\ 662022852 \\ Section 17 }
\end{center}

\bigskip\noindent
\underline{NOTES}
\begin{enumerate}
\item Practice problems listed below and taken from the textbook are for your own practice, and are not to be turned in.
\item There are two parts of the Problem Set, an objective part consisting of multiple choice questions (with no partial credit available) and a subjective part (with partial credit possible).  Please complete all questions.
\item Writing your solutions in {\LaTeX} is preferred but not required.
\item Show all work for problems in the subjective part.  Illegible or undecipherable solutions will not be graded. 
\item Figures, if any, should be neatly drawn by hand, properly labelled and captioned.  
\item Your completed work is to be submitted electronically to LMS  as a \textcolor{red}{single pdf file}. Be sure that the pages are properly oriented and well lighted.  (\textcolor{blue}{Please do not e-mail your work to Muhammad or me.})
\end{enumerate}

\bigskip\noindent
{\bf Practice Problems from the textbook} (Not to be turned in)
\begin{itemize}
\item
Exercises from Chapter 2, page 38--40: 2(a,c,f), 3(a,c), 10(a,b), 12.
\item
Exercises from Chapter 3, page 44: 1, 2, 4.
\item
Exercises from Chapter 3, pages 50--51: 1, 2, 3(e,g), 4(c,g), 6(a,b).
\end{itemize}

\bigskip\noindent
{\bf Objective part} (Choose A, B, C or D; no work need be shown, no partial credit available)

\begin{enumerate}

\item (5 points) A population $y(t)$ satisfies the IVP $y'=y^2-4y+3$, $y(0)=y_0$.  Which of the following choices describes the behavior of the population?
\begin{description}
\item[A] $y(t)=3$ if $y_0=3$
\item[B] $y(t)\rightarrow1$ as $t\rightarrow\infty$ if $y_0=2$		
\item[C] $y(t)\rightarrow+\infty$ as $t\rightarrow\infty$ if $y_0=5$
\item\textcolor{red}{\textbf{[D] All of these choices}}
\end{description}

\item (5 points) A population $y(t)$ solves $y\sp\prime=f(y)$, and has a semi-stable equilibrium at $y=0$, an unstable equilibrium at $y=1$, and an asymptotically stable equilibrium at $y=2$.  Which function $f(y)$ best describes the behavior of the population?
\begin{description}
\item[A] $f(y)=3y\sp2(y-1)(y-2)$
\item\textcolor{red}{\textbf{[B] }$\bm{f(y)=2y(y\sp2-y)(2-y)}$}
\item[C] $f(y)=4(y\sp2-y)(y\sp2-2y)$
\item[D] $f(y)=5(y\sp2-1)(y\sp2-2)$
\end{description}

\newpage
\item (5 points) Consider the linear equation $(\sin t)y\sp{\prime\prime}+(\ln(t-1))y\sp\prime-(\sqrt{5-t})y=0$ with initial conditions $y(4)=0$ and $y\sp\prime(4)=-1$.  For which interval does a unique solution of the IVP exist?
\begin{description}
\item[A] $1<t<5$
\item[B] $1<t<\pi/2$		
\item[C] $\pi<t<5$
\item[D] None of these choices
\end{description}

\end{enumerate}

\bigskip\noindent
{\bf Subjective part} (Show work, partial credit available)

\begin{enumerate}

\item (15 points) A population $y(t)$ of fish in a lake satisfies the rate equation
\[
y'=f(y),\qquad f(y)=2y\left(1-{y\over5}\right)-H,
\]
where $H\ge0$ is a (constant) rate at which fish are removed from the lake due to an active group of fishermen.
\begin{enumerate}
\item
Plot $f(y)$ versus $y$ for the case $H=0$ and let $y_0$ denote the stable equilibrium population of fish.  Determine $y_0$ from the phase plot.  This value corresponds to the stable population when there is no fishing.\\

For the case where there is no hunting, and $H=0$, the stable equilibrium population of fish $y_0 = 5$ fish in the lake.\\

\item
At the beginning of a fishing season, the fishermen become energetic and begin catching fish at a rate $H=1$.  Plot $f(y)$ versus $y$ for $H=1$ and let $y_1$ denote the new stable equilibrium population of fish assuming $y(0)=y_0$.  Determine $y_1$.\\

For the case where there is some hunting and $H=1$, the stable equilibrium population of fish $y_0$ approaches the value 4.5 fish in the lake.\\

\item
Note how the phase plots change from $H=0$ to $H=1$, and let $H_c$ denote a critical fishing rate such that for $H>H_c$ the fish population tends to zero for any initial state.  Determine $H_c$, which corresponds to the maximum allowable fishing rate that supports a nonzero fish population in the lake.\\

Somewhere between Hc 2.4-2.5 there is a breakdown such that any initial condition leads to a population of zero fish in the lake.

\end{enumerate}

%\bigskip
\newpage
\item (15 points) Let $y_1(t)=\sqrt{t}$ and $y_2(t)=t\sp{-2}$, and consider the linear, homogeneous, second-order ODE
\[
y\sp{\prime\prime}+{5\over2t}\,y\sp\prime-{1\over t\sp2}\,y=0,\qquad t>0
\]
\begin{enumerate}
\item
Verify that $y_1(t)$ and $y_2(t)$ are solutions of the ODE, and compute the Wronskian of $y_1(t)$ and $y_2(t)$ to show that the solutions are independent (and thus form a fundamental set of solutions).\\

For $y_1(t)$, find ${y_1}\sp\prime(t)$ and ${y_1}\sp{\prime\prime}(t)$, then substitute into the ODE to check if it solves for $t>0$:\\
${y_1}\sp\prime(t) = \frac{d}{dt}[t\sp{(1/2)}] = \frac{1}{2}t\sp{(-1/2)}$\\
${y_1}\sp{\prime\prime}(t) = \frac{d}{dt}[\frac{1}{2}t\sp{(-1/2)}] = -\frac{1}{4}t\sp{(-3/2)}$\\

Substitute into the ODE to check:\\
$y\sp{\prime\prime}+{5\over2t}\,y\sp\prime-{1\over t\sp2}\,y=0$\\
$(-\frac{1}{4}t\sp{(-3/2)})+{5\over2t}\,(\frac{1}{2}t\sp{(-1/2)})-{1\over t\sp2}\,(t\sp{(1/2)})=0$\\
$-\frac{1}{4}t\sp{(-3/2)}+{5\over4}t\sp{(-3/2)}-t\sp{(-3/2)}=0$\\

This checks out! The terms all sum to 0, and it works for values of $t>0$, so $y_1(t)$ is a valid solution.\\

For $y_2(t)$, find ${y_2}\sp\prime(t)$ and ${y_2}\sp{\prime\prime}(t)$, then substitute into the ODE to check if it solves for $t>0$:\\
${y_2}\sp\prime(t) = \frac{d}{dt}[t\sp{-2}] = -2t\sp{-3}$\\
${y_2}\sp{\prime\prime}(t) = \frac{d}{dt}[-2t\sp{-3}] = 6t\sp{-4}$\\

Substitute into the ODE to check:\\
$y\sp{\prime\prime}+{5\over2t}\,y\sp\prime-{1\over t\sp2}\,y=0$\\
$(6t\sp{-4})+{5\over2t}\,(-2t\sp{-3})-{1\over t\sp2}\,(t\sp{-2})=0$\\
$6t\sp{-4}-5t\sp{-4}-t\sp{-4}=0$\\

This one also checks out! The terms all sum to 0, and it works for values of $t>0$, so $y_2(t)$ is a valid solution.\\

\textcolor{red}{CHECK THE WRONSKIAN}\\

\item
The general solution of the ODE has the form $y(t)=C_1y_1(t)+C_2y_2(t)$, where $C_1$ and $C_2$ are constants.  Find the constants satisfying the initial conditions $y(1)=2$ and $y\sp\prime(1)=-1$.\\

$y(t)=C_1(t\sp{(1/2)})+C_2(t\sp{-2})$\\
Plug in the IC $y(1)=2$:\\
$2=C_1(1)\sp{(1/2)}+C_2(1)\sp{-2}$\\
$C_1 + C_2 = 2$\\
$C_1 = 2 - C_2$\\


Find $y\sp\prime(t)$ and then apply the second IC:\\
$y(t)=C_1(t\sp{(1/2)})+C_2(t\sp{-2})$\\
$y\sp\prime(t)=\frac{1}{2}C_1(t\sp{(-1/2)}) - 2C_2(t\sp{-3})$\\
Plug in the IC $y\sp\prime(1)=-1$:\\
$-1=\frac{1}{2}C_1(1)\sp{(-1/2)} - 2C_2(1)\sp{-3}$\\
$\frac{1}{2}C_1 - 2C_2 = -1$\\
Plug in $C_1 = 2 - C_2$, found above:\\
$\frac{1}{2}(2 - C_2) - 2C_2 = -1$\\
$1 - \frac{1}{2}C_2 - 2C_2 = -1$\\
$- \frac{5}{2}C_2 = -2$\\
$C_2 = -2(-\frac{2}{5})$\\
$\bm{C_2 = \frac{4}{5}}$\\

Now, solve for $C_1$:\\
$C_1 = 2 - C_2$\\
$C_1 = 2 - (\frac{4}{5})$\\
$\bm{C_1 = \frac{6}{5}}$\\

\textbf{So, the general solution of the ODE is:}\\
$\bm{y(t)=\frac{6}{5}t\sp{(1/2)}+\frac{4}{5}t\sp{-2}}$\\
\end{enumerate}


\bigskip
\item (15 points) Consider the two constant-coefficient, second-order ODEs:
\[
\hbox{(DE 1)}\quad y\sp{\prime\prime}+2y\sp\prime-8y=0\qquad\qquad\hbox{(DE 2)}\quad 3y\sp{\prime\prime}+4y\sp\prime-4y=0
\]
\begin{enumerate}
\item Find all solutions of the form $y(t)=e\sp{rt}$, where $r$ is a constant, for both ODEs.\\

For this subset of special ODEs, the solutions are of a predictable form: $y(t)=e\sp{rt}$\\
Since we know this is the case, we can substitute in this general form for any $y(t)$ in the ODEs.\\

\textbf{For (DE 1):}\\
$y\sp{\prime\prime}+2y\sp\prime-8y=0$\\
$(e\sp{rt})\sp{\prime\prime}+2(e\sp{rt})\sp\prime-8(e\sp{rt})=0$\\
Evaluate each derivative (in prime notation above) with respect to $t$:\\
$r\sp{2}(e\sp{rt})+2r(e\sp{rt})-8(e\sp{rt})=0$\\
Since $e\sp{rt}$ is a leftover common term, factor it out:\\
$(e\sp{rt})(r\sp{2}+2r-8)=0$\\
In order to find solutions, we're looking for zeroes/roots. Since the term $e\sp{rt}$ can never $=0$ for values of $t$, the roots must be due to the quadratic.\\
Take $(r\sp{2}+2r-8)=0$ and factor:\\
$(r+4)(r-2)=0$\\

\textbf{Roots are $\bm{r=-4}$ and $\bm{r=2}$, so there are two valid solutions for (DE 1):}\\
\textbf{$\bm{y_{1}(t) = e\sp{-4t}}$ and $\bm{y_{2}(t) = e\sp{2t}}$}\\

\textbf{For (DE 2):}\\
$3y\sp{\prime\prime}+4y\sp\prime-4y=0$\\
$3(e\sp{rt})\sp{\prime\prime}+4(e\sp{rt})\sp\prime-4(e\sp{rt})=0$\\
Evaluate each derivative (in prime notation above) with respect to $t$:\\
$3r\sp{2}(e\sp{rt})+4r(e\sp{rt})-4(e\sp{rt})=0$\\
Since $e\sp{rt}$ is a leftover common term, factor it out:\\
$(e\sp{rt})(3r\sp{2}+4r-4)=0$\\
In order to find solutions, we're looking for zeroes/roots. Since the term $e\sp{rt}$ can never $=0$ for values of $t$, the roots must be due to the quadratic.\\
Take $(3r\sp{2}+4r-4)=0$ and apply the quadratic formula:\\
Where $a=3$, $b=4$, $c=-4$\\
$r=\frac{-b\pm\sqrt{b^2-4ac}}{2a}=\frac{-(4)\pm\sqrt{(4)^2-4(3)(-4)}}{2(3)}=\frac{-4\pm\sqrt{16+48}}{6}=\frac{-4\pm8}{6}$\\
$r=\frac{4}{6}$ or $r=\frac{-12}{6}$\\\\
$r=\frac{2}{3}$ or $r=-2$\\

\textbf{Roots are $\bm{r=\frac{2}{3}}$ and $\bm{r=-2}$, so there are two valid solutions for (DE 2):}\\
\textbf{$\bm{y_{3}(t) = e\sp{(2/3)t}}$ and $\bm{y_{4}(t) = e\sp{-2t}}$}\\

\item Find the solution satisfying the initial conditions $y(0)=0$ and $y\sp\prime(0)=1$ for both ODEs.\\

Since the term $e\sp{rt}\neq0$, an individual solution of an ODE cannot satisfy the initial conditions here. Each ODE must then have one \textit{overall} solution given these ICs, which will be based on the combination (sum) of its previously found 2 solutions. By adding the individual solutions like $y_1(t)$ and $y_2(t)$ for (DE 1), and assuming they have constant coefficients, the coefficients could feasibly sum to $=0$, as is necessary.\\

\textbf{For (DE 1), this looks like:}\\
$y(t) = C_{1}e\sp{-4t} + C_{2}e\sp{2t}$\\
Substitute $y(0)=0$ into $y(t)$:\\
$0 = C_{1}e\sp{0} + C_{2}e\sp{0}$\\
$C_{1} + C_{2}=0$\\
Cannot solve for both constants with only one equation, so determine $y\prime(t)$ and substitute in the second IC, $y\prime(0)=1$:\\
$y\prime(t) = (-4)C_{1}e\sp{-4t} + (2)C_{2}e\sp{2t}$\\
$1 = (-4)C_{1}e\sp{0} + (2)C_{2}e\sp{0}$\\
$-4C_{1} + 2C_{2} = 1$\\

So $C_{2}=\frac{1+4C_{1}}{2}$\\
Plug this into $C_{1} + C_{2}=0$:\\
$C_{1} + \frac{1+4C_{1}}{2}=0$\\
$C_{1} + \frac{1}{2}+2C_{1}=0$\\
$3C_{1}=-\frac{1}{2}$\\
$\bm{C_{1}=-\frac{1}{2}*\frac{1}{3} = -\frac{1}{6}}$\\
Now, solve for $C_{2}$ as well\\
$C_{1} + C_{2}=0$\\
$-\frac{1}{6} + C_{2}=0$\\
$\bm{C_{2}=\frac{1}{6}}$\\

\textbf{So the final solution for (DE 1) is:}\\
$\bm{y(t) = -\frac{1}{6}e\sp{-4t} + \frac{1}{6}e\sp{2t}}$\\

%%%%%%%%% DE %$$$$$$$$$$$$$$$$$$
\rule{15cm}{2pt}\\
\textbf{For (DE 2), the general solution looks like:}\\
$y(t) = C_{3}e\sp{(2/3)t} + C_{4}e\sp{-2t}$\\
Substitute $y(0)=0$ into $y(t)$:\\
$0 = C_{3}e\sp{0} + C_{4}e\sp{0}$\\
$C_{3} + C_{4}=0$\\
$C_{3}= -C_{4}$\\

Cannot solve for both constants with only one equation, so determine $y\prime(t)$ and substitute in the second IC, $y\sp\prime(0)=1$:\\
$y\sp\prime(t) = (2/3)C_{3}e\sp{(2/3)t} + (-2)C_{4}e\sp{-2t}$\\
$1 = (2/3)C_{3}e\sp{0} + (-2)C_{4}e\sp{0}$\\
$\frac{2}{3}C_{3} - 2C_{4} = 1$\\

Plug in $C_{3}= -C_{4}$:\\
$-\frac{2}{3}C_{4} - 2C_{4} = 1$\\\\
$-\frac{2}{3}C_{4} - \frac{6}{3}C_{4} = 1$\\\\
$-\frac{8}{3}C_{4}= 1$\\
$\bm{C_{4}= -\frac{8}{3}}$\\

Now, solve for $C_{3}$ as well\\
$C_{3}= -C_{4}$\\
$\bm{C_{3}= \frac{8}{3}}$\\

\textbf{So the final solution for (DE 2) is:}\\
$\bm{y(t) = \frac{8}{3}e\sp{(2/3)t} - \frac{8}{3}e\sp{-2t}}$\\

\end{enumerate}



\end{enumerate}


\end{document}

