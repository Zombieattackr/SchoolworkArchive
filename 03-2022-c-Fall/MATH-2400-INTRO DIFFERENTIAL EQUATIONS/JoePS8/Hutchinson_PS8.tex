\documentclass{article}

\usepackage{amsmath}
\usepackage{amssymb}
\usepackage{bm}
\usepackage{graphicx}
\usepackage{epstopdf}
\DeclareGraphicsRule{.tif}{png}{.png}{`convert #1 `basename #1 .tif`.png}
\usepackage{color}
\usepackage{xcolor}
\usepackage{pdfsync}
\pagestyle{plain}

\textheight 9 true in
\textwidth 6.5 true in
\hoffset -.75 true in
\voffset -.75 true in
\mathsurround=2pt
\parskip=2pt

\begin{document}

\begin{center}
\large{ MATH-2400 \hspace{.27in}  INTRODUCTION TO DIFFERENTIAL EQUATIONS \hspace{.27in}FALL 2022\bigskip\\ {\bf Problem Set 8} \smallskip\\ Due: 11pm, Tuesday, November 15, 2022}\\
\textbf{\\Submitted to LMS By:\\ Joseph Hutchinson\\ 662022852 \\ Section 17 }
\end{center}

\bigskip\noindent
\underline{NOTES}
\begin{enumerate}
\item Practice problems listed below and taken from the textbook are for your own practice, and are not to be turned in.
\item There are two parts of the Problem Set, an objective part consisting of multiple choice questions (with no partial credit available) and a subjective part (with partial credit possible).  Please complete all questions.
\item Writing your solutions in {\LaTeX} is preferred but not required.
\item Show all work for problems in the subjective part.  Illegible or undecipherable solutions will not be graded. 
\item Figures, if any, should be neatly drawn by hand, properly labelled and captioned.  
\item Your completed work is to be submitted electronically to LMS  as a \textcolor{red}{single pdf file}. Be sure that the pages are properly oriented and well lighted.  (\textcolor{blue}{Please do not e-mail your work to Muhammad or me.})
\end{enumerate}

\bigskip\noindent
{\bf Practice Problems from the textbook} (Not to be turned in)
\begin{itemize}
\item
Exercises from Chapter 7, pages 179--180: 1(a,c,e), 2(a,c), 3(b,d,e).
\item
Exercises from Chapter 7, pages 186--187: 1(b,d), 2(b,d), 5(b,d), 7(b,d)
\end{itemize}

\bigskip\noindent
{\bf Objective part} (Choose A, B, C or D; no work need be shown, no partial credit available)

\begin{enumerate}
\item (5 points)  Identify the correct statement, or select \lq\lq None of these choices'' if none of the statements are correct.
\begin{description}
\item\textcolor{red}{\textbf{[A] The only solution of the DE $\bm{y\sp{\prime\prime}-y=0}$ with BCs $\bm{y(0)=0}$ and $\bm{y(\pi)=0}$ is the trivial solution.}}
% A is CORRECT, because we get constants such that $y(x)=0$ is the only solution (trivial)
\item[B] The only solution of the DE $y\sp{\prime\prime}+y=0$ with BCs $y(0)=0$ and $y(\pi)=0$ is the trivial solution.
% B is incorrect, as there is the infinite set of solutions $y(x) = C_2 sin(x)$, meaning that the trivial is NOT the only solution
\item[C] The DE $x\sp2y\sp{\prime\prime}+xy\sp\prime+y=0$ with BCs $y(1)=0$ and $y(2)=0$ has nontrivial solutions.
% C is incorrect, because the only working solution is the trivial, where $y(x)=0$
\item[D] None of these choices
\end{description}

\item (5 points) Identify the correct statement, or select \lq\lq All of these choices'' if all of the statements are correct.  For each statement, $\lambda$ is a constant.
\begin{description}
\item[A] Setting $u(x,y)=F(x)G(y)$ in the PDE $u_{xx}+u_{yy}=0$ leads to the separated equations $F\sp{\prime\prime}-\lambda F=0$ and $G\sp{\prime\prime}+\lambda G=0$.
% A is correct
\item[B] Setting $v(x,t)=F(x)G(t)$ in the PDE $v_{tt}=v_{xx}$ leads to the separated equations $F\sp{\prime\prime}+\lambda F=0$ and $G\sp{\prime\prime}+\lambda G=0$.
% B is correct
\item[C] Setting $w(r,t)=F(r)G(t)$ in the PDE $w_{t}=w_{rr}+{1\over r}w_r$ leads to the separated equations\\ $rF\sp{\prime\prime}+F\sp{\prime}+\lambda rF=0$ and $G\sp{\prime}+\lambda G=0$.
% C is correct
\item\textcolor{red}{\textbf{[D] All of these choices}}
% Since A,B,C are all correct, D must be the answer
\end{description}

\end{enumerate}

\newpage\noindent
{\bf Subjective part} (Show work, partial credit available)

\begin{enumerate}

\item (15 points)  For each boundary-value problem, determine whether or not a solution exists.  If a solution exists, then determine whether or not it is unique.
\[
\begin{array}{lc}
\hbox{(a)}\qquad & y\sp{\prime\prime}+2y\sp\prime-3y=4e\sp{x},\qquad y(0)=0,\quad y\sp\prime(1)=0 \medskip\\

\hbox{(b)}\qquad & y\sp{\prime\prime}+4y=6\cos4x,\qquad y\sp\prime(0)=0,\quad y\sp\prime(\pi)=0
\end{array}
\]
\textbf{(a)} Need to find the general solution. Find homogeneous solution first, using substitution $y=e^{rt}$:\\
$y\sp{\prime\prime}+2y\sp\prime-3y=0 $\\
$r\sp{2} + 2r -3=0$\\
$(r-1)(r+3)=0$\\
$r = 1, -3$\\
$y_h(x) = C_1e^{x} + C_2e^{-3x}$\\\\
For the particular solution, make a guess for $y_p(x)$ and plug it into the DE. The guess must be multiplied by a factor of x, to avoid resonance with the homog. solutions:\\
$y_p(x) = x(Ae^x)$\\
$y_p\,\sp{\prime}(x) = Ae^x + x(Ae^x)$\\
$y_p\,\sp{\prime\prime}(x) = 2Ae^x + x(Ae^x)$\\\\
$2Ae^x + x(Ae^x) + 2(Ae^x + x(Ae^x)) - 3(x(Ae^x))= 4e^x$\\
$4Ae^x + 3x(Ae^x) - 3x(Ae^x)= 4e^x$\\
$4Ae^x = 4e^x$\\
$A=1$, so $y_p(x) = xe^x$\\\\
General solution:\\
$y(x) = C_1e^{x} + C_2e^{-3x} + xe^x$\\\\
Plug in the first BC, $y(0)=0$:\\
$y(0) = C_1e^{0} + C_2e^{0} + 0 = 0$\\
$C_1 + C_2= 0$, so $C_1 = -C_2$\\\\
Find $y\sp\prime(x)$ then plug in the BC $y\sp\prime(1)=0$:\\
$y\sp\prime(x) = C_1e^{x} - 3C_2e^{-3x} + e^x + xe^x$\\
$y\sp\prime(1) = C_1e^{1} - 3C_2e^{-3} + e^1 + e^1 = 0$\\
$C_1e - 3C_2e^{-3} + 2e = 0$\\
$(-C_2) - 3C_2e^{-4} = -2$\\
$C_2(-1 - 3e^{-4}) = -2$\\
$C_2 = {2 \over (1 + 3e^{-4})}$\\\\
Multiply top and bottom by $e^4$:\\
$C_2 = {2e^4 \over e^4 + 3}$, so $C_1 = -{2e^4 \over e^4 + 3}$\\\\
\textcolor{teal}{\textbf{This is a \emph{unique} solution:}}\\
$\textcolor{teal}{\bm{y(x) = -{2e^4 \over e^4 + 3}e^{x} + {2e^4 \over e^4 + 3}e^{-3x} + xe^x}}$\\
\newpage
\textbf{(b)} Need to find the general solution. Find homogeneous solution first:\\\\
$y\sp{\prime\prime}+4y=6\cos4x$\\
$r^2+4=0$\\
$r=\pm 2i$\\
$y_h(x) = C_1\cos(2x) + C_2\sin(2x)$\\\\
Find particular solution by making a guess for $y_p(x)$:\\
$y_p(x) = B\cos(4x)$\\
$y_p\,\sp{\prime}(x) = -4B\sin(4x)$\\
$y_p\,\sp{\prime\prime}(x) = -16B\cos(4x)$\\\\
Plug into the DE:\\
$-16B\cos(4x) + 4B\cos(4x)= 6cos(4x)$\\
$-12B = 6$\\
$B = -{1\over2}$\\
$y_p(x) = -{1\over2}\cos(4x)$\\\\
General solution:\\
$y(x) = C_1\cos(2x) + C_2\sin(2x) -{1\over2}\cos(4x)$\\
$y\,\sp\prime(x) = -2C_1\sin(2x) + 2C_2\cos(2x) + 2\sin(4x)$\\\\
Plug in the BC $y\,\sp\prime(0)=0$:\\
$y\,\sp\prime(0) = -2C_1\sin(0) + 2C_2\cos(0) + 2\sin(0) = 0$\\
$C_2 = 0$\\\\
Plug in the other BC $y\,\sp\prime(\pi)=0$:\\
$y\,\sp\prime(\pi) = -2C_1\sin(2\pi) + 2\sin(4\pi) = 0$\\
$-2C_1\sin(2\pi) + 2\sin(4\pi) = 0$\\
$C_1(0)= 0$\\
Any value of $C_1$ works, so:\\\\
$\textcolor{teal}{\bm{y(x) = C_1\cos(2x) - {1\over2}\cos(4x)}}$\\
\textcolor{teal}{\textbf{Where $\bm{C_1}$ is arbitrary, so that there are an infinite number of solutions (non-unique)}}\\

\newpage
\item (15 points)  Consider the eigenvalue problem
\[
y\sp{\prime\prime}+\lambda y=0,\qquad y(0)=0,\quad y\sp\prime(1)=0
\]
\begin{enumerate}
\item
Find all eigenvalues $\lambda$ and corresponding eigenfunctions $y(x)$ for the case $\lambda>0$.  (Note that the boundary condition at $x=1$ involves the derivative of $y(x)$.)\\\\
Find general solution:\\
$r^2 + \lambda = 0$\\
$r = \pm\sqrt{-\lambda}$\\
$r = \pm i\sqrt{\lambda}$\\
$y(x) = C_1\cos(x\sqrt{\lambda}) + C_2\sin(x\sqrt{\lambda})$\\
$y\sp\prime(x) =(0) + \sqrt{\lambda}C_2\cos(x\sqrt{\lambda})$\\\\
Apply BC that $y(0)=0$:\\
$y(0) = C_1\cos(0) + C_2\sin(0) = 0$\\
$C_1 = 0$\\\\
Apply BC that $y\sp\prime(1) = 0$:\\
$y\sp\prime(1) = \sqrt{\lambda}C_2\cos(\sqrt{\lambda}) = 0$\\
$C_2\cos(\sqrt{\lambda}) = 0$\\\\
Either $C_2 = 0$ or $\cos(\sqrt{\lambda})=0$, but $C_2=0$ would just lead to the trivial solution, so:\\
$\cos(\sqrt{\lambda}) = 0$\\
$\sqrt{\lambda} = ({2n-1\over2})\pi$ where (n=1,2,3...)\\\\
\textcolor{teal}{\textbf{Eigenvalues:\\
$\bm{\lambda = ({2n-1\over2})^2\pi^2}$ where ($\bm{n=1,2,3...}$), $\bm{n\neq 0}$\\\\
Eigenfunction:\\
$\bm{y(x) = C_2\sin(({2n-1\over2})\pi x)}$ where ($\bm{n=1,2,3...}$), $\bm{n\neq 0}$}}\\

\item
Determine whether $\lambda=0$ is an eigenvalue.\\\\
$y\sp{\prime\prime}+\lambda y=0$\\
$y\sp{\prime\prime}=0$\\
$y\sp{\prime}=A$\\
$y = Ax + B$\\\\
Apply BC $y(0)=0$:\\
$y(0) = 0 + B = 0$\\
$B = 0$\\
Apply BC $y\sp\prime(1)=0$:\\
$y\sp{\prime}(1) = A = 0$\\
$A = 0$\\\\
Since $A=0$ and $B=0$, $y(x)=0$ is the only solution for $\lambda=0$. Thus:\\
\textcolor{teal}{\textbf{$\bm{\lambda = 0}$ is NOT an eigenvalue, and $\bm{y(x)=0}$ is NOT an eigenfunction.}}

\end{enumerate}


\newpage
\item (20 points) The temperature $u(x,t)$ in a metal bar solves the heat equation
\[
u_t=3u_{xx},\qquad 0<x<1,\quad t>0
\]
subject to the boundary conditions
\[
u(0,t)=0,\qquad u_x(1,t)=0,\qquad t>0
\]
and the initial condition
\[
u(x,0)=2\sin\left({\pi x\over2}\right),\qquad 0<x<1
\]
Follow the steps below to find the solution of the heat flow problem using separation of variables.
\begin{enumerate}
\item Let $u(x,t)=F(x)G(t)$.  Separate the variables in the PDE to verify that the separated equations are $F\sp{\prime\prime}+\lambda F=0$ and $G\sp\prime+3\lambda G=0$, where $\lambda$ is a constant.\\\\
$u_t=3u_{xx}$\\\\
$u_t = {du\over dt} = FG\,\sp\prime$ and $u_{xx} = {d^2u\over dx^2} = GF\,\sp{\prime\prime}$, so:\\
$FG\,\sp\prime = 3GF\,\sp{\prime\prime}$\\
${G\,\sp\prime\over 3G} = {F\,\sp{\prime\prime} \over F}$\\
The ratio of these terms is constant. Set them equal to $-\lambda$:\\
${G\,\sp\prime\over 3G} = {F\,\sp{\prime\prime} \over F} = -\lambda$\\\\
\textcolor{teal}{\textbf{$\bm{F\,\sp{\prime\prime} + \lambda F = 0}$ and $\bm{G\,\sp\prime + 3\lambda G = 0}$}}\\


\item Determine boundary conditions for $F(x)$ and solve the resulting eigenvalue problem.  (Hint: recall your work on a previous problem.)\\\\
For $u(0,t)=0$:\\
$F(0)G(0)=0$\\
So \textcolor{teal}{$\bm{F(0)=0}$}\\\\
For $u_x(1,t)=0$:\\
${d\over dx}[F(1)G(0)]=0$\\
$G(0){d\over dx}[F(1)]=0$\\
And \textcolor{teal}{$\bm{F\sp\prime(1)=0}$}\\

With $F\,\sp{\prime\prime} + \lambda F=0$ and these conditions (identical to problem 2), the solution is:\\
$F(x)=C_2\sin(({2n-1\over2})\pi x)$ where ($n=1,2,3...$), $n\neq 0$\\
Pick $C_2 = 1$:\\\\
\textcolor{teal}{$\bm{F(x)=\sin(({2n-1\over2})\pi x)}$ \textbf{where ($\bm{n=1,2,3...}$),} $\bm{n\neq 0}$\\
$\bm{\lambda = ({2n-1\over2})^2\pi^2}$ \textbf{where ($\bm{n=1,2,3...}$),} $\bm{n\neq 0}$}\\

\item Solve the separated equation for $G(t)$.  Sum over all available solutions for $F(x)G(t)$ to determine the general solution for $u(x,t)$ satisfying the PDE and the BCs.\\\\
$G\,\sp\prime + 3\lambda G = 0$\\
$G\,\sp\prime = -3\lambda G$\\
Let $G=e^{rt}$ and solve:\\
$r = -3\lambda$\\\\
$G(t)= Ae\sp{-3\lambda t} = Ae\sp{-3({2n-1\over2})^2\pi^2 t}$\\
$u(x,t) = G(t)F(x) = Ae\sp{-3({2n-1\over2})^2\pi^2 t} * \sin(({2n-1\over2})\pi x)$\\\\
Sum over all values of n:
\[
\textcolor{teal}{\bm{u(x,t) = \sum_{n=1}^{\infty} \;{ A_ne\sp{-3({2n-1\over2})^2\pi^2 t} * \sin\left(({2n-1\over2})\pi x\right)}\;}}
\]

\item Apply the initial condition to determine the solution of the heat flow problem.\\\\
Using the IC $u(x,0)=2\sin\left({\pi x\over2}\right)$. With $t=0$, the $e\,^{\text{stuff}}$ term becomes $e^0 = 1$:\\\\
$u(x,0) = \sum_{n=1}^{\infty} \;[ A_n \sin\left(({2n-1\over2})\pi x\right)] = 2\sin\left({\pi x\over2}\right)$\\\\
Expand the Fourier Sine Series:\\
$A_1 \sin\left(({2n-1\over2})\pi x\right) + A_2 \sin\left(({2n-1\over2})\pi x\right) + A_3 \sin\left(({2n-1\over2})\pi x\right) +...= 2\sin\left({\pi x\over2}\right)$\\\\
Only care about the term with $n=1$. Set $A_1=2$, and $A_n = 0$ when $n\geq2$\\\\
The solution of the heat flow problem is:
\[
\textcolor{teal}{\bm{u(x,t) = 2e\sp{-3\left({\pi\over2}\right)^2 t} * \sin\left({\pi x\over 2}\right)}}
\]


\end{enumerate}


\end{enumerate}


\end{document}

