\documentclass{article}

\usepackage{amsmath}
\usepackage{amssymb}
\usepackage{bm}
\usepackage{graphicx}
\usepackage{epstopdf}
\DeclareGraphicsRule{.tif}{png}{.png}{`convert #1 `basename #1 .tif`.png}
\usepackage{color}
\usepackage{pdfsync}
\pagestyle{plain}

\textheight 9 true in
\textwidth 6.5 true in
\hoffset -.75 true in
\voffset -.75 true in
\mathsurround=2pt
\parskip=2pt

\begin{document}

\begin{center}
\large{ MATH-2400 \hspace{.27in}  INTRODUCTION TO DIFFERENTIAL EQUATIONS \hspace{.27in}FALL 2022\bigskip\\ {\bf Problem Set 4} \smallskip\\ Due: 5pm, Friday, September 30, 2022}
\\Hayden Fuller
\end{center}

\bigskip\noindent
\underline{NOTES}
\begin{enumerate}
\item Practice problems listed below and taken from the textbook are for your own practice, and are not to be turned in.
\item There are two parts of the Problem Set, an objective part consisting of multiple choice questions (with no partial credit available) and a subjective part (with partial credit possible).  Please complete all questions.
\item Writing your solutions in {\LaTeX} is preferred but not required.
\item Show all work for problems in the subjective part.  Illegible or undecipherable solutions will not be graded. 
\item Figures, if any, should be neatly drawn by hand, properly labelled and captioned.  
\item Your completed work is to be submitted electronically to LMS  as a \textcolor{red}{single pdf file}. Be sure that the pages are properly oriented and well lighted.  (\textcolor{blue}{Please do not e-mail your work to Muhammad or me.})
\end{enumerate}

\bigskip\noindent
{\bf Practice Problems from the textbook} (Not to be turned in)
\begin{itemize}
\item
Exercises from Chapter 3, page 50--51: 3(j), 4(h,i,j), 5(a,d,g,f), 6(c).
\item
Exercises from Chapter 3, pages 77--78: 1(a,b,c,d), 2(a,b).
\end{itemize}

\bigskip\noindent
{\bf Objective part} (Choose A, B, C or D; no work need be shown, no partial credit available)

\begin{enumerate}

\item (5 points) Select the linear, {\em homogeneous} DE for which $y(t)=e\sp{-3t}$ is a solution
\begin{description}
\item[A] $y\sp{\prime\prime}+2y\sp\prime=3e\sp{-3t}$
\item[B] $y\sp{\prime\prime}+9y=0$
\item[C] X$ty\sp{\prime\prime}-y\sp\prime-3(1+3t)y=0$X
\item[D] None of these choices.
\end{description}

\item (5 points) Assume $y(t)$ solves the ODE $y\sp{\prime\prime}+by\sp\prime+cy=0$ and the initial conditions $y(0)=0$, $y\sp\prime(0)=1$.  For what values of $b$ and $c$ does the solution decay to zero as $t\rightarrow\infty$:
\begin{description}
\item[A] X$b=4$ and $c=4$X
\item[B] $b=-2$ and $c=6$
\item[C] Both choices A and B
\item[D] Neither choice A or B
\end{description}

\item (5 points) Select the Cauchy-Euler equation for which $y(x)=x\sp2\cos(\ln x)$, $x>0$, is a solution
\begin{description}
\item[A] $x\sp2y\sp{\prime\prime}-5xy\sp\prime+5y=0$
\item[B] X$x\sp2y\sp{\prime\prime}-3xy\sp\prime+5y=0$X
\item[C] $x\sp2y\sp{\prime\prime}-3xy\sp\prime+y=0$
\item[D] None of these choices
\end{description}

\end{enumerate}

\newpage\noindent
{\bf Subjective part} (Show work, partial credit available)

\begin{enumerate}

\bigskip
\item (15 points) Consider the linear, homogeneous, second-order ODE
\[
y\sp{\prime\prime}+{3\over2t}\,y\sp\prime-{3\over t\sp2}\,y=0,\qquad t>0
\]
\begin{enumerate}
\item
Verify that $y_1(t)=t\sp{-2}$ is a solution of the ODE, and find a second solution $y_2(t)$ using the method of reduction of order.
\\$y=t^{-2}$\qquad$y'=-2t^{-3}$\qquad$y''=6t^{-4}$
\\$6t^{-4}+\frac{3}{2t}(-2)t^{-3}-\frac{3}{t^2}t^{-2}=0$
\\$6t^{-4}-3t^{-4}-3t^{-4}=0$
\\$0=0$
\\$y_2(t)=y_1(t)h(t)=t^{-2}h(t)$
\\$y'_2(t)=-2t^{-3}h(t)+t^{-2}h'(t)$
\\$y''_2(t)=(6t^{-4}h(t)-2t^{-3}h'(t))+(-2t^{-3}h'(t)+t^{-2}h''(t))=6t^{-4}h(t)-4t^{-3}h'(t)+t^{-2}h''(t)$
\\$y''_2+\frac{3}{2t}y'_2-\frac{3}{t^2}y=0$
\\$(6t^{-4}h(t)-4t^{-3}h'(t)+t^{-2}h''(t))+\frac{3}{2t}(-2t^{-3}h(t)+t^{-2}h'(t))-\frac{3}{t^2}t^{-2}h(t)=0$
\\$6t^{-4}h(t)-4t^{-3}h'(t)+t^{-2}h''(t)+-3t^{-4}h(t)+\frac{3}{2}t^{-3}h'(t)-3t^{-4}h(t)=0$
\\$-\frac{5}{2}t^{-3}h'(t)+t^{-2}h''(t)=0$
\\$u=h'$, $u'=h''$
\\$-\frac{5}{2}t^{-3}u(t)+t^{-2}u'(t)=0$
\\$t^{-2}u'=\frac{5}{2}t^{-3}u$
\\$\frac{1}{u}u'=\frac{5}{2}t^{-1}$
\\$\int\frac{1}{u}du=\int\frac{5}{2}t^{-1}dt$
\\$\ln|u|=\frac{5}{2}\ln|t|+C$
\\$u=Ce^{\frac{5}{2}\ln|t|}=Ct^{\frac{5}{2}}$
\\$h(t)=\int u dt=\int Ct^{\frac{5}{2}}dt$
\\$h(t)=\frac{2}{7}Ct^{\frac{7}{2}}+D$
\\$C=\frac{7}{2}$ and $D=0$
\\$h(t)=t^{\frac{7}{2}}$
\\$y_2(t)=y_1(t)h(t)=(t^{-2})(t^{\frac{7}{2}})$
\\$y_2(t)=t^{\frac{3}{2}}$
\item
Compute the Wronskian of $y_1(t)$ and $y_2(t)$ to show that the solutions are independent (and thus form a fundamental set of solutions).
\\$y_1(t)=t^{-2}$\qquad$y_2(t)=t^{\frac{3}{2}}$
\\$y'_1(t)=-2t^{-3}$\qquad$y'_2(t)=\frac{3}{2}t^{\frac{1}{2}}$
\\$W(t)=\det\begin{bmatrix}t^{-2} & t^{\frac{3}{2}} \\ -2t^{-3} & \frac{3}{2}t^{\frac{1}{2}}\end{bmatrix}=(t^{-2}*\frac{3}{2}t^{\frac{1}{2}})-(t^{\frac{3}{2}}*-2t^{-3})=\frac{3}{2}t^{\frac{-3}{2}}+2t^{\frac{-3}{2}}=\frac{7}{2}t^{\frac{-3}{2}}$
\\$\frac{7}{2}t^{\frac{-3}{2}}\ne0$ for $t>0$, so the solutions are independent.
\end{enumerate}

\item (15 points)  Consider the initial-value problem
\[
y\sp{\prime\prime}+4y\sp\prime+13y=0,\qquad y(0)=-1,\quad y\sp\prime(0)=5
\]
\begin{enumerate}
\item Find real-valued solutions $y_1(t)$ and $y_2(t)$ in the general solution $y(t)=C_1y_1(t)+C_2y_2(t)$ of the constant-coefficient ODE, and then apply the initial conditions to determine the constants in the general solution.
\\$r=\frac{-4\pm\sqrt{16-4*13}}{2}=\frac{-4\pm\sqrt{-36}}{2}=\frac{-4\pm6i}{2}=-2\pm3i$
\\$y_1(t)=e^{(-2+3i)t}$\qquad$y_2(t)=e^{(-2-3i)t}$
\\$y_1(t)=e^{-2t}\cos(3t)$\qquad$y_2(t)=e^{-2t}\sin(3t)$
\\$y'_1(t)=-2e^{-2t}\cos(3t)-3e^{-2t}\sin(3t)$\qquad$y'_2=-2e^{-2t}\sin(3t)+3e^{-2t}\cos(3t)$
\\$y(t)=C_1e^{-2t}\cos(3t)+C_2e^{-2t}sin(3t)$
\\$y'(t)=C_1(-2e^{-2t}\cos(3t)-3e^{-2t}\sin(3t))+C_2(-2e^{-2t}\sin(3t)+3e^{-2t}\cos(3t))$
\\$y(0)=-1=C_1e^0\cos(0)+C_2e^0sin(0)=C_1=-1$
\\$y'(0)=5=-(-2e^0\cos(0)-3e^0\sin(0))+C_2(-2e^0\sin(0)+3e^0\cos(0))$
\\$y'(0)=5=(2)+C_2(3)$\qquad$3C_2=3$\qquad$C_2=1$
\\$C_1=-1$\qquad$C_2=1$
\\$y(t)=-e^{-2t}\cos(3t)+e^{-2t}sin(3t)$
\item Write the solution in part (a) in the \lq\lq polar'' form $y(t)=Re\sp{\lambda t}\cos(\omega t-\phi)$ following an example discussed in class.  Give the constants $R$, $\lambda$, $\omega$ and $\phi$, and use the polar form to sketch the solution.
\\$C_1=-1=R\cos(\phi)$\qquad$C_2=1=R\sin(\phi)$
\\$-\cos(\phi)=\sin(\phi)$\qquad$\phi=-\frac{\pi}{4}+k\pi$, use $k=0$, $\phi=-\frac{\pi}{4}$
\\$1=R\sin(-\frac{\pi}{4})=R(\frac{-\sqrt{2}}{2})$\qquad$R=\frac{2}{-\sqrt{2}}=-\sqrt{2}$
\\$R=-\sqrt{2}$\qquad$\phi=\frac{-\pi}{4}$\qquad$\lambda=-2$\qquad$\omega=3$
\\$y(t)=-\sqrt{2}e^{-2t}\cos(3t+\frac{\pi}{4})$
\\I can't figure out how to add a polar graph, but it can be found at:
\\https://www.desmos.com/calculator/yewobyquvi
\end{enumerate}


\bigskip
\item (15 points)
\begin{enumerate}
\item
Find $y(t)$ satisfying the initial-value problem
\[
9y\sp{\prime\prime}+6y\sp\prime+y=0,\qquad y(0)=1,\quad y\sp\prime(0)={2\over3}
\]
\\$y(t)=e^{rt}$
\\$9r^2+6r+1=0$
\\$(3r+1)^2=0$
\\$r=-\frac{1}{3}=r_1=r_2$
\\$y_1(t)=e^{rt}$\qquad$y_2(t)=te^{rt}$
\\$y_1(t)=e^{-\frac{1}{3}t}$\qquad$y_2(t)=te^{-\frac{1}{3}t}$
\\$y(t)=C_1e^{-\frac{1}{3}t}+C_2te^{-\frac{1}{3}t}$
\\$y'(t)=-\frac{1}{3}C_1e^{-\frac{1}{3}t}+C_2(e^{-\frac{1}{3}t}-\frac{1}{3}te^{-\frac{1}{3}t})$
\\$y(0)=1=C_1e^0+C_20e^0=C_1=1$
\\$y'(0)=\frac{2}{3}=-\frac{1}{3}1e^0+C_2(e^0-0)=-\frac{1}{3}+C_2$
\\$C_1=1$\qquad$C_2=1$
\\$y(t)=e^{-\frac{1}{3}t}+te^{-\frac{1}{3}t}$
\\$y(t)=(1+t)e^{-\frac{1}{3}t}$
\item
Find real-valued functions $u_1(x)$ and $u_2(x)$ in the general solution $u(x)=C_1u_1(x)+C_2u_2(x)$ of the Cauchy-Euler equation
\[
4x\sp2u\sp{\prime\prime}+8xu\sp\prime+u=0,\qquad x>0
\]
Find $C_1$ and $C_2$ in the general solution so that $u(1)=0$ and $u\sp\prime(1)=3$.
\\$u=x^r$
\\$ax^2(r(r-1)x^{r-2})+bx(rx^{x-1}+cx^r=0$
\\$a=4$\qquad$b=8$\qquad$c=1$
\\$x^r(ar(r-1)+br+c)=0$
\\$ar(r-1)+br+c=0$
\\$4r(r-1)+8r+1=0$
\\$4r^2+4r+1=0$
\\$(2r+1)^2=0$
\\$r=r_1=r_2=-\frac{1}{2}$
\\$u_1(x)=x^r$\qquad$u_2(x)=x^r\ln(x)$
\\$u(x)=C_1x^r+C_2x^r\ln(x)$
\\$u(x)=C_1x^{-\frac{1}{2}}+C_2x^{-\frac{1}{2}}\ln(x)$
\\$u'(x)=-\frac{1}{2}C_1x^{-\frac{3}{2}}+C_2(-\frac{1}{2}x^{-\frac{3}{2}}\ln(x)+x^{-\frac{1}{2}}\frac{1}{x})$
\\$u(1)=0=C_1*1+C_2*1*0=C_1=0$
\\$u'(1)=3=-\frac{1}{2}*0*1+C_2(-\frac{1}{2}*1*0+1*1)=-\frac{1}{2}*0+C_2(1)=C_2=3$
\\$C_1=0$\qquad$C_2=3$
\\$u(x)=0*x^{-\frac{1}{2}}+3*x^{-\frac{1}{2}}\ln(x)$
\\$u(x)=3x^{-\frac{1}{2}}\ln(x)$





\end{enumerate}

\end{enumerate}

\end{document}




























