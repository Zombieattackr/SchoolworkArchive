\documentclass{article}

\usepackage{amsmath}
\usepackage{amssymb}
\usepackage{bm}
\usepackage{graphicx}
\usepackage{epstopdf}
\DeclareGraphicsRule{.tif}{png}{.png}{`convert #1 `basename #1 .tif`.png}
\usepackage{color}
\usepackage{pdfsync}
\pagestyle{plain}

\textheight 9 true in
\textwidth 6.5 true in
\hoffset -.75 true in
\voffset -.75 true in
\mathsurround=2pt
\parskip=2pt

\begin{document}

\begin{center}
\large{ MATH-2400 \hspace{.27in}  INTRODUCTION TO DIFFERENTIAL EQUATIONS \hspace{.27in}FALL 2022\bigskip\\ {\bf Problem Set 8} \smallskip\\ Due: 11pm, Tuesday, November 15, 2022}
\\Hayden Fuller
\end{center}

\bigskip\noindent
\underline{NOTES}
\begin{enumerate}
\item Practice problems listed below and taken from the textbook are for your own practice, and are not to be turned in.
\item There are two parts of the Problem Set, an objective part consisting of multiple choice questions (with no partial credit available) and a subjective part (with partial credit possible).  Please complete all questions.
\item Writing your solutions in {\LaTeX} is preferred but not required.
\item Show all work for problems in the subjective part.  Illegible or undecipherable solutions will not be graded. 
\item Figures, if any, should be neatly drawn by hand, properly labelled and captioned.  
\item Your completed work is to be submitted electronically to LMS  as a \textcolor{red}{single pdf file}. Be sure that the pages are properly oriented and well lighted.  (\textcolor{blue}{Please do not e-mail your work to Muhammad or me.})
\end{enumerate}

\bigskip\noindent
{\bf Practice Problems from the textbook} (Not to be turned in)
\begin{itemize}
\item
Exercises from Chapter 7, pages 179--180: 1(a,c,e), 2(a,c), 3(b,d,e).
\item
Exercises from Chapter 7, pages 186--187: 1(b,d), 2(b,d), 5(b,d), 7(b,d)
\end{itemize}

\bigskip\noindent
{\bf Objective part} (Choose A, B, C or D; no work need be shown, no partial credit available)

\begin{enumerate}
\item (5 points)  Identify the correct statement, or select \lq\lq None of these choices'' if none of the statements are correct.
\begin{description}
\item[A] XThe only solution of the DE $y\sp{\prime\prime}-y=0$ with BCs $y(0)=0$ and $y(\pi)=0$ is the trivial solution.X
\item[B] The only solution of the DE $y\sp{\prime\prime}+y=0$ with BCs $y(0)=0$ and $y(\pi)=0$ is the trivial solution.
\item[C] The DE $x\sp2y\sp{\prime\prime}+xy\sp\prime+y=0$ with BCs $y(1)=0$ and $y(2)=0$ has nontrivial solutions.
\item[D] None of these choices
\end{description}

\item (5 points) Identify the correct statement, or select \lq\lq All of these choices'' if all of the statements are correct.  For each statement, $\lambda$ is a constant.
\begin{description}
\item[A] Setting $u(x,y)=F(x)G(y)$ in the PDE $u_{xx}+u_{yy}=0$ leads to the separated equations $F\sp{\prime\prime}-\lambda F=0$ and $G\sp{\prime\prime}+\lambda G=0$.
\item[B] Setting $v(x,t)=F(x)G(t)$ in the PDE $v_{tt}=v_{xx}$ leads to the separated equations $F\sp{\prime\prime}+\lambda F=0$ and $G\sp{\prime\prime}+\lambda G=0$.
\item[C] Setting $w(r,t)=F(r)G(t)$ in the PDE $w_{t}=w_{rr}+{1\over r}w_r$ leads to the separated equations $rF\sp{\prime\prime}+F\sp{\prime}+\lambda rF=0$ and $G\sp{\prime}+\lambda G=0$.
\item[D] XAll of these choicesX
\end{description}

\end{enumerate}

\newpage\noindent
{\bf Subjective part} (Show work, partial credit available)

\begin{enumerate}

\item (15 points)  For each boundary-value problem, determine whether or not a solution exists.  If a solution exists, then determine whether or not it is unique.
\[
\begin{array}{lc}
\hbox{(a)}\qquad & y\sp{\prime\prime}+2y\sp\prime-3y=4e\sp{x},\qquad y(0)=0,\quad y\sp\prime(1)=0 \medskip\\
\hbox{(b)}\qquad & y\sp{\prime\prime}+4y=6\cos4x,\qquad y\sp\prime(0)=0,\quad y\sp\prime(\pi)=0
\end{array}
\]
\\a)
\\$r^2+2r-3=0=(r+3)(r-1)$
\\$r=1,-3$
\\$y_h=C_1e^x+C_2e^{-3x}$
\\guess $y_p=Ae^x$, resonance
\\$y_p=Axe^x$, no resonance
\\$y'_p=A(e^x+xe^x)$
\\$y''_p=A(e^x+e^x+xe^x)$
\\$A(e^x+e^x+xe^x)+2A(e^x+xe^x)-3Axe^x=4e^x$
\\$A(2e^x+xe^x)+A(2e^x+2xe^x)-3Axe^x=4e^x$
\\$A(4e^x+3xe^x)-3Axe^x=4e^x$
\\$A4e^x=4e^x$
\\$A=1$
\\$y_p=xe^x$
\\$y(x)=C_1e^x+C_2e^{-3x}+xe^x$
\\$y(0)=0=C_1+C_2$
\\$C_1=-C_2$
\\$y'(x)=C_1e^x-3C_2e^{-3x}+e^x+xe^x$
\\$y'(1)=0=C_1e^1-3C_2e^{-3}+e^1+1e^1$
\\$0=C_1e-3C_2e^{-3}+2e$
\\$0=C_1e+3C_1e^{-3}+2e$
\\$0=C_1+3C_1e^{-4}+2$
\\$-2=C_1(1+3e^{-4})$
\\$C_1=\frac{-2}{1+3e^{-4}}$
\\$C_2=\frac{2}{1+3e^{-4}}$
\\$y(x)=\frac{-2}{1+3e^{-4}}e^x+\frac{2}{1+3e^{-4}}e^{-3x}+xe^x$
\\Yes, there is a solution, and it is unique.
\\
\\b)
\\$r^2+4=0$
\\$r=\pm2i$
\\$y_h=C_1\cos(2x)+C_2\sin(2x)$
\\guess $y_p=A\sin(4x)+B\cos(4x)$, no resonance
\\$y'_p=A4\cos(4x)-B4\sin(4x)$
\\$y''_p=-A16\sin(4x)-B16\cos(4x)$
\\let $s=\sin(4x)$ and $c=\cos(4x)$
\\$-A16s-B16c+4(As+Bc)=6c$
\\$-A16s-B16c+A4s+B4c=6c$
\\$s(-A16+A4)+c(-B16+B4)=6c$
\\$-A16+4A=0$; $A=0$
\\$-B16+B4=6$
\\$-B12=6$; $B=-\frac{1}{2}$
\\$y_p=0\sin(4x)+\frac{-1}{2}\cos(4x)=\frac{-1}{2}\cos(4x)$
\\$y(x)=C_1\cos(2x)+C_2\sin(2x)+\frac{-1}{2}\cos(4x)$
\\$y'(x)=-C_12\sin(2x)+C_22\cos(2x)+2\sin(4x)$
\\$y'(0)=0=-C_12\sin(0)+C_22\cos(0)+2\sin(0)$
\\$0=C_22$
\\$C_2=0$
\\$y'(\pi)=0=-C_12\sin(2\pi)+C_22\cos(2\pi)+2\sin(4\pi)$
\\$0=C_22$
\\$C_2=0$
\\Yes, a solution exists, but there is not a unique solution since any $C_1$ will satisfy the IC's.

\bigskip
\item (15 points)  Consider the eigenvalue problem
\[
y\sp{\prime\prime}+\lambda y=0,\qquad y(0)=0,\quad y\sp\prime(1)=0
\]
\begin{enumerate}
\item
Find all eigenvalues $\lambda$ and corresponding eigenfunctions $y(x)$ for the case $\lambda>0$.  (Note that the boundary condition at $x=1$ involves the derivative of $y(x)$.)
\item
Determine whether $\lambda=0$ is an eigenvalue.
\end{enumerate}
a)
\\$r^2+\lambda=0$
\\$r^2=-\lambda$
\\$r=i\sqrt{\lambda}$
\\$y(x)=y_h=C_1\cos(\sqrt{\lambda}x)+C_2\sin(\sqrt{\lambda}x)$
\\$y(0)=0=C_1\cos(0)+C_2\sin(0)$
\\$0=C_1$
\\$y(x)=C_2\sin(\sqrt{\lambda}x)$
\\$y'(x)=\sqrt{\lambda}C_2\cos(\sqrt{\lambda}x)$
\\$y'(1)=0=\sqrt{\lambda}C_2\cos(\sqrt{\lambda})$
\\$0=\sqrt{\lambda}C_2\cos(\sqrt{\lambda})$
\\$0=C_2\cos(\sqrt{\lambda})$
\\$0=\cos(\sqrt{\lambda})$
\\let $n\in\mathbb{Z}$ (or is it $n\in\mathbb{N}$? Sorry, I've been out sick, I've seen both)
\\$\sqrt{\lambda}=(2n-1)\frac{\pi}{2}$
\\$\lambda=((2n-1)\frac{\pi}{2})^2$
\\$y(x)=C_2\sin(\sqrt{((2n-1)\frac{\pi}{2})^2}x)$
\\$y(x)=C_2\sin((2n-1)\frac{\pi}{2}x)$
\\
\\b)
\\$y''+0=0$
\\$y''=0$
\\$y'=A$
\\$y=Ax+B$
\\$y(0)=0=Ax+B=B$
\\$y'(1)=0=A$
\\$y(x)=0$
\\trivial solution, so $\lambda=0$ is not an eigenvalue and $y(x)=0$ is not an eigenfunction.


\bigskip
\item (20 points) The temperature $u(x,t)$ in a metal bar solves the heat equation
\[
u_t=3u_{xx},\qquad 0<x<1,\quad t>0
\]
subject to the boundary conditions
\[
u(0,t)=0,\qquad u_x(1,t)=0,\qquad t>0
\]
and the initial condition
\[
u(x,0)=2\sin\left({\pi x\over2}\right),\qquad 0<x<1
\]
Follow the steps below to find the solution of the heat flow problem using separation of variables.
\begin{enumerate}
\item Let $u(x,t)=F(x)G(t)$.  Separate the variables in the PDE to verify that the separated equations are $F\sp{\prime\prime}+\lambda F=0$ and $G\sp\prime+3\lambda G=0$, where $\lambda$ is a constant.
\item Determine boundary conditions for $F(x)$ and solve the resulting eigenvalue problem.  (Hint: recall your work on a previous problem.)
\item Solve the separated equation for $G(t)$.  Sum over all available solutions for $F(x)G(t)$ to determine the general solution for $u(x,t)$ satisfying the PDE and the BCs.
\item Apply the initial condition to determine the solution of the heat flow problem.
\end{enumerate}
a)
\\$u=F(x)G(t)$
\\$u_x=F'(x)G(t)$
\\$u_{xx}=F''(x)G(t)$
\\$u_t=F(x)G'(t)$
\\$u_{tt}=F(x)G''(t)$
\\$F(x)G'(t)=3F''(x)G(t)$
\\$\frac{G'(t)}{3G(t)}=\frac{F''(x)}{F(x)}=-\lambda$
\\$\frac{G'(t)}{3G(t)}=-\lambda$
\\$G'(t)=-\lambda3G(t)$
\\$G'(t)+\lambda3G(t)=0$
\\$\frac{F''(x)}{F(x)}=-\lambda$
\\$F''(x)=-\lambda F(x)$
\\$F''(x)+\lambda F(x)=0$
\\
\\b)
\\$u(0,t)=0=F(0)G(t)$
\\$F(0)=0$
\\$u_x(1,t)=0=F'(1)G(t)$
\\$F'(1)=0$
\\Identical to problem 2, $y(x)=F(x)$
\\$r^2+\lambda=0$
\\$r^2=-\lambda$
\\$r=i\sqrt{\lambda}$
\\$F(x)=F_h=C_1\cos(\sqrt{\lambda}x)+C_2\sin(\sqrt{\lambda}x)$
\\$F(0)=0=C_1\cos(0)+C_2\sin(0)$
\\$0=C_1$
\\$F(x)=C_2\sin(\sqrt{\lambda}x)$
\\$F'(x)=\sqrt{\lambda}C_2\cos(\sqrt{\lambda}x)$
\\$F'(1)=0=\sqrt{\lambda}C_2\cos(\sqrt{\lambda})$
\\$0=\sqrt{\lambda}C_2\cos(\sqrt{\lambda})$
\\$0=C_2\cos(\sqrt{\lambda})$
\\$0=\cos(\sqrt{\lambda})$
\\let $n\in\mathbb{Z}$
\\$\sqrt{\lambda}=(2n-1)\frac{\pi}{2}$
\\$\lambda=((2n-1)\frac{\pi}{2})^2$
\\$F(x)=C_2\sin(\sqrt{((2n-1)\frac{\pi}{2})^2}x)$
\\$F(x)=C_2\sin((2n-1)\frac{\pi}{2}x)$
\\$F(x)=A\sin((2n-1)\frac{\pi}{2}x)$
\\
\\c)
\\$G'(t)+3\lambda G(t)=0$
\\$r+3\lambda=0$
\\$r=-3\lambda$
\\$G(t)=e^{-3\lambda t}$
\\$G(t)=e^{-3((2n-1)\frac{\pi}{2})^2 t}$
\\$u(x,t)=\sum_{n=1}^{\infty}A_ne^{-3((2n-1)\frac{\pi}{2})^2 t}\sin((2n-1)\frac{\pi}{2}x)$
\\
\\d)
\\$u(x,0)=2\sin(\frac{\pi x}{2})$
\\$u(x,t)=\sum_{n=1}^{\infty}A_ne^{-3((2n-1)\frac{\pi}{2})^2 t}\sin((2n-1)\frac{\pi}{2}x)$
\\$u(x,0)=\sum_{n=1}^{\infty}A_ne^{0}\sin((2n-1)\frac{\pi}{2}x)=2\sin(\frac{\pi x}{2})$
\\$\sum_{n=1}^{\infty}A_n\sin((2n-1)\frac{\pi}{2}x)=2\sin(\frac{\pi x}{2})$
\\$A_1=2$; $A_n=0$ for $n\ge2$
\\$u(x,t)=\sum_{n=1}^{\infty}A_ne^{-3((2n-1)\frac{\pi}{2})^2 t}\sin((2n-1)\frac{\pi}{2}x)$
\\$u(x,t)=A_1e^{-3((2(1)-1)\frac{\pi}{2})^2 t}\sin((2(1)-1)\frac{\pi}{2}x)$
\\$u(x,t)=2e^{-3(\frac{\pi}{2})^2 t}\sin(\frac{\pi}{2}x)$
\\$u(x,t)=2e^{-\frac{3}{4}\pi^2 t}\sin(\frac{\pi x}{2})$

\end{enumerate}


\end{document}

















































