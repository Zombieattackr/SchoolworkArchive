\documentclass[11pt]{article}
\usepackage{datetime}
\usepackage{color,array,graphics}
\usepackage{enumerate}
\usepackage[pdftex, colorlinks, linkcolor=red,citecolor=red,urlcolor=blue]{hyperref}
\usepackage{ulem}

\setlength{\parindent}{0cm}

\setlength{\parskip}{0.3cm plus4mm minus3mm}

\textwidth  6.5in
\oddsidemargin +0.0in
\evensidemargin +0.0in
\textheight 9.0in
\topmargin -0.5in

\usepackage{upquote,textcomp}
\usepackage{amssymb,amsmath,amsfonts,amsthm}
\usepackage{graphicx}
\usepackage{multicol}
\usepackage[T1]{fontenc}

\def\OR{\vee}
\def\AND{\wedge}
\def\imp{\rightarrow}

\DeclareSymbolFont{AMSb}{U}{msb}{m}{n}
\DeclareMathSymbol{\N}{\mathbin}{AMSb}{"4E}
\DeclareMathSymbol{\Z}{\mathbin}{AMSb}{"5A}
\DeclareMathSymbol{\R}{\mathbin}{AMSb}{"52}
\DeclareMathSymbol{\Q}{\mathbin}{AMSb}{"51}
\DeclareMathSymbol{\I}{\mathbin}{AMSb}{"49}
\DeclareMathSymbol{\C}{\mathbin}{AMSb}{"43}

\begin{document}
\thispagestyle{empty}   %% skips page number on the first page

\begin{center}
\large
\textbf{CSCI 2200 --- Foundations of Computer Science (FoCS) \\
Homework 3 (document version 1.2)}
\end{center}

\section*{Overview}
\begin{itemize}
\item This homework is due by 11:59PM on Thursday, October~20
\item You may work on this homework in a group of no more than four students;
  unlike recitation problem sets,
  \textbf{your teammates may be in any section}
\item You may use at most \textbf{three} late days on this assignment
\item Please start this homework early and ask questions during
  office hours; % and at your September~14 recitation section;
  also ask (and answer) questions on the Discussion Forum 
\item Please be concise in your answers;
  even if your solution is correct, if it is not well-presented,
  you may still lose points
\item You can type or hand-write (or both) your solutions
  to the required graded problems below;
  \textbf{all work must be organized in one PDF that lists
  all teammate names}
\item You are strongly encouraged to use LaTeX, in particular for
  mathematical symbols;
  see references in Course Materials
\item \textbf{EARNING LATE DAYS:}
  for each homework that you complete using LaTeX
  (including any tables, graphs, etc., i.e.,~no hand-written anything),
  you earn one additional late day;
  you can draw graphs and other diagrams
  in another application and include them as image files
\item To earn a late day, you must submit your LaTeX files
  (i.e.,~\verb+*.tex+) along with your one required PDF file---please name
  the PDF file \verb+hw3.pdf+
\item Also note that the earned late day can be used
  retroactively, even back to the first homework assignment!
\end{itemize}

\vspace{0.2in}

%\begin{center}
%\includegraphics[scale=0.5]{math-bitmoji.png}
%\end{center}

\newpage
\section*{Warm-up exercises}
The problems below are good practice problems to work on.
Do not submit these as part of your homework submission.
\textbf{These are ungraded problems.}

\begin{multicols}{2}
\begin{itemize}

% 10+
\item \textbf{Problem 7.11.}
\item \textbf{Problem 7.12(a-b).}\\(See Problem 7.28 for hints.)
\item \textbf{Problem 7.21.}
\item \textbf{Problem 7.41.}
\item \textbf{Problem 7.44.}
\item \textbf{Problem 7.45(a-b,d-f).}
\item \textbf{Problem 7.46.}
\item \textbf{Problem 7.47.}
\item \textbf{Problem 8.12(a-c).}
\item \textbf{Problem 8.13.}
\item \textbf{Problem 8.18.}

\end{itemize}
\end{multicols}

\section*{Graded problems}
The problems below are required and will be graded.
\begin{itemize}

% 8+
\item \textbf{*Problem 7.9.}
\item \textbf{*Problem 7.12(c).} (See Problem 7.28 for hints.)
\item \textbf{*Problem 7.13(a).}
\item \textbf{*Problem 7.19(d).}
%\item \textbf{*Problem 7.19(k).}
\item \textbf{*Problem 7.42.}
\item \textbf{*Problem 7.45(c).}
\item \textbf{*Problem 7.49.}
\item \textbf{*Problem 8.12(d).}
\item \textbf{*Problem 8.14.}

\end{itemize}

\textbf{(v1.1)}~Some of the above problems (graded an ungraded)
are transcribed in the pages that follow.

Graded problems are noted with an asterisk~(*).

If any typos exist below, please use the textbook description.

\newpage
\begin{itemize}

\item \textbf{*Problem 7.9.}
$G_0=0$, $G_1=1$, and $G_n=7G_{n-1}-12G_{n-2}$ for $n>1$.
Compute $G_5$.
\\$G_5=7G_4-12G_3$
\\$G_4=7G_3-12G_2$
\\$G_3=7G_2-12G_1$
\\$G_2=7G_1-12G_0$
\\$G_2=7=7$
\\$G_3=7*7-12=37$
\\$G_4=7*(7*7-12)-12*7=175$
\\$G_5=7*(7*(7*7-12)-12*7)-12(7*7-12)=781$
\\
Show $G_n=4^n-3^n$ for $n\ge 0$.
\\We prove the statement with induction.
\\Base case: $G_0=4^0-3^0=0$
\\Base case: $G_1=4^1-3^1=1$
\\Induction: assume $G_n=4^n-3^n$
\\$7G_{n-1}-12G_{n-2}=4^n-3^n$
\\$7G_{n}-12G_{n-1}=4^{n+1}-3^{n+1}$
\\$7(4^n-3^n)-12(4^{n-1}-3^{n-1})=4^{n+1}-3^{n+1}$
\\$(7*4^n-7*3^n)-(3*4^{n}-4*3^{n})=4^{n+1}-3^{n+1}$
\\$7*4^n-7*3^n-3*4^{n}+4*3^{n}=4^{n+1}-3^{n+1}$
\\$4*4^n-3*3^n=4^{n+1}-3^{n+1}$
\\$4^{n+1}-3^{n+1}=4^{n+1}-3^{n+1}$
\\therefore $G_n=4^n-3^n$ for $n\ge 0$.$\hfill\blacksquare$

\vspace{0.1in}

\item \textbf{Problem 7.11.}
In each case tinker.
Then, guess a formula that solves the recurrence, and prove it.
\begin{enumerate}[(a)]
\item $P_0=0$, $P_1=a$, and $P_n=2P_{n-1}-P_{n-2}$, for $n>1$.
\item $G_1=1$; $G_n=(1-1/n)\cdot G_{n-1}$, for $n>1$.
\end{enumerate}

\vspace{0.1in}

\item \textbf{Problem 7.12(a-b).} (See Problem 7.28 for hints.)
Tinker to guess a formula for each recurrence and prove it.
In each case, $A_1=1$ and for $n>1$:
\begin{enumerate}[(a)]
\item $A_n=10A_{n-1}+1$.
\item $A_n=nA_{n-1}/(n-1)+n$.
\end{enumerate}

\vspace{0.1in}

\item \textbf{*Problem 7.12(c).} (See Problem 7.28 for hints.)
Tinker to guess a formula for each recurrence and prove it.
In each case, $A_1=1$ and for $n>1$:
\begin{enumerate}[(a)]
\setcounter{enumi}{2}
\item $A_n=10nA_{n-1}/(n-1)+n=\frac{10nA_{n-1}}{n-1}+n$.
\\$A_n=\frac{n(10^n-1)}{9}$
\\We prove this with induction.
\\Base case: $A_1=\frac{1(10^1-1)}{9}=1$
\\Incuction: assume $A_n=\frac{n(10^n-1)}{9}$
\\$\frac{10nA_{n-1}}{n-1}+n=\frac{n(10^n-1)}{9}$
\\$\frac{10(n+1)A_{n}}{n}+(n+1)=\frac{(n+1)(10^{n+1}-1)}{9}$
\\$\frac{10(n+1)\frac{n(10^n-1)}{9}}{n}+(n+1)=\frac{(n+1)(10^{n+1}-1)}{9}$
\\$\frac{10(n+1)n(10^n-1)}{9n}+(n+1)=\frac{(n+1)(10^{n+1}-1)}{9}$
\\$\frac{10(n+1)(10^n-1)}{9}+(n+1)=\frac{(n+1)(10^{n+1}-1)}{9}$
\\$\frac{10(n+1)(10^n-1)+9(n+1)}{9}=\frac{(n+1)(10^{n+1}-1)}{9}$
\\$\frac{(n+1)(10^{n+1}-10)+9(n+1)}{9}=\frac{(n+1)(10^{n+1}-1)}{9}$
\\$(n+1)\frac{(10^{n+1}-10)+9}{9}=\frac{(n+1)(10^{n+1}-1)}{9}$
\\$(n+1)\frac{10^{n+1}-1}{9}=\frac{(n+1)(10^{n+1}-1)}{9}$
\\$\frac{(n+1)(10^{n+1}-1)}{9}=\frac{(n+1)(10^{n+1}-1)}{9}$
\\therefore $A_n=\frac{n(10^n-1)}{9}$ for $\hfill\blacksquare$

\end{enumerate}

\vspace{0.1in}

\item \textbf{*Problem 7.13(a).}
Analyze these very fast-growing recursions.
[Hint: Take logarithms.]
\begin{enumerate}[(a)]
\item $M_1=2$ and $M_n=aM_{n-1}^2$ for $n>1$.
Guess and prove a formula for $M_n$.
Tinker, tinker.
\\$2^{2^{n-1}}*a^{2^n-1}$
\\$M_1=2$
\\$M_2=aM_1^2$
\\$M_3=aM_2^2$
\\$M_4=aM_3^2$
\\$M_5=aM_4^2$
\\$M_2=a(2^1a^0)^2=a(2^2a^0)=2^2a^1$
\\$M_3=a(2^2a^1)^2=a(2^4a^2)=2^4a^3$
\\$M_4=a(2^4a^3)^2=a(2^8a^6)=2^8a^7$
\\$M_5=a(2^8a^7)^2=a(2^16a^14)=2^16a^15$

\\We prove this with induction.
\\Base: 
\end{enumerate}

\vspace{0.1in}

\item \textbf{*Problem 7.19(d).}
Recall the Fibonacci numbers: $F_1,F_2=1$; and, $F_n=F_{n-1}+F_{n-2}$ for $n>2$.
\begin{enumerate}[(a)]
\setcounter{enumi}{3}
\item Prove that every third Fibonacci number, $F_{3n}$, is even.
\\We prove this with induction.
\\let $j,k\in\N_0$ so $2k$ is even and $2k+1$ is odd.
\\Base: $F_3=F_2+F_1=1+1=2=2k$ is even.
\\Induction: assume every third $F_n$ is even so $F_{3n}=2k$
\\$F_{3n}=F_{3n-1}+F_{3n-2}$
\\$F_{3(n+1)}=F_{3(n+1)-1}+F_{3(n+1)-2}$
\\$F_{3k}=F_{3n+2}+F_{3n+1}$
\\$2k=F_{3n+2}+F_{3n+1}$
\\$2k=F_{3n+1}+F_{3n}+F_{3n+1}$
\\$2k=2F_{3n+1}+2j$
\\$2k=2(F_{3n+1}+j)$
\\therefore $F_{3n}$ is even $\hfill\blacksquare$
\end{enumerate}

\vspace{0.1in}

\item \textbf{Problem 7.21.}
Show that every $n\ge 1$ is a sum of distinct Fibonacci numbers,
e.g.,~$11=F_4+F_6$; $20=F_3=F_5+F_7$.
(There can be many ways to do it, e.g.,~$6=F_1+F_5=F_2+F_3+F_4$.)
[Hints: Greedy algorithm; strong induction.]

\vspace{0.1in}

\item \textbf{Problem 7.41.}
Refer to the pseudocode on the right.
\begin{verbatim}
out=S([arr],i,j)
 if(j<i) out=0;
 else
  out=arr[j]+S([arr],i,j-1);
\end{verbatim}
\begin{enumerate}[(a)]
\item What is the function being implemented?
\item Prove that the output is correct for every valid input.
\item Give a recurrence for the runtime $T_n$, where $n=j-i$.
\item Guess and prove a formula for $T_n$.
\end{enumerate}

\vspace{0.1in}

\item \textbf{*Problem 7.42.}
Give pseudocode for a recursive function that computes $3^{2^n}$ on input $n$.
\begin{enumerate}[(a)]
\item Prove that your function correctly computes $3^{2^n}$ for every $n\ge 0$.
\item Obtain a recurrence for the runtime $T_n$.
  Guess and prove a formula for $T_n$.
\end{enumerate}

\vspace{0.1in}

\item \textbf{Problem 7.44.}
We give two implementations of \verb+Big(n)+ from page~90
(\verb+iseven(n)+ tests if $n$ is even).
\begin{multicols}{2}
\begin{enumerate}[(a)]
\item
\begin{verbatim}
out=Big(n)
 if(n==0) out=1;
 elseif(iseven(n))
    out=Big(n/2)*Big(n/2);
 else out=2*Big(n-1)
\end{verbatim}
\item
\begin{verbatim}
out=Big(n)
 if(n==0) out=1;
 elseif(iseven(n))
    tmp=Big(n/2); out=tmp*tmp;
 else out=2*Big(n-1)
\end{verbatim}
\end{enumerate}
\end{multicols}

\begin{enumerate}[(i)]
\item For each, prove that the output is $2^n$ and give a recurrence for the runtime $T_n$.
  (\verb+iseven(n)+ is two operations.)
\item For each, compute runtimes $T_n$ for $n=1,\ldots,10$.
  Compare runtimes with Exercise~7.10 on page~90.
\end{enumerate}
\vspace{0.1in}

\item \textbf{*Problem 7.45(c).}
Give recursive definitions for the set $\mathcal{S}$ in each of the following cases.
\begin{enumerate}[(a)]
\setcounter{enumi}{2}
\item $\mathcal{S}=\{$\ all strings with the same number of 0's and 1's\ $\}$
  (e.g.,~0011, 0101, 100101).
\end{enumerate}

\vspace{0.1in}

\item \textbf{Problem 7.45(a-b,d-f).}
Give recursive definitions for the set $\mathcal{S}$ in each of the following cases.
\begin{enumerate}[(a)]
\item $\mathcal{S}=\{0,3,6,9,12,\dots\}$, the multiples of~3.
\item $\mathcal{S}=\{1,2,3,4,6,7,8,9,11,\dots\}$, the numbers which are not multiples of~5.
\setcounter{enumi}{3}
\item The set of odd multiples of~3.
\item The set of binary strings with an even number of 0's.
\item The set of binary strings of even length.
\end{enumerate}

\vspace{0.1in}

\item \textbf{Problem 7.46.}
What is the set $\mathcal{A}$ defined recursively as shown?
(By default, nothing else is in $\mathcal{A}$---minimality.)
\begin{enumerate}[(1)]
\item $1\in\mathcal{A}$
\item $x,y\in\mathcal{A}\imp x+y\in\mathcal{A}$ \\
  $x,y\in\mathcal{A}\imp x-y\in\mathcal{A}$
\end{enumerate}

\vspace{0.1in}

\item \textbf{Problem 7.47.}
What is the set $\mathcal{A}$ defined recursively as shown?
(By default, nothing else is in $\mathcal{A}$---minimality.)
\begin{enumerate}[(1)]
\item $3\in\mathcal{A}$
\item $x,y\in\mathcal{A}\imp x+y\in\mathcal{A}$ \\
  $x,y\in\mathcal{A}\imp x-y\in\mathcal{A}$
\end{enumerate}

\vspace{0.1in}

\item \textbf{*Problem 7.49.}
There are 5 rooted binary trees (RBTs) with 3 nodes.
How many have 4~nodes?
\\$12$

\vspace{0.1in}

\item \textbf{Problem 8.12(a-c).}
A set $\mathcal{P}$ of parenthesis strings has a recursive definition (right).
\begin{enumerate}[(1)]
\item $\varepsilon\in\mathcal{P}$
\item $x\in\mathcal{P}\imp [x]\in\mathcal{P}$ \\
  $x,y\in\mathcal{P}\imp xy\in\mathcal{P}$
\end{enumerate}
\begin{enumerate}[(a)]
\item Determine if each string is in $\mathcal{P}$ and
  give a derivation if it is in $\mathcal{P}$. \\
  (i)~$[[[]]]][$\ \ \ \ (ii)~$[][[]][[]]$\ \ \ \ (iii)~$[[][][]$
\item Give two derivations of $[][][[]]$ whose steps are not a simple reordering of each other.
\item Prove by structural induction that every string in $\mathcal{P}$ has even length.
\end{enumerate}

\vspace{0.1in}

\item \textbf{*Problem 8.12(d).}
A set $\mathcal{P}$ of parenthesis strings has a recursive definition (right).
\begin{enumerate}[(1)]
\item $\varepsilon\in\mathcal{P}$
\item $x\in\mathcal{P}\imp [x]\in\mathcal{P}$ \\
  $x,y\in\mathcal{P}\imp xy\in\mathcal{P}$
\end{enumerate}
\begin{enumerate}[(a)]
\setcounter{enumi}{3}
\item Prove by structural induction that every string in $\mathcal{P}$ is balanced.
\end{enumerate}

\vspace{0.1in}

\item \textbf{Problem 8.13.}
Recursively define the binary strings that contain more 0's than 1's.
Prove:
\begin{enumerate}[(a)]
\item Every string in your set has more 0's than 1's.
\item Every string which has more 0's than 1's is in your set.
\end{enumerate}

\vspace{0.1in}

\item \textbf{*Problem 8.14.}
A set $\mathcal{A}$ is defined recursively as shown.
\begin{enumerate}[(1)]
\item $3\in\mathcal{A}$
\item $x,y\in\mathcal{A}\imp x+y\in\mathcal{A}$ \\
  $x,y\in\mathcal{A}\imp x-y\in\mathcal{A}$
\end{enumerate}
\begin{enumerate}[(a)]
\item Prove that every element of $\mathcal{A}$ is a multiple of~3.

\item Prove that every multiple of 3 is in $\mathcal{A}$.
\end{enumerate}

\vspace{0.1in}

\item \textbf{Problem 8.18.}
Recursively define rooted binary trees (RBTs) and rooted full binary trees (RFBTs).
\begin{enumerate}[(a)]
\item Give examples, with derivations, of RBTs and RFBTs with 5, 6, and 7 vertices.
\item Prove by structural induction that every RFBT has an odd number of vertices.
\end{enumerate}

\end{itemize}

\end{document}
