\documentclass[11pt]{article}
\usepackage{datetime}
\usepackage{color,array,graphics}
\usepackage{enumerate}
\usepackage[pdftex, colorlinks, linkcolor=red,citecolor=red,urlcolor=blue]{hyperref}
\usepackage{ulem}

\setlength{\parindent}{0cm}

\setlength{\parskip}{0.3cm plus4mm minus3mm}

\textwidth  6.5in
\oddsidemargin -0.5in
\evensidemargin +0.5in
\textheight 9.0in
\topmargin -1in

\usepackage{upquote,textcomp}
\usepackage{amssymb,amsmath,amsfonts,amsthm}
\usepackage{graphicx}
\usepackage{multicol}
\usepackage[T1]{fontenc}

\def\NOT{\neg}
\def\OR{\vee}
\def\AND{\wedge}
\def\imp{\rightarrow}

\DeclareSymbolFont{AMSb}{U}{msb}{m}{n}
\DeclareMathSymbol{\N}{\mathbin}{AMSb}{"4E}
\DeclareMathSymbol{\Z}{\mathbin}{AMSb}{"5A}
\DeclareMathSymbol{\R}{\mathbin}{AMSb}{"52}
\DeclareMathSymbol{\Q}{\mathbin}{AMSb}{"51}
\DeclareMathSymbol{\I}{\mathbin}{AMSb}{"49}
\DeclareMathSymbol{\C}{\mathbin}{AMSb}{"43}


\begin{document}

CSCI-2200 FOCS F 2022 Crib Sheet Exam 1 Tuesday, October 5, 2022 Hayden Fuller
\begin{large}
\noindent Notes:
\\NEGATION: $a\OR b$ => $\NOT a\AND\NOT b$; $a\AND b$ => $\NOT a\OR\NOT b$; if $a$, then $b$=$a\imp b$ => $a\AND\NOT b$
\\$\forall x$, $A(x)$ => $\exists x : \NOT A(x)$; $\exists x:A(x)$ => $\forall x \NOT A(x)$
\\$\forall x:(\exists y:2x-y=0)=T$(already know $x$); $\exists y:(\forall x:2x-y=0)=F$(can't predict $x$); $\exists y:(\forall x:xy=0)=T$(x doesn't matter, y=0)
\\statement: $p\imp q$ converse: $q\imp p$ inverse: $\NOT p\imp\NOT q$ contrapositive: $\NOT q\imp\NOT p$
\\$\cup$ union; $\cap$ intersection; $\subset$ proper subset; $\subseteq$ subset(can be equal);

What type of proof is appropriate? Contradiction:
\\There is a prime number greater than $ab$
\\$2^{\frac{1}{p}}$ is irrational for any integer $p>2$, given: $a,b,c,n\in\N$, $n>2$, $a^n+b^n\ne c^n$
\\Contraposition:
\\Direct:
\\There is an even number grater than $ab$ (either $ab+1$ or $ab+2$ is even)
\\if $n$ and $q$ are natural numbers, then there exists unique intergers $d$ and $r$ satisfying $n=dq+r$, with $d\in{0}\cup\N$ and $0\le r\le q$.
\\Leaping induction:
\\Induction:
\\$(a+b)^n\ge a^n+b^n$, when $n$ is a natural number.
\\$n^2\le2^n$ for all $n\ge4$
\\$5$ divides $11^n-6$ for all $n\ge5$; show $5$ divides $11^5-6$ and show for $n\ge5$, if $5$ divides $11^n-6$, then $5$ divides $11^{n+1}-6$
\\Weak induction:
\\$11^n-6$ is divisible by $5$ if $n$ is a natural number.

\end{large}

\end{document}



























