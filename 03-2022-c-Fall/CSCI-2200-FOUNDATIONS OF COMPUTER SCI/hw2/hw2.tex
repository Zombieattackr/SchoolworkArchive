
\documentclass[11pt]{article}
\usepackage{datetime}
\usepackage{color,array,graphics}
\usepackage{enumerate}
\usepackage[pdftex, colorlinks, linkcolor=red,citecolor=red,urlcolor=blue]{hyperref}
\usepackage{ulem}

\setlength{\parindent}{0cm}

\setlength{\parskip}{0.3cm plus4mm minus3mm}

\textwidth  6.5in
\oddsidemargin +0.0in
\evensidemargin +0.0in
\textheight 9.0in
\topmargin -0.5in

\usepackage{upquote,textcomp}
\usepackage{amssymb,amsmath,amsfonts,amsthm}
\usepackage{graphicx}
\usepackage{multicol}
\usepackage[T1]{fontenc}

\def\OR{\vee}
\def\AND{\wedge}
\def\imp{\rightarrow}

\DeclareSymbolFont{AMSb}{U}{msb}{m}{n}
\DeclareMathSymbol{\N}{\mathbin}{AMSb}{"4E}
\DeclareMathSymbol{\Z}{\mathbin}{AMSb}{"5A}
\DeclareMathSymbol{\R}{\mathbin}{AMSb}{"52}
\DeclareMathSymbol{\Q}{\mathbin}{AMSb}{"51}
\DeclareMathSymbol{\I}{\mathbin}{AMSb}{"49}
\DeclareMathSymbol{\C}{\mathbin}{AMSb}{"43}

\begin{document}
\thispagestyle{empty}   %% skips page number on the first page

\begin{center}
\large
\textbf{CSCI 2200 --- Foundations of Computer Science (FoCS) \\
Homework 2}
\\Hayden Fuller, Ben Dennison, Alex Litchfield, Kaylin Rackley
\end{center}

\begin{itemize}

\item \textbf{*Problem 3.59 (Closure).} 
\begin{enumerate}[(a)]
\item $\N$
\\addition, multiplication, exponentiation
\item $\Z$
\\addition, subtraction, multiplication, exponentiation
\item $\Q$
\\addition, subtraction, multiplication, division, exponentiation
\item $\R$
\\addition, subtraction, multiplication, division, exponentiation
\end{enumerate}

\vspace{0.1in}

\item \textbf{*Problem 4.7(b).} 
\\$n \in \Z \imp n^2+n$ is even.
\\We prove the implication using a direct proof.
\\assume $n \in \Z$
\\if $n$ is even, $n=2k$
\\$(2k)^2+2k=4k^2+2k=2(2k^2+k)$
\\therefore $n^2+n$ is even for even $n$
\\if $n$ is odd, $n=2k+1$
\\$(2k+1)^2+(2k+1)=4k^2+4k+1+2k+1=4k^2+6k+2=2(2k^2+3k+1)$
\\therefore $n^2+n$ is even for odd $n$
\\so $n^2+n$ is even for $n \in \N$
\\therefore $n \in \Z \imp n^2+n$ is even.$\hfill\blacksquare$

\vspace{0.1in}

\item \textbf{*Problem 4.10(k-l).} 
\\(k)
\\$3$ divides $n-2 \imp n$ is not a perfect square.
\\We will prove this by contra position:
\\Contraposition:
\\n is a perfect square $\rightarrow$ 3 does not divide $n-2$
\\assume n is a perfect square $k^2$
\\$k^2 \rightarrow$3 does not divide $k^2 - 2$
\\Because n is an integer, we can prove this for both even and odd cases
\\\textbf{Even Case:}
\\Assume k is an even number 4m + 2
\\$k^2=(4m+2)^2=16m^2+8m+4=2(8m^2+4m+2)$
\\This means that $k^2$ is even when k is even which implies that $k^2 - 2$ will also be even which means it will not be divisible by 3
\\\textbf{Odd Case:}
\\Assume k is an odd number 2m+1
\\$k^2=(2m+1)^2=4m^2+4m+1=2(2m^2+2m+1)$
\\This means that $k^2$ is even which also implies that $k^2 - 2$ will always be even. This means $k^2 - 2$ will not be divisible by 3.$\hfill{\blacksquare}$
\\(\textbf{L})
If $p > 2$ is prime, then $p^2+1$ is composite
\\We prove the implication using contraposition.
\\Contraposition:
\\If $p^2 + 1$ is prime $\imp$ $p > 2$ is composite
\\A prime is always odd. This means that $p^2 + 1$ will always be odd if we assume that our statement \textbf{P} is true.
\\therefore, $p^2$ is always even. Even numbers are always composite because they can be divided by 1, 2, and itself.
\\This means that if $p^2 + 1$ is prime, $p > 2$ must be even which makes it a composite number.$\hfill{\blacksquare}$

\vspace{0.1in}

\item \textbf{*Problem 4.48(c).} 
\\Use the concept of "without loss of generality" to prove these claims.
\\For any non-zero real number $x$, $x^2+\frac{1}{x^2} \ge 2$.
\\This is a direct proof. Suppose $x$ is positive or negative. Then, $x^2$ will always be positive. 
Without loss of generality, $x=\sqrt{2}$.
\\$\sqrt{2}^2 + 1/\sqrt{2}^2 \ge 2$
\\$2 + 1/2 \ge 2$
\\$2 + 1/2 \ge 2$ means this must be true for all cases.$\hfill\blacksquare$

\vspace{0.1in}

\item \textbf{*Problem 5.12(d).} 
\\For $n \ge 1$, prove by induction:
\\$3^n > n^2$.
\\Base Step P(1):
\\$3^1 > 1^2$.
\\$3>1$ so $3^n > n^2$ is true for $n=1$
\\Induction Step P(n+1):
\\$3^{n+1} > (n+1)^2$
\\$\frac{3^{n+1}}{n+1} > n+1$
\\$\frac{3^{n+1}}{n+1}-n > 1$
\\With this we can plug in n = 0
\\$\frac{3^1}{1} > 1$
\\$3 > 1$ means this must be true of all cases.$\hfill\blacksquare$

\vspace{0.1in}

\item \textbf{*Problem 5.20.} 
\\Prove, by induction, that every $n \ge 1$ is a sum of distinct powers of 2.
\\Base Step P(1):
\\$n = 1 = \sum_{i=0}^{k=1}{2^k} = 2^1 - 2^0 = 2 - 1 = 1$
\\$\sum_{i=0}^{k=1}{2^k} = 1$ so $n \ge 1$ being a distinct power of 2 is true.
\\Induction Step P(n+1):
\\$\sum_{i=0}^{k=n+1}{2^k} = n+1$
\\$2^{n+1} - 2^0 = n + 1$
\\$2^{n+1} - 1 = n + 1$
\\$2^{n+1}  = n + 2$ so $n \ge 1$ being a distinct power of 2 is true.$\hfill\blacksquare$

\vspace{0.1in}

\item \textbf{*Problem 5.39.} 
\\Prove you can make any postage greater than 12c using only 4c and 5c stamps. 
\\We prove this with leaping induction
\\$c=4a+5b$
\\bases) $12=4*3+5*0$, $13=4*2+5*1$, $14=4*1+5*2$, $15=4*0+5*3$
\\Case 1) c=4k
\\Prove $c=4a+5b \imp c+4=4c+5d$
\\assume $c=4a+5b$
\\$d=0$
\\$4k+4=4a+5(0)$
\\$4(k+1)=4a+5(0)$
\\$a=k+1$
\\therefore $c=4a+5b$ is true for $c=4k$$\hfill\blacksquare$
\\Case 2) c=4k+1
\\Prove $c=4a+5b \imp c+4=4c+5d$
\\assume $c=4a+5b$
\\$d=1$
\\$4k+1+4=4a+5(1)$
\\$4k+5=4a+5$
\\$a=k$
\\therefore $c=4a+5b$ is true for $c=4k+1$$\hfill\blacksquare$
\\Case 3) c=4k+2
\\Prove $c=4a+5b \imp c+4=4c+5d$
\\assume $c=4a+5b$
\\$d=2$
\\$4k+2+4=4a+5(2)$
\\$4k+6=4a+10$
\\$4(k-1)+10=4a+10$
\\$a=k-1$
\\therefore $c=4a+5b$ is true for $c=4k+2$$\hfill\blacksquare$
\\Case 4) c=4k+3
\\Prove $c=4a+5b \imp c+4=4c+5d$
\\assume $c=4a+5b$
\\$d=3$
\\$4k+3+4=4a+5(3)$
\\$4k+7=4a+15$
\\$4(k-2)+15=4a+15$
\\$a=k-2$
\\therefore $c=4a+5b$ is true for $c=4k+3$$\hfill\blacksquare$
\\therefore $c=4a+5b$ is true for $c\ge12$$\hfill\blacksquare$

\vspace{0.1in}

\item \textbf{*Problem 6.8.} 
\\Prove $n^7<2^n$ for $n\ge37$.
\\(a) Use induction
\\Base Step:
\\$n = 37$
\\$37^7<2^{37}$
\\$9.4931877133\times10^{10} < 1.3743895347\times10^{11}$
\\so $n^7 < 2^n$ for $n = 37$ is true.
\\Induction Step P(n+1)
\\$2^{k+1}=2*2^k>2k^7>k^7+7k^6+21k^5+35k^4+35k^3+21k^2+7k+1=(k+1)^7$
\\$2^{k+1}>(k+1)^7$
\\Therefore $n^7<2^n$ for $n\ge37$.
\\ 
\\(b) Use leaping induction
\\Base Step 1:
\\$n = 37$
\\$37^7<2^{37}$
\\$9.4931877133\times10^{10} < 1.3743895347\times10^{11}$
\\so $n^7 < 2^n$ for $n = 37$ is true.
\\Induction step 1:
\\$2^{k+2}=4*2^k>4k^7>k^7+14k^6+84k^5+280k^4+560k^3+672k^2+448k+128=(k+2)^7$
\\$2^{k+2}>(k+2)^7$
\\Therefore $n^7<2^n$ for $n=37+2k$ where $k\in\N_0$$\hfill\blacksquare$
\\Base Step 2:
\\$n = 38$
\\$38^7<2^{38}$
\\$1.1441558259\times10^{11} < 2.7487790694\times10^{11}$
\\so $n^7 < 2^n$ for $n = 38$ is true.
\\Induction step 2:
\\$2^{k+2}=4*2^k>4k^7>k^7+14k^6+84k^5+280k^4+560k^3+672k^2+448k+128=(k+2)^7$
\\$2^{k+2}>(k+2)^7$
\\Therefore $n^7<2^n$ for $n=38+2k$ where $k\in\N_0$$\hfill\blacksquare$
\\Therefore $n^7<2^n$ for $n\ge37$$\hfill\blacksquare$

\vspace{0.1in}

\item \textbf{*Problem 6.43.} 
\\for a 3x3 sliding puzzle, the number of inversions always stays odd
\\This is a direct proof. Suppose $x$ represents the number of inversions and that $x = 1$.
\\Suppose:
\\Vertical sliding of tile in either direction can be represented as $2b$, where $b \in \N$.
\\Horizontal sliding of tile in either does not change the number of inversions.
\\$x = 1 + 2b$
\\$x = 2b + 1$ is an odd number, therefore the number of inversions remains odd.$\hfill\blacksquare$

\vspace{0.1in}

\item \textbf{*Problem 7.4(c).} 
\\Guess a formula for $A_n$ and prove it by induction.
\\(a)
$A_0=0$ and $A_n=A_{n-1}+1$ for $n \ge 1.$
\\$A_n=n$
\\We prove this by induction
\\$A_0=0$ is True.
\\assume $A_n=n$
\\$A_{n+1}=n+1$
\\$A_{n+1}=A_{n+1-1}+1=A_n+1=A_n+1=n+1$
\\$A_n+1=n+1$
\\$n+1=n+1$
\\therefore $A_n=n$$\hfill\blacksquare$
\\
\\(b)
$A_0=1$; $A_1=2$; $A_n=2A_{n-1}-A_{n-2}+2$ for $n \ge 2$. [method of differences]
\\$A_n=n^2+1$
\\We prove this by induction
\\$A_0=0^2+1=1$ is True.
\\assume $A_n=n^2+1$
\\$A_{n+1}=(n+1)^2+1$
\\$A_{n+1}=2A_{n+1-1}-A_{n+1-2}+2=2A_{n}-A_{n-1}+2=(n+1)^2+1$
\\$2(n^2+1)-((n-1)^2+1)+2=(n+1)^2+1$
\\$2n^2+2-((n^2-2n+1)+1)+2=(n+1)^2+1$
\\$2n^2+2-(n^2-2n+2)+2=(n+1)^2+1$
\\$2n^2+2-n^2+2n-2+2=(n+1)^2+1$
\\$2n^2+2-n^2+2n=(n+1)^2+1$
\\$n^2+2n+1+1=(n+1)^2+1$
\\$(n+1)^2+1=(n+1)^2+1$
\\therefore $A_n=n^2+1$ is True$\hfill\blacksquare$

\end{itemize}

\end{document}
