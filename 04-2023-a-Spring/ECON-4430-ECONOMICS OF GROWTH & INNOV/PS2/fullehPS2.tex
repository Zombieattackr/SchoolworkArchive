
\documentclass[11pt]{article}
\usepackage{datetime}
\usepackage{color,array,graphics}
\usepackage{enumerate}
\usepackage[pdftex, colorlinks, linkcolor=red,citecolor=red,urlcolor=blue]{hyperref}
\usepackage{ulem}

\setlength{\parindent}{0cm}

\setlength{\parskip}{0.3cm plus4mm minus3mm}

\textwidth  6.5in
\oddsidemargin +0.0in
\evensidemargin +0.0in
\textheight 9.0in
\topmargin -0.5in

\usepackage{upquote,textcomp}
\usepackage{amssymb,amsmath,amsfonts,amsthm}
\usepackage{graphicx}
\usepackage{multicol}
\usepackage[T1]{fontenc}

\def\OR{\vee}
\def\AND{\wedge}
\def\imp{\rightarrow}

\DeclareSymbolFont{AMSb}{U}{msb}{m}{n}
\DeclareMathSymbol{\N}{\mathbin}{AMSb}{"4E}
\DeclareMathSymbol{\Z}{\mathbin}{AMSb}{"5A}
\DeclareMathSymbol{\R}{\mathbin}{AMSb}{"52}
\DeclareMathSymbol{\Q}{\mathbin}{AMSb}{"51}
\DeclareMathSymbol{\I}{\mathbin}{AMSb}{"49}
\DeclareMathSymbol{\C}{\mathbin}{AMSb}{"43}

\begin{document}
\thispagestyle{empty}   %% skips page number on the first page

\begin{center}
\large
\textbf{Econ of Growth and Innovation \\
PS2}
\\Hayden Fuller
\end{center}

\begin{itemize}

\item \textbf{1} 
\begin{enumerate}[(a)]
\item law of motion
\\I understand the equations when I see them, but I'm still really 
\\struggling to find the steps to derive them...
\\$Y=K$
\\$\dot k=sf(k)-(n+\delta)k$
\\$Y=K^\alpha H^\beta(AL)^{1-\alpha-\beta}$
\\$\dot k=s_kf(k,h,AL)-(n+\delta)k$
\\$\dot h=s_hf(k,h,AL)-(n+\delta)h$
\\..?
\item equilibrium points
\\$\dot k=\dot h=0$
\\$k^*=(\frac{s_h^\beta s_k^{1-\beta}A}{n+\delta})^{\frac{1}{1-\alpha-\beta}}$
\\$h^*=(\frac{s_h^\alpha s_k^{1-\alpha}A}{n+\delta})^{\frac{1}{1-\alpha-\beta}}$
\\
\\$\hat k^*=(\frac{s_h^\beta s_k^{1-\beta}}{n+\delta})^{\frac{1}{1-\alpha-\beta}}$
\\$\hat h^*=(\frac{s_h^\alpha s_k^{1-\alpha}}{n+\delta})^{\frac{1}{1-\alpha-\beta}}$
\\?
\item Comparative statics \#1
\vspace{2in}
\item Comparative statics \#2
\vspace{2in}
\end{enumerate}
\vspace{2in}

\vspace{0.1in}

\item \textbf{2} 
\begin{enumerate}[(a)]
\item increased immigration
\\increased rate of population growth leads to a greater rate of depreciation. This means rate of depreciation will be greater than the rate of investment, leading to a decrease in k towards the new equilibrium point
\item $y=Rk+w$
\vspace{2in}
\item factor price
\vspace{2in}
\item policy
\\This model does show why people are understandably worried about increases in immigration rates, but it also fails to capture some other real world aspects. This model shows why people are worried that increased immigration rates will lead to decreased wages and an increase in interest rates, but this is all assuming one single perfectly competitive closed market. Realistically, the unskilled labor that people tend to be scared of taking their jobs are in very different markets from them, many markets are far from perfectly competitive, and many jobs are easily sent overseas. This new labor source and lower wages also decreases the incentive to offshore jobs. Additionally, immigration accounts for a quite small percentage of our population growth.
\end{enumerate}
\vspace{2in}


\end{itemize}

\end{document}




















































