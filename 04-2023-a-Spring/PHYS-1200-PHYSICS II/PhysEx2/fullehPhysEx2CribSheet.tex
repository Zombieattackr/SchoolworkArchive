\documentclass{article}

\usepackage{amsmath}
\usepackage{amssymb}
\usepackage{bm}
\usepackage{graphicx}
\usepackage{epstopdf}
\DeclareGraphicsRule{.tif}{png}{.png}{`convert #1 `basename #1 .tif`.png}
\usepackage{color}
\usepackage{pdfsync}
\pagestyle{plain}

\textheight 10.5 true in
\textwidth 8 true in
\hoffset -1.5 true in
\voffset -1.5 true in
\mathsurround=2pt
\parskip=2pt


\begin{document}

Physics II S 2023 Crib Sheet Exam 1 Hayden Fuller
\begin{large}
\noindent Notes:
\\\indent Culombs Law, conductors, insulators, polarization, induced charges, adding vector fields and forces
\\$\vec F_{1on2}=\vec F_{12}=-\vec{F}_{21}=q_2\vec{E}_1=k\frac{q_1q_2}{r_{12}^2}\frac{\vec{r}_{12}}{r_{12}}=k\frac{q_1q_2}{r_{12}^2}\hat r_{12}$; $\vec{F}_{tot}=q_0\vec{E}_{tot}$; $\vec{E}_{tot}(X_0,y_0,z_0)=\int \vec{dE}(x',y',z')=\int k\frac{dq'(x',y',z')}{r_0'^2}\frac{\vec{r}'_0}{r'_0}$, $\vec{r}'_0=\vec{r}_0-\vec{r}'=(x_0-x')\hat i+$... , $\vec{r}'=x'\hat i +$...
\\distance away from line charge linearly, line starts at 0, at x=-D, $\vec{E}=-k\int_0^L\frac{\lambda dx'}{(D+x')^2}\hat i$, $V=k\lambda\ln(\frac{D+L}{D})$
\\with $\theta$ up from x axis, $r_x=x\cos\theta$, $r_y=y\sin\theta$, $r=\sqrt{r_x^2+r_y^2}$, $k=9*10^9=\frac{1}{4\pi \epsilon_0}$, $\epsilon_0=8.85*10^{-12}$
\\\indent Electric field for point charges, electric field for a continuous distribution of charge
\\$\vec{F}_E=q\vec{E}$; $\vec{E}_s=k\frac{q_s}{r^2}\frac{\vec{r}}{r}=k\frac{q_s}{r^2}\hat r$
\\\indent Gauss's law and elecctric flux through a surface, Use of Gauss's law to find field
\\$\Phi_E=\oint\vec{E}\cdot d\vec{A}=\int E \cdot dA \cos \phi=\frac{Q_{encl}}{\epsilon_0}$, $\phi=\angle \vec E-d\vec A$, $d\vec A= dA\hat n$ net elec field $\vec{E}=0$, $V=c$ within a cond.
\\gauss sphere: $\Phi_E=\oint\vec E (r)\cdot d\vec A=E(r)4\pi r^2$, $E(r)=k\frac{q}{r^2}$,
\\sphere radius R: outside or point charge: $V=k\frac{q}{r}$, $E=k \frac{q}{r^2}$ inside: cond: $V=k\frac{q}{R}$, $E=0$, insulating: $E=k\frac{qr}{R^3}$
\\long thin wire: $E(r)=\lambda/(2\pi r\epsilon_0)$ \quad thin flat sheet: $E=\sigma/(2\epsilon_0)$, stepped: go from in to out matching net $Q_in$
\\infinite plane w/ cylinder in it, $E=\sigma/\epsilon_0$
\\\indent Electric potential for point charge, distribution. Electric field vs potential, equipotential. Potential for group of points, conservation of energy.
\\Change Elec Pot Engry $\Delta U=-\int_{\vec{r}_A}^{\vec{r}_B} q\vec{E}\cdot d\vec{s}=-W_{AB}$; Change Elec Pot $\Delta V =\frac{\Delta U_E}{q}= -\int_{\vec{r}_A}^{\vec{r}_B} \vec{E}\cdot d\vec{s}$ so $\Delta U_E = q\Delta V$
\\Point charge, $\Sigma$ for system $V(r)=\frac{kq}{r}$, $U_E=k\frac{q_1q_2}{R_{12}}+$...; Field from pot: $E_x=-\Delta V=-\frac{\delta V}{\delta x}-$... .
\\work on closed path =0; 
\\\indent Caps, Dielectrics, steads state, equiv, energy storage, electric fielld energy density
\\$C=Q/V=\frac{\epsilon_0 A}{d}=kC_0$, ElcPotEnrInCap $U_E=.5 Q V =.5 Q^2/C = .5 C V^2$, EnrFieldDen $u_E=.5 \epsilon_0 E^2$, $E=\frac{\sigma}{k\epsilon_0}$, $V_1=V\frac{C_{equiv}}{C_1}$
\\\indent Current and densityJ, Resistance and itivity, Power relations and dissipation, DC steady state, KCVL Ohms
\\$I=\frac{dQ}{dT}$, $I=\vec J d\vec A$, $\vec J=q n \vec v_d=I/A$. $E=\rho J$, $V=IR$, $R=\rho L/A$, $P=IV=I^2R=V^2/R$; $V_{bat}=$EMF$-Ir$
\\Temp: conductor: $\rho(T)=\rho_0+\rho_0\alpha(T-T_0)$ semi: $\rho(T)=\rho_0e^{(\frac {E_a}{kT})}$, $E_a=$ actiEngr, $k=1.38e-23=$bolt const.
\\\indent Magnetic forces and fields
\\$\vec{F}=q \vec v \times \vec B$, finger velocity, curl field, thumb force, flip for negative. $\vec F_B=I \vec L \times \vec B$, $r=\frac{mv}{|q|B}$
\\\indent misc
\\$W=q\Delta V$, Centripital force $F=mv^2/r$, $E=-\Delta V/d$, $V=kq/r$, $V=\Delta KE=-\Delta PE$, $KE=0.5*mv^2$
\\$F=ma$, earth south is north, use conventional, $\vec c=\vec a \times\vec b$, $|\vec c|=|\vec a| |\vec b|\sin\theta_{ab}$, cross is det, dot is sum
\\RMS = $\sqrt{\sum(x^2)}$, \%error = (act-exp)/exp
\\
\begin{tabular}{ c c c c c }
Force 		& $F$ 		& $kg*m/s^2$ 	& Newton 		& $N$ \\
Energy/Work	& $U, KE | W$	& $N*m, W*s$ 	& Joule 		& $J$ \\
Charge 		& $Q$		& $A*s$ 		& Coulomb 	& $C$ \\
Chg den linear	& $\lambda$ 	& $C/m$		& -- 			& $C/m$ \\
Chg den surface	& $\sigma$ 	& $C/m^2$	& -- 			& $C/m^2$ \\
Chg den volume	& $\rho$ 		& $C/m^3$	& -- 			& $C/m^3$ \\
Elec Field 		& $E$		& $N/C$ 		& -- 			& $N/C$ \\
Elec Flux 		& $\Phi$ 		& $N*m^2/C$ 	& -- 			& $Nm^2/C$ \\
Elec Potential	& $V$ 		& $J/C, W/A$	& Volt		& $V$ \\
Current		& $I$			& $C/s$		& Amp		& $A$ \\
Current density	& $J$		& $I/m^2$		& --			& $I/m^2$ \\
Resistance		& $R$		& $V/A$		& Ohm		& $\Omega$ \\
Resistivity		& $\rho$		& $E/J, RA/L$	& -- 			& $\Omega m$ \\
Power		& $P$		& $VA, J/s$	& Watt		& $W$ \\
Capacitance 	& $C$		& $Q/V$		& Farad		& $F$ \\
Magnetic field	& $B$		& $Ns/Cm,N/mA$	& Tesla	& $T$ \\
Magnetic field	& $\Phi$		& $Tm^2$		& Weber		& $Wb$ \\

\end{tabular}
\newpage
\indent Sources of magnetic fields, law of Biot-Savart for moving charges and current elements, Magnetic fields of current carrying wires and loops, Magnetic forces between conductors
\\field from point charge moving $\vec B=\frac{\mu_0}{4\pi} \frac{q\vec v\times\vec r}{r^3}=\frac{\mu_0}{4\pi} \frac{q\vec v\times\hat r}{r^2}$, $B=\frac{\mu_0}{4\pi}\frac{qv\sin\theta}{r^2}$, velocity, radius to measurement, from current element Biot-Savart swap $q\vec v>\int Id\vec l$, right hand, thumb conventional current/positive charge. axis of loop: $B_x=\frac{\mu_0Ia^2}{2(x^2+a^2)^{3/2}}=\frac{\mu_0\mu}{2\pi(s^2+a^2)^{3/2}}$, $\mu=IA$, $x=$far -> $B=\frac{\mu}{x^3}$
\\Current same direction, fields oppose, attract. $F=L\frac{\mu_0I_1I_2}{2\pi r}$
\\Straight wire: $B=\frac{\mu_0I}{2\pi r}$, Center of a loop: $B=\frac{\mu_0I}{2r}$, inside:$\frac{\mu_0I}{2\pi R^2}r$
\\Solenoid: inductancec: $L=\frac{\Phi_B}{i}=\frac{N\Phi_{B,loop}}{i}=\frac{NBA_{loop}}{i}=\frac{Nu_0niA_{loop}}{i}=\mu_0N\frac{N}{l}A_{loop}=\frac{\mu_0N^2\pi r_s^2}{l}=\pi\mu_0n^2r_s^2l$
\\inside a Solenoid: $B=\mu_0nI$, voltage $\int_a^b\vec E_{nc}\cdot d\vec l=-\frac{d\Phi_B}{dt}=-L\frac{di}{dt}$, i from + to - increase, EMF 
\\\indent Ampere's law, calculating magnetic fields from ampere's law. Maagnetic moments and magnetism, magnetic force and torque on a current loop/magnetic moment
\\Ampere's law: $\oint\vec B\cdot d\vec l=\mu_0I_{enc}$ total field in a circular path around a wire is equal to $\mu_0$ times current enclosed
\\current density $\vec J$, $I_{enc}=\int\vec J_{net}\cdot d\vec A=J\cdot A \cos\theta$
\\Magnetic moment: $\vec \mu=I\vec A$, current in loop times area of loop, right hand direction. Torque $\tau_{B,net}=\vec\mu\times\vec B$, right hand rule for spin direction
\\\indent Magnetic flux, Faraday's law, Lenz's law, Electromagnetic Induction.
\\Magnetic flux $\Phi_B=\int\vec V\cdot d\vec A=\int B dA\cos\theta$, Faraday's law: EMF from changing Mflux $\epsilon=\oint\vec E\cdot d\vec l=-\frac{d}{dt}\Phi_B$, for $N$ loops, $\cdot N$.\qquad Lenz's law, this EMF induces oposing(attracting) magnetic field. B increase up, EMF and i cw, induced B down, net small B up 
\\\indent Displacement current, Maxwell's equations:
"displacement current" is built up charge, $I_d=\epsilon_0\frac{d}{dt}\Phi_E$, fixed Ampere's $\oint\vec B\cdot d\vec l=\mu_0(I_c+I_d)_enc=\mu_0I_{c,enc}+\mu_0\epsilon_0\frac{d}{dt}\Phi_{E,enc}$
\\Maxwell's: Gauss's for $\vec E$: $\oint\vec E\cdot d\vec A=\frac{q_{enc}}{\epsilon_0}$ for $\vec B$: $\oint\vec B\cdot d\vec A=0$
\\Faraday stationary: $\oint\vec E\cdot d\vec l=-\frac{d}{dt}\Phi_B$, Ampere stationary: $\oint\vec B\cdot d\vec l=\mu_0(i_c+\epsilon_0\frac{d}{dt}\Phi_E)_enc$
\\\indent Self and mutual Inductance, EMF and current in circuits, Magnetic field energy and energy density
\\self inductance: $\Phi_B=Li$, $L=\frac{\Phi_B}{i}$, Mutual: $M=M_{12}=\frac{N_1\Phi_{B1}}{i_2}=M_{21}=\frac{N_2\Phi_{B2}}{i_1}$, $\frac{d\Phi_B}{dt}=\frac{d}{dt}Li=L\frac{di}{dt}$, $\epsilon_L=-L\frac{di}{dt}$, $\epsilon_1=-M\frac{di_2}{dt}$
\\magnetic energy in an inductor $U_B=0.5Li^2$, region in field $\vec B$ has energy density $u_B=\frac{U_B}{v}=\frac{B^2}{2\mu_0}$
\\\indent Circuit Transients, RC, RL, LC, and RLC. Charactaristic decay times and oscillation frequencies
\\$I(C)=C\frac{dV_C}{dt}$, $V(L)=L\frac{dI_L}{dt}$
\\RC: charge $q(t)=C\epsilon(1-e^{-t/RC})$, $i=\frac{dq}{dt}$, $i(t)=\frac{\epsilon}{R}e^{-t/RC}$ discharge: $q(t)=Q_0e^{-t/RC}$, $i(t)=-\frac{Q_0}{RC}e^{-t/RC}$
\\RL: charge $i(t)=\frac{\epsilon}{R}(1-e^{-tR/L})$, discharge $i(t)=i_0e^{-tR/L}$
\\LC: Q(C) $q(t)=Q\cos(\omega t+\phi)$, I(L) $i(t)=\frac{dq}{dt}=-\omega Q\sin(\omega t+\phi)$, $\omega=1/\sqrt{LC}$ $T=\frac{2\pi}{\omega}$, $\omega=2\pi*\omega$, $U_E=\frac{(q(t))^2}{2C}$, $U_B=0.5L(i(t))^2$, $U_{tot}=U_E+U_B=\frac{Q^2}{2C}$, $L\frac{di}{dt}=-\frac{q}{C}$, $\frac{d^2q}{dt^2}=-\frac{1}{LC}q$
\\\indent Alternating current circuits, phasors, reactance, impedance, resonance, power, transformers
\\AC: $RMS=\frac{1}{\sqrt{2}} max$, $X_L=\omega L$, $V_L$ is 90 ahead, $X_C=\frac{1}{\omega C}$, $V_C$ is 90 behind
\\$i(t)=I\cos(\omega t)$,  L: $V_L(t)=\omega LI\cos(\omega t+\pi/2)=V_L\cos(\omega t+\pi/2)$
\\series LRC AC: $V=\sqrt{V_R^2+(V_L-V_C)^2}=I\sqrt{R^2+(X_L-X_C)^2}$, *net* impedance $Z=\sqrt{R^2+(X_L-X_C)^2}$
\\current phasor is shared, $V_R$ matches, $V_L$ leads 90, $V_C$ lags 90, $V_S=VR+VL+VC$, some phase inbetween $\phi$, $\tan\phi=\frac{V_L-V_C}{V_R}=\frac{X_L-X_C}{R}$, resonance: at $\omega_0$, $X_L=X_C$, $Z=R$, 
\\$q(t)=Qe^{-t/\tau_d}\cos(\omega't+\phi)$, $\tau_d=2L/R$, $\omega'=\sqrt{\frac{1}{LC}-(\frac{R}{2L})}^2$
\\Power: $P_{average}=0.5V_{amp}I_{amp}\cos\phi_{V-I}=V_{RMS}I_{RMS}\cos\phi_{V-I}$, $\cos\phi=R/Z$ for series LRC
\\Transformer: $\frac{V2}{V1}=\frac{N2}{N1}$

\end{large}

\end{document}




























