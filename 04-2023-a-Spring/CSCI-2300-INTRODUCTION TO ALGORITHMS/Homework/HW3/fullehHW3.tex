
\documentclass[11pt]{article}
\usepackage{datetime}
\usepackage{color,array,graphics}
\usepackage{enumerate}
\usepackage[pdftex, colorlinks, linkcolor=red,citecolor=red,urlcolor=blue]{hyperref}
\usepackage{ulem}

\setlength{\parindent}{0cm}

\setlength{\parskip}{0.3cm plus4mm minus3mm}

\textwidth  6.5in
\oddsidemargin +0.0in
\evensidemargin +0.0in
\textheight 9.0in
\topmargin -0.5in

\usepackage{upquote,textcomp}
\usepackage{amssymb,amsmath,amsfonts,amsthm}
\usepackage{graphicx}
\usepackage{multicol}
\usepackage[T1]{fontenc}

\def\OR{\vee}
\def\AND{\wedge}
\def\imp{\rightarrow}

\DeclareSymbolFont{AMSb}{U}{msb}{m}{n}
\DeclareMathSymbol{\N}{\mathbin}{AMSb}{"4E}
\DeclareMathSymbol{\Z}{\mathbin}{AMSb}{"5A}
\DeclareMathSymbol{\R}{\mathbin}{AMSb}{"52}
\DeclareMathSymbol{\Q}{\mathbin}{AMSb}{"51}
\DeclareMathSymbol{\I}{\mathbin}{AMSb}{"49}
\DeclareMathSymbol{\C}{\mathbin}{AMSb}{"43}

\begin{document}
\thispagestyle{empty}   %% skips page number on the first page

\begin{center}
\large
\textbf{CSCI 2300 --- Algo \\
Homework 2}
\\Hayden Fuller
\end{center}

\begin{itemize}

\item \textbf{Q1} 
\\Is $4^{1536}\equiv9^{4824}\mod35$
\\$4^4\mod35=256\mod35=11$
\\$9^2\mod35=81\mod35=11$
\\$4^{1536}\equiv4^{4*384}\equiv11^384\mod35$
\\$9^{4824}\equiv=9^{2*2412}\equiv11^{2412}\equiv11^{6*384+108}\equiv11^{6*384}*11^{108}\mod35$
\\$11^{108}\equiv121^{54}\equiv6^{54}\equiv36^{27}\equiv1^27\mod35=1$
\\$11^{6*384}\equiv11^{384}\mod35$
\\
\\Yes, $4^{1536}\equiv9^{4824}\mod35$

\vspace{0.1in}

\item \textbf{Q2} 
\\$x^{86}\equiv6\mod29$
\\$x^{28}\equiv1\mod29$
\\$x^{86}\equiv x^2\mod29$
\\$x^2\equiv6\mod29$
\\$x^2\equiv64\mod29$
\\$x=8$
\vspace{0.1in}

\item \textbf{Q3} 
\\Prove that GCD$(F_{n+1},F_n)=1$, for $n\ge1$, where $F_n$ is the n-th Fibonacci element.
\\We prove this with induction
\\Base: $n=1$, $F_2=1$, $F_1=1$, GCD$(F_2,F_1)=$GCD$(1,1)=1$.
\\Induction: Assume GCD$(F_{n+1},F_n)=1$
\\$F_{n+2}=F_{n+1}+F_n$, $F_n=F_{n+2}-F_{n+1}$
\\GCD$(F_{n+1},F_{n})=$GCD$(F_{n+1},F_{n+2}-F_{n+1})=$GCD$(F_{n+1},F_{n+2})=$GCD$(F_{n+2},F_{n+1})=1$
\\GCD$(F_{n+1},F_n)=1\imp$GCD$(F_{n+2},F_{n+1})=1$
\\therefore, GCD$(F_{n+1},F_n)=1$, for $n\ge1$ $\hfill{\blacksquare}$

\end{itemize}

\end{document}
























